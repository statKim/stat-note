\chapter{Math symbol test}

\section{Theorem symbol}

\begin{theorem}[Pythagorean theorem]
\label{pythagorean}
This is a theorem about right triangles and can be summarised in the next 
equation 
\[ x^2 + y^2 = z^2 \]
\end{theorem}
\begin{proof}
	dkdkdk
\end{proof}

And a consequence of theorem \ref{pythagorean} is the statement in the next 
corollary.

\begin{corollary}
There's no right rectangle whose sides measure 3cm, 4cm, and 6cm.
\end{corollary}

You can reference theorems such as \ref{pythagorean} when a label is assigned.

\begin{proposition}[Consistnecy]
ddfafa
\end{proposition}

\begin{lemma}
Given two line segments whose lengths are \(a\) and \(b\) respectively there is a 
real number \(r\) such that \(b=ra\).
\end{lemma}

\begin{remark}
This statement is true, I guess.
\end{remark}

\begin{definition}[Fibration]
A fibration is a mapping between two topological spaces that has the homotopy lifting property for every space \(X\).
\end{definition}

\begin{example}[Continuous prob]
This statement is true, I guess.
\end{example}