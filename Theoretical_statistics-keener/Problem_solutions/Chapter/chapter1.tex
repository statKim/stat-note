\chapter{Probability and Measure}

%% # 1.1
\section{Problem 1.1}
Prove (1.1). If measurable sets $B_n, ~ n \ge 1$, are increasing, with $B = \union_{n=1}^\infty B_n$, called the limit of the sequence, then
$$ \mu(B) = \lim_{n \rightarrow \infty} \mu(B_n). $$
\begin{proof}[\underline{\textbf{Solution}}] $\newline$
    First, we showed that $A_n$'s are disjoint.
    If $j < k$, then $B_j \subseteq B_{k-1} $. \\
    Since $A_j \subset B_j \subseteq B_{k-1}$ and $A_k \subset B_{k-1}^c$, $A_j$ and $A_k$ are disjoint.
    
    Also $B_n = \union_{j=1}^n A_j$ and $\union_{n=1}^\infty A_n = B$,
    $$ \mu(B) = \sum_{i=1}^\infty \mu(A_i) = \lim_{n \rightarrow \infty} \sum_{i=1}^n \mu(A_i) =  \lim_{n \rightarrow \infty} \mu\left(\union_{i=1}^n A_i\right) = \lim_{n \rightarrow \infty} \mu(B_n). $$
\end{proof}


%% # 1.8
\section{Problem 1.8}
Prove {\em Boole's inequality}: For any events $B_1, B_2, \dots $,
$$ P\left(\union_{i \ge 1} B_i\right) \le \sum_{i \ge 1}P(B_i). $$
\begin{proof}[\underline{\textbf{Solution}}] $\newline$
    Let $B = \union_{i=1}^\infty B_i$, then $1_B \le \sum 1_{B_i}$. Then by Fubini's theorem,
    $$ P(B) = \int 1_B dP \le \int \sum 1_{B_i} dP = \sum \int 1_{B_i} dP = \sum P(B_i).$$
\end{proof}


%% # 1.10
\section{Problem 1.10}
Let $\mu$ and $\nu$ be measures on $(\calE, \calB)$.
\begin{itemize}
	\item[a)] Show that the sum $\eta$ defined by $\eta(B) = \mu(B) + \nu(B)$ is also a measure.
	\begin{proof}[\underline{\textbf{Solution}}] $\newline$
		We should check following 2 conditions.
		\begin{itemize}
			\item[(i)] For arbitrary set $A \in \calB$, $\mu(A) \ge 0$ and $\nu(A) \ge 0$. $\Rightarrow \eta(A) = \mu(A) + \nu(A) \ge 0.$ \\
            		$\therefore \eta: \calB \rightarrow [0, \infty].$
			\item[(ii)] For disjoint set $B_1, B_2, \dots \subset \calB$, then
            	\begin{align*}
            		\eta \left( \union_{i=1}^\infty B_i \right) &= \mu\left(\union B_i\right) + \nu\left(\union B_i\right) \\
            									  &= \sum \mu(B_i) + \sum \nu(B_i) \\
            									  &= \sum \left\{ \mu(B_i) + \nu(B_i) \right\} \\
            									  &= \sum \eta(B_i)
            	\end{align*}
		\end{itemize}
            	$\therefore \eta$ is a measure.
	\end{proof}
	
	\item[b)] If $f$ is a non-negative measurable function, show that
	$$\int f d\eta = \int f d\mu + \int f d\nu.$$
	\begin{proof}[\underline{\textbf{Solution}}] $\newline$
		We show it by 2 stage.
		\begin{itemize}
			\item[(i)] Let $f = \sum_{i=1}^n a_i 1_{A_i}$, the non-negative simple function. Then,
                    		\begin{align*}
                    			\int f d\eta &= \int \sum_{i=1}^n a_i 1_{A_i} d\eta = \sum a_i\eta(A_i) \\
                    					 &= \sum a_i \left\{ \mu(A_i) + \nu(A_i) \right\}  = \int f d\mu + \int f d\nu.
                    		\end{align*}
			\item[(ii)] For general case, let $f_n$ is the sequence of non-negative simple functions increasing to $f$. (i.e. $f_1 \le f_2 \le \dots \le f$)
                    		\begin{align*}
                    			\int f d\eta &= \lim_{n \to \infty} \int f_n d\eta = \lim_{n \to \infty} \left( \int f_n d\mu + \int f_n d\nu \right)  \\
                    					&= \lim_{n \to \infty} \int f_n d\mu + \lim_{n \to \infty} \int f_n d\nu = \int f d\mu + \int f d\nu.
                    		\end{align*}
		\end{itemize}
	\end{proof}
\end{itemize}


%% 1.11
\section{Problem 1.11}
Suppose $f$ is the simple function $1_{(1/2, \pi]} + 21_{(1, 2]}$, and let $\mu$ be a measure on $\bbR$ with $\mu\{(0,a^2]\} = a, ~ a > 0$. Evaluate $\int f d\mu.$
\begin{proof}[\underline{\textbf{Solution}}] $\newline$
By the integral of simple function and the finite additivity, it can be simply computed as
\begin{align*}
	\int f d\mu &= \mu\{(1/2, \pi] \} + 2\mu\{(1,2]\} \\
			&= \left[ \mu\{(0, \pi]\} - \mu\{(1/2, \pi] \} \right] + 2\left[ \mu\{ (0, 2] \} -  \mu\{(1,2]\} \right]  ~~ \text{(finite additivity)}\\
			&= \left( \sqrt{\pi} - 1/\sqrt{2} \right) + 2\left( \sqrt{2} - 1 \right)
\end{align*}
\end{proof}


%% 1.12
\section{Problem 1.12}
Suppose that $\mu\{ (0, a) \} = a^2$ for $a > 0$ and that $f$ is defined by
$$ f(x) = \begin{cases}
    0, ~~~ x \le 0, \\
    1, ~~~ 0 < x < 2, \\
    \pi, ~~~ 2 \ge x < 5, \\
    0, ~~~ x \ge 5.
\end{cases}$$
Compute $\int f d\mu.$
\begin{proof}[\underline{\textbf{Solution}}] $\newline$
Since $f = 1_{(0, 2)} + 1_{[2, 5)}$ is simple, the integral can be computed as
\begin{align*}
	\int f d\mu &= \mu\{ (0, 2) \} + \pi \mu\{ [2, 5) \} \\
			&= \mu\{ (0, 2) \} + \pi \left[ \mu\{ (0, 5) \} - \mu\{ (0, 2) \} \right] \\
			&= 4 - \pi(25-4)
\end{align*}
\end{proof}


%% 1.13
\section{Problem 1.13}
Define the function $f$ by
$$ f(x) = \begin{cases}
    x, ~~~ 0 \le x \le 1, \\
    0, ~~~ \text{otherwise.}
\end{cases}$$
Find simple functions $f_1 \le f_2 \le \cdots$ increasing to $f$ (i.e. $f(x) = \limn f_n(x)$ for all $x \in \bbR$).
Let $\mu$ be Lebesgue measure on $\bbR$.
Using our formal definition of an integral and the faProblemct that $\mu\big( (a, b] \big) = b-a$ whenever $b > a$ (this might be used to formally define Lebesgue measure), show that $\int f d\mu = 1/2$.
\begin{proof}[\underline{\textbf{Solution}}] $\newline$
Let $f_n = \lfloor 2^nx \rfloor / 2^n$ for $0 < x \le 1$ and 0 otherwise. ($\lfloor y \rfloor$ is a floor function.)
Then,
\begin{align*}
    f_1(x) &= \lfloor 2x \rfloor / 2 = \begin{cases}
    0, ~~~ x < 1/2, \\
    1/2, ~~~ 1/2 \le x < 1, \\
    1, ~~~ x = 1
    \end{cases} \\
    f_2(x) &= \lfloor 2^2x \rfloor / 2^2 = \begin{cases}
    0, ~~~ x < 1/2^2, \\
    1/2^2, ~~~ 1/2^2 \le x < 2/2^2, \\
    2/2^2, ~~~ 2/2^2 \le x < 3/2^2, \\
    3/2^2, ~~~ 3/2^2 \le x < 1, \\
    1, ~~~ x = 1
    \end{cases}\\
    &\vdots \\
    f_n(x) &= \lfloor 2^nx \rfloor / 2^n = \begin{cases}
    0, ~~~ x < 1/2^n, \\
    1/2^n, ~~~ 1/2^n \le x < 2/2^n, \\
    \vdots \\
    (2^n-1)/2^n, ~~~ (2^n-1)/2^n \le x < 1, \\
    1, ~~~ x = 1
    \end{cases}\\
\end{align*}
\begin{align*}
\therefore \int f_n d\mu &= \frac{1}{2^n} \left( \frac{1}{2^n} + \frac{2}{2^n} + \cdots + \frac{2^n-1}{2^n} \right) \\
		    &= \frac{1+2+\cdots + (2^n-1)}{4^n} \\
		    &= \frac{2^n(2^n-1)}{2 \cdot 4^n} \\
		    &\longrightarrow \frac{1}{2}.
\end{align*}

{\color{blue} \underline{My solution} \\
Let the simple function $f_n = \sum_{i=1}^n \frac{i}{n}1_{(\frac{i-1}{n}, \frac{i}{n} ]}$. Then,
$$ \int f_n d\mu = \sum_{i=1}^n \frac{i}{n} \frac{1}{n} = \frac{1}{n^2} \frac{n(n+1)}{2} \longrightarrow \frac{1}{2}.$$
}
\end{proof}


%% 1.16
\section{Problem 1.16}
Define $F(a-) = \lim_{x\uparrow a}F(x)$. Then, if $F$ is non-decreasing, $F(a-) = \limn F(a-1/n)$.
Use (1.1)[Continuity of measure] to show that if a random variable $X$ has cumulative distribution function $F_X$,
$$ P(X < a) = F_X(a-). $$
Also, show that
$$ P(X=a) = F_X(a) - F_X(a-). $$
\begin{proof}[\underline{\textbf{Solution}}] $\newline$
\begin{itemize}
    \item[(i)] Let $B_n = \{ X \le a-1/n \}$ and $\union_{n=1}^\infty B_n = \{ X < a \}$.
    By continuity of measure,
    $$ P(B) = P(X < a) = \limn P(X \le a-1/n) = \limn F_X(a-1/n) = F_X(a-). $$
    \item[(ii)] Since $\{ X<a \}$ and $\{X=a\}$ are disjoint with union $\{X \le a\}$,
    $$ P(X < a) + P(X = a) = P(X \le a).$$
    $$\therefore P(X = a) = F_X(a) - F_X(a-).$$
\end{itemize}
\end{proof}


%% 1.17
\section{Problem 1.17}
Suppose $X$ is a geometric random variable with mass function
$$ p(x) = P(X=x) = \theta(1-\theta)^x, ~~~ x= 0,1,\dots, $$
where $\theta \in (0,1)$ is a constant. Find the probability that $X$ is even.
\begin{proof}[\underline{\textbf{Solution}}]
\begin{align*}
P(X\text{ is even}) &= P(X=0) + P(X=2) + P(X=4) \cdots \\
			     &= \theta + \theta(1-\theta)^2 + \theta(1-\theta)^4 + \cdots \\
			     &= \frac{\theta}{1-(1-\theta)^2} \\
			     &= \frac{1}{2-\theta}
\end{align*}
\end{proof}


%% 1.18
\section{Problem 1.18}
Let $X$ be a function mapping $\calE$ into $\bbR$.
Recall that if $B$ is a subset of $\bbR$, then $X^{-1}(B) = \{e \in \calE : X(e) \in B\}.$
Use this definition to prove that
$$ X^{-1}(A \cap B) = X^{-1}(A) \cap X^{-1}(B), $$
$$ X^{-1}(A \cup B) = X^{-1}(A) \cup X^{-1}(B), $$
and
$$ X^{-1}\left(\union_{i=0}^\infty A_i \right) = \union_{i=0}^\infty X^{-1}(A_i). $$
\begin{proof}[\underline{\textbf{Solution}}] $\newline$
\begin{itemize}
\item[(i)] \begin{align*}
    e \in X^{-1}(A \cap B) &\Leftrightarrow X(e) \in A \cap B \\
    				&\Leftrightarrow X(e) \in A \text{ and } X(e) \in B \\
    				&\Leftrightarrow e \in X^{-1}(A) \text{ and } e \in X^{-1}(B) \\
    				&\Leftrightarrow e \in X^{-1}(A) \cap X^{-1}(B).
    \end{align*}

\item[(ii)] Similarly,
    \begin{align*}
    e \in X^{-1}(A \cup B) &\Leftrightarrow X(e) \in A \cup B \\
    				&\Leftrightarrow X(e) \in A \text{ or } X(e) \in B \\
    				&\Leftrightarrow e \in X^{-1}(A) \text{ or } e \in X^{-1}(B) \\
    				&\Leftrightarrow e \in X^{-1}(A) \cup X^{-1}(B).
    \end{align*}

\item[(iii)] \begin{align*}
    e \in X^{-1}(\union_{i=0}^\infty A_i) &\Leftrightarrow X(e) \in \union_{i=0}^\infty A_i \\
    				&\Leftrightarrow X(e) \in A_i \text{ for some } i \\
    				&\Leftrightarrow e \in X^{-1}(A_i) \text{ for some } i \\
    				&\Leftrightarrow e \in  \union_{i=0}^\infty X^{-1}(A_i).
    \end{align*}

\end{itemize}
\end{proof}



