\documentclass[10pt]{book}

\usepackage{fancyhdr}
\usepackage{extramarks}
\usepackage{amsmath}
\usepackage{amsthm}
\usepackage{amsfonts}
\usepackage{tikz}
%\usepackage[plain]{algorithm}
%\usepackage{algpseudocode}

\usetikzlibrary{automata,positioning}

%%%%%%%%%%%%%%%%%%%%%%%%
%%% Useful packages and commands
%%%%%%%%%%%%%%%%%%%%%%%%
% packages
\usepackage{amsmath,amssymb,amsthm}
%\usepackage{times}
%\usepackage{setspace}
\usepackage{indentfirst}
\usepackage{epsfig}
\usepackage{graphicx}
\usepackage{latexsym}
\usepackage{amscd}
\usepackage{multirow}
\usepackage{array}
\usepackage{caption}
\usepackage{rotating}
\usepackage{subfig}
\usepackage{color}
\usepackage{natbib}
\usepackage{lscape}
\usepackage{graphics}
\usepackage{enumerate}
%\usepackage{fancyvrb}
%\usepackage{mathtools}
\usepackage{verbatim}
\usepackage{afterpage}
%\usepackage[ruled,vlined]{algorithm2e}
\usepackage{hyperref}
\usepackage[flushleft]{threeparttable}
\usepackage{rotating}
\usepackage{kotex}   % for Korean

% new commands
%\DeclarePairedDelimiter\abs{\lvert}{\rvert}
%\DeclarePairedDelimiter\norm{lVert}{\rVert}
\long\def\comment#1{}

%\newtheorem*{thm}{Theorem}
\newtheorem{thm}{Theorem}[section]
\newtheorem{cor}[thm]{Corollary}
\newtheorem{lem}[thm]{Lemma}
\newcommand{\rb}[1]{\raisebox{-.5em}[0pt]{#1}}
% \renewcommand{\baselinestretch}{1.8}
\renewcommand{\mid}{\, | \ }
\newcommand{\eighth}{{\textstyle \frac{1}{8}}}

\def \bY { \mathbf{ Y } }
\def \bX { \mathbf{ X } }
\def \bU { \mathbf{ U } }
\def \bmu { \boldsymbol{ \mu } }
\def \bSigma { \boldsymbol{ \Sigma } }
\def \bphi { \boldsymbol{ \phi } }
\def \bepsilon { \boldsymbol{ \epsilon } }
\def \bD { \boldsymbol{\mathcal{D}} }

\newcommand{\eqdis}{\overset{\mathrm{d}}{=\joinrel=}}
\newcommand{\ba}{\mbox{\boldmath $a$}}
\newcommand{\bg}{\mbox{\boldmath $g$}}
\newcommand{\bx}{\mbox{\boldmath $x$}}
\newcommand{\by}{\mbox{\boldmath $y$}}
\newcommand{\bd}{\mbox{\boldmath $d$}}
\newcommand{\bff}{\mbox{\boldmath $f$}}
\newcommand{\bz}{\mbox{\boldmath $z$}}
\newcommand{\bu}{\mbox{\boldmath $u$}}
\newcommand{\bv}{\mbox{\boldmath $v$}}
\newcommand{\bW}{\mbox{\boldmath $W$}}
\newcommand{\bI}{\mbox{\boldmath $I$}}
\newcommand{\bJ}{\mbox{\boldmath $J$}}
\newcommand{\bL}{\mbox{\boldmath $L$}}
\newcommand{\bQ}{\mbox{\boldmath $Q$}}
\newcommand{\bZ}{\mbox{\boldmath $Z$}}
\newcommand{\bV}{\mbox{\boldmath $V$}}
\newcommand{\bG}{\mbox{\boldmath $G$}}
\newcommand{\bdm}{\begin{displaymath}}
\newcommand{\edm}{\end{displaymath}}
\newcommand{\bnu}{\mbox{\boldmath $\nu$}}
\newcommand{\btau}{\mbox{\boldmath $\tau$}}
\newcommand{\biota}{\mbox{\boldmath $\iota$}}
\newcommand{\bbeta}{\mbox{\boldmath $\beta$}}
\newcommand{\bomega}{\mbox{\boldmath $\omega$}}
\newcommand{\btheta}{\mbox{\boldmath $\theta$}}
\newcommand{\bep}{\mbox{\boldmath $\epsilon$}}
\newcommand{\bdelta}{\mbox{\boldmath $\delta$}}
\newcommand{\balpha}{\mbox{\boldmath $\alpha$}}
\newcommand{\bxi}{\mbox{\boldmath $\xi$}}
\newcommand{\bgamma}{\mbox{\boldmath $\gamma$}}
\newcommand{\bOmega}{\mbox{\boldmath $\Omega$}}
\newcommand{\bPi}{\mbox{\boldmath $\Pi$}}
\newcommand{\bzeta}{\mbox{\boldmath $\zeta$}}
\newcommand{\bpsi}{\mbox{\boldmath $\psi$}}
\newcommand{\bPsi}{\mbox{\boldmath $\Psi$}}
\newcommand{\bl}{\mbox{\boldmath $l$}}
\newcommand{\C}{{\rm Cov}}
\newcommand{\bH}{\bold H}
\newcommand{\blambda}{\mbox{\boldmath $\lambda$}}
\newcommand{\bbh}{\bld h}
\newcommand{\calA}{\mathcal{A}}
\newcommand{\calB}{\mathcal{B}}
\newcommand{\calE}{\mathcal{E}}
\newcommand{\calX}{\mathcal{X}}
\newcommand{\calY}{\mathcal{Y}}
\newcommand{\bbR}{\mathbb{R}}
\newcommand{\union}{\bigcup}
\newcommand{\intersect}{\bigcap}
\newcommand{\eqdef}{\overset{\mathrm{def}}{=}}
\newcommand{\limn}{\lim_{n \to \infty}}


%
% Basic Document Settings
%

\topmargin=-0.45in
\evensidemargin=0in
\oddsidemargin=0in
\textwidth=6.5in
\textheight=9.0in
\headsep=0.25in

\linespread{1.2}

\pagestyle{fancy}
\lhead{\leftmark}
%\chead{\hmwkClass\ (\hmwkClassInstructor\ \hmwkClassTime): \hmwkTitle}
\rhead{\authorName}
%\lfoot{\lastxmark}
\cfoot{\thepage}

\renewcommand\headrulewidth{0.4pt}
\renewcommand\footrulewidth{0.4pt}

\setlength\parindent{0pt}

%%
%% Create Problem Sections
%%
%
%\newcommand{\enterProblemHeader}[1]{
%    \nobreak\extramarks{}{Problem \arabic{#1} continued on next page\ldots}\nobreak{}
%    \nobreak\extramarks{Problem \arabic{#1} (continued)}{Problem \arabic{#1} continued on next page\ldots}\nobreak{}
%}
%
%\newcommand{\exitProblemHeader}[1]{
%    \nobreak\extramarks{Problem \arabic{#1} (continued)}{Problem \arabic{#1} continued on next page\ldots}\nobreak{}
%    \stepcounter{#1}
%    \nobreak\extramarks{Problem \arabic{#1}}{}\nobreak{}
%}
%
%\setcounter{secnumdepth}{0}
%\newcounter{partCounter}
%\newcounter{homeworkProblemCounter}
%\setcounter{homeworkProblemCounter}{1}
%\nobreak\extramarks{Problem \arabic{homeworkProblemCounter}}{}\nobreak{}
%
%%
%% Homework Problem Environment
%%
%% This environment takes an optional argument. When given, it will adjust the
%% problem counter. This is useful for when the problems given for your
%% assignment aren't sequential. See the last 3 problems of this template for an
%% example.
%%
%\newenvironment{homeworkProblem}[1][-1]{
%    \ifnum#1>0
%        \setcounter{homeworkProblemCounter}{#1}
%    \fi
%    \section{Problem \arabic{homeworkProblemCounter}}
%    \setcounter{partCounter}{1}
%    \enterProblemHeader{homeworkProblemCounter}
%}{
%    \exitProblemHeader{homeworkProblemCounter}
%}

%
% Homework Details
%   - Title
%   - Due date
%   - Class
%   - Section/Time
%   - Instructor
%   - Author
%

%\newcommand{\hmwkTitle}{}
%\newcommand{\hmwkClass}{Theoretical Statistics: Topics for a Core Course\\ Solution}
%\newcommand{\hmwkClassTime}{}
%\newcommand{\hmwkClassInstructor}{Robert W. Keener}

\newcommand{\courseTitle}{Theoretical Statistics: Topics for a Core Course\\ Problem Solutions}
\newcommand{\bookAuthor}{Robert W. Keener}
\newcommand{\authorName}{Hyunsung Kim}



%
% Title Page
%

\title{
    \vspace{1.5in}
%    \textmd{\textbf{\hmwkClass:\ \hmwkTitle}}\\
    \textmd{\textbf{\courseTitle}}\\
%    \normalsize\vspace{0.1in}\hmwkDueDate\\
    \vspace{0.1in}\large{\textit{\bookAuthor}}
    \vspace{2in}
}
\date{Update: \today}
\author{
    {\sc \authorName} \\
    Department of Statistics\\
    Chung-Ang University
}



\renewcommand{\part}[1]{\textbf{\large Part \Alph{partCounter}}\stepcounter{partCounter}\\}

%
% Various Helper Commands
%

% Useful for algorithms
\newcommand{\alg}[1]{\textsc{\bfseries \footnotesize #1}}

% For derivatives
\newcommand{\deriv}[1]{\frac{\mathrm{d}}{\mathrm{d}x} (#1)}

% For partial derivatives
\newcommand{\pderiv}[2]{\frac{\partial}{\partial #1} (#2)}

% Integral dx
\newcommand{\dx}{\mathrm{d}x}
\newcommand{\dmu}{\mathrm{d}\mu}
\newcommand{\dP}{\mathrm{d}P}

% Alias for the Solution section header
\newcommand{\solution}{\textbf{\large Solution}}

% Probability commands: Expectation, Variance, Covariance, Bias
\newcommand{\E}{\mathrm{E}}
\newcommand{\Var}{\mathrm{Var}}
\newcommand{\Cov}{\mathrm{Cov}}
\newcommand{\Bias}{\mathrm{Bias}}

% Specify level of toc(table of content)
\setcounter{tocdepth}{0}

% Specify level of Section number
\setcounter{secnumdepth}{0}

% Setting for equation breaks
\allowdisplaybreaks

% Setting for blank page appeared when begin new chapter
\let\cleardoublepage=\clearpage


\newtheorem{theorem}{Theorem}[section]
\newtheorem{proposition}[theorem]{Proposition}
\newtheorem{corollary}[theorem]{Corollary}
\newtheorem{lemma}[theorem]{Lemma}
\newtheorem{definition}[theorem]{Definition}

\theoremstyle{definition}
\newtheorem{remark}[theorem]{Remark}
\newtheorem{example}[theorem]{Example}
%\newtheorem{problem}[theorem]{Problem}


%%% Main part
\begin{document}

% Title page
\maketitle

% Table of Contents
\tableofcontents

% Preface
\addcontentsline{toc}{chapter}{Preface}
\chapter*{Preface}
%\section{Preface}
This note contains summaries of the textbook, {\em Theoretical Statistics: Topics for a Core Course}, and it was created by Hyunsung Kim, who is a Ph.D. student.
I made it when I study a theoretical statistics based on this textbook on my own.

\section*{Textbook}
\begin{itemize}
	\item Keener, {\em Theoretical Statistics: Topics for a Core Course}.
\end{itemize}

\section*{Reference}
\begin{itemize}
	\item Durrett, {\em Probability: Theory and Examples, 5th edition}.
	\item Royden, {\em Real Analysis, 4th edition}.
\end{itemize}

% Test file
%\chapter{Math symbol test}

\section{Theorem symbol}

\begin{theorem}[Pythagorean theorem]
\label{pythagorean}
This is a theorem about right triangles and can be summarised in the next 
equation 
\[ x^2 + y^2 = z^2 \]
\end{theorem}
\begin{proof}
	dkdkdk
\end{proof}

And a consequence of theorem \ref{pythagorean} is the statement in the next 
corollary.

\begin{corollary}
There's no right rectangle whose sides measure 3cm, 4cm, and 6cm.
\end{corollary}

You can reference theorems such as \ref{pythagorean} when a label is assigned.

\begin{proposition}[Consistnecy]
ddfafa
\end{proposition}

\begin{lemma}
Given two line segments whose lengths are \(a\) and \(b\) respectively there is a 
real number \(r\) such that \(b=ra\).
\end{lemma}

\begin{remark}
This statement is true, I guess.
\end{remark}

\begin{definition}[Fibration]
A fibration is a mapping between two topological spaces that has the homotopy lifting property for every space \(X\).
\end{definition}

\begin{example}[Continuous prob]
This statement is true, I guess.
\end{example}

% Chapter 1. Probability and Measure
\chapter{Probability and Measure}

%% # 1.1
\section{Problem 1.1}
Prove (1.1). If measurable sets $B_n, ~ n \ge 1$, are increasing, with $B = \union_{n=1}^\infty B_n$, called the limit of the sequence, then
$$ \mu(B) = \lim_{n \rightarrow \infty} \mu(B_n). $$
\begin{proof}[\underline{\textbf{Solution}}] $\newline$
    First, we showed that $A_n$'s are disjoint.
    If $j < k$, then $B_j \subseteq B_{k-1} $. \\
    Since $A_j \subset B_j \subseteq B_{k-1}$ and $A_k \subset B_{k-1}^c$, $A_j$ and $A_k$ are disjoint.
    
    Also $B_n = \union_{j=1}^n A_j$ and $\union_{n=1}^\infty A_n = B$,
    $$ \mu(B) = \sum_{i=1}^\infty \mu(A_i) = \lim_{n \rightarrow \infty} \sum_{i=1}^n \mu(A_i) =  \lim_{n \rightarrow \infty} \mu\left(\union_{i=1}^n A_i\right) = \lim_{n \rightarrow \infty} \mu(B_n). $$
\end{proof}


%% # 1.8
\section{Problem 1.8}
Prove {\em Boole's inequality}: For any events $B_1, B_2, \dots $,
$$ P\left(\union_{i \ge 1} B_i\right) \le \sum_{i \ge 1}P(B_i). $$
\begin{proof}[\underline{\textbf{Solution}}] $\newline$
    Let $B = \union_{i=1}^\infty B_i$, then $1_B \le \sum 1_{B_i}$. Then by Fubini's theorem,
    $$ P(B) = \int 1_B dP \le \int \sum 1_{B_i} dP = \sum \int 1_{B_i} dP = \sum P(B_i).$$
\end{proof}


%% # 1.10
\section{Problem 1.10}
Let $\mu$ and $\nu$ be measures on $(\calE, \calB)$.
\begin{itemize}
	\item[a)] Show that the sum $\eta$ defined by $\eta(B) = \mu(B) + \nu(B)$ is also a measure.
	\begin{proof}[\underline{\textbf{Solution}}] $\newline$
		We should check following 2 conditions.
		\begin{itemize}
			\item[(i)] For arbitrary set $A \in \calB$, $\mu(A) \ge 0$ and $\nu(A) \ge 0$. $\Rightarrow \eta(A) = \mu(A) + \nu(A) \ge 0.$ \\
            		$\therefore \eta: \calB \rightarrow [0, \infty].$
			\item[(ii)] For disjoint set $B_1, B_2, \dots \subset \calB$, then
            	\begin{align*}
            		\eta \left( \union_{i=1}^\infty B_i \right) &= \mu\left(\union B_i\right) + \nu\left(\union B_i\right) \\
            									  &= \sum \mu(B_i) + \sum \nu(B_i) \\
            									  &= \sum \left\{ \mu(B_i) + \nu(B_i) \right\} \\
            									  &= \sum \eta(B_i)
            	\end{align*}
		\end{itemize}
            	$\therefore \eta$ is a measure.
	\end{proof}
	
	\item[b)] If $f$ is a non-negative measurable function, show that
	$$\int f d\eta = \int f d\mu + \int f d\nu.$$
	\begin{proof}[\underline{\textbf{Solution}}] $\newline$
		We show it by 2 stage.
		\begin{itemize}
			\item[(i)] Let $f = \sum_{i=1}^n a_i 1_{A_i}$, the non-negative simple function. Then,
                    		\begin{align*}
                    			\int f d\eta &= \int \sum_{i=1}^n a_i 1_{A_i} d\eta = \sum a_i\eta(A_i) \\
                    					 &= \sum a_i \left\{ \mu(A_i) + \nu(A_i) \right\}  = \int f d\mu + \int f d\nu.
                    		\end{align*}
			\item[(ii)] For general case, let $f_n$ is the sequence of non-negative simple functions increasing to $f$. (i.e. $f_1 \le f_2 \le \dots \le f$)
                    		\begin{align*}
                    			\int f d\eta &= \lim_{n \to \infty} \int f_n d\eta = \lim_{n \to \infty} \left( \int f_n d\mu + \int f_n d\nu \right)  \\
                    					&= \lim_{n \to \infty} \int f_n d\mu + \lim_{n \to \infty} \int f_n d\nu = \int f d\mu + \int f d\nu.
                    		\end{align*}
		\end{itemize}
	\end{proof}
\end{itemize}


%% 1.11
\section{Problem 1.11}
Suppose $f$ is the simple function $1_{(1/2, \pi]} + 21_{(1, 2]}$, and let $\mu$ be a measure on $\bbR$ with $\mu\{(0,a^2]\} = a, ~ a > 0$. Evaluate $\int f d\mu.$
\begin{proof}[\underline{\textbf{Solution}}] $\newline$
By the integral of simple function and the finite additivity, it can be simply computed as
\begin{align*}
	\int f d\mu &= \mu\{(1/2, \pi] \} + 2\mu\{(1,2]\} \\
			&= \left[ \mu\{(0, \pi]\} - \mu\{(1/2, \pi] \} \right] + 2\left[ \mu\{ (0, 2] \} -  \mu\{(1,2]\} \right]  ~~ \text{(finite additivity)}\\
			&= \left( \sqrt{\pi} - 1/\sqrt{2} \right) + 2\left( \sqrt{2} - 1 \right)
\end{align*}
\end{proof}


%% 1.12
\section{Problem 1.12}
Suppose that $\mu\{ (0, a) \} = a^2$ for $a > 0$ and that $f$ is defined by
$$ f(x) = \begin{cases}
    0, ~~~ x \le 0, \\
    1, ~~~ 0 < x < 2, \\
    \pi, ~~~ 2 \ge x < 5, \\
    0, ~~~ x \ge 5.
\end{cases}$$
Compute $\int f d\mu.$
\begin{proof}[\underline{\textbf{Solution}}] $\newline$
Since $f = 1_{(0, 2)} + 1_{[2, 5)}$ is simple, the integral can be computed as
\begin{align*}
	\int f d\mu &= \mu\{ (0, 2) \} + \pi \mu\{ [2, 5) \} \\
			&= \mu\{ (0, 2) \} + \pi \left[ \mu\{ (0, 5) \} - \mu\{ (0, 2) \} \right] \\
			&= 4 - \pi(25-4)
\end{align*}
\end{proof}


%% 1.13
\section{Problem 1.13}
Define the function $f$ by
$$ f(x) = \begin{cases}
    x, ~~~ 0 \le x \le 1, \\
    0, ~~~ \text{otherwise.}
\end{cases}$$
Find simple functions $f_1 \le f_2 \le \cdots$ increasing to $f$ (i.e. $f(x) = \limn f_n(x)$ for all $x \in \bbR$).
Let $\mu$ be Lebesgue measure on $\bbR$.
Using our formal definition of an integral and the faProblemct that $\mu\big( (a, b] \big) = b-a$ whenever $b > a$ (this might be used to formally define Lebesgue measure), show that $\int f d\mu = 1/2$.
\begin{proof}[\underline{\textbf{Solution}}] $\newline$
Let $f_n = \lfloor 2^nx \rfloor / 2^n$ for $0 < x \le 1$ and 0 otherwise. ($\lfloor y \rfloor$ is a floor function.)
Then,
\begin{align*}
    f_1(x) &= \lfloor 2x \rfloor / 2 = \begin{cases}
    0, ~~~ x < 1/2, \\
    1/2, ~~~ 1/2 \le x < 1, \\
    1, ~~~ x = 1
    \end{cases} \\
    f_2(x) &= \lfloor 2^2x \rfloor / 2^2 = \begin{cases}
    0, ~~~ x < 1/2^2, \\
    1/2^2, ~~~ 1/2^2 \le x < 2/2^2, \\
    2/2^2, ~~~ 2/2^2 \le x < 3/2^2, \\
    3/2^2, ~~~ 3/2^2 \le x < 1, \\
    1, ~~~ x = 1
    \end{cases}\\
    &\vdots \\
    f_n(x) &= \lfloor 2^nx \rfloor / 2^n = \begin{cases}
    0, ~~~ x < 1/2^n, \\
    1/2^n, ~~~ 1/2^n \le x < 2/2^n, \\
    \vdots \\
    (2^n-1)/2^n, ~~~ (2^n-1)/2^n \le x < 1, \\
    1, ~~~ x = 1
    \end{cases}\\
\end{align*}
\begin{align*}
\therefore \int f_n d\mu &= \frac{1}{2^n} \left( \frac{1}{2^n} + \frac{2}{2^n} + \cdots + \frac{2^n-1}{2^n} \right) \\
		    &= \frac{1+2+\cdots + (2^n-1)}{4^n} \\
		    &= \frac{2^n(2^n-1)}{2 \cdot 4^n} \\
		    &\longrightarrow \frac{1}{2}.
\end{align*}

{\color{blue} \underline{My solution} \\
Let the simple function $f_n = \sum_{i=1}^n \frac{i}{n}1_{(\frac{i-1}{n}, \frac{i}{n} ]}$. Then,
$$ \int f_n d\mu = \sum_{i=1}^n \frac{i}{n} \frac{1}{n} = \frac{1}{n^2} \frac{n(n+1)}{2} \longrightarrow \frac{1}{2}.$$
}
\end{proof}


%% 1.16
\section{Problem 1.16}
Define $F(a-) = \lim_{x\uparrow a}F(x)$. Then, if $F$ is non-decreasing, $F(a-) = \limn F(a-1/n)$.
Use (1.1)[Continuity of measure] to show that if a random variable $X$ has cumulative distribution function $F_X$,
$$ P(X < a) = F_X(a-). $$
Also, show that
$$ P(X=a) = F_X(a) - F_X(a-). $$
\begin{proof}[\underline{\textbf{Solution}}] $\newline$
\begin{itemize}
    \item[(i)] Let $B_n = \{ X \le a-1/n \}$ and $\union_{n=1}^\infty B_n = \{ X < a \}$.
    By continuity of measure,
    $$ P(B) = P(X < a) = \limn P(X \le a-1/n) = \limn F_X(a-1/n) = F_X(a-). $$
    \item[(ii)] Since $\{ X<a \}$ and $\{X=a\}$ are disjoint with union $\{X \le a\}$,
    $$ P(X < a) + P(X = a) = P(X \le a).$$
    $$\therefore P(X = a) = F_X(a) - F_X(a-).$$
\end{itemize}
\end{proof}


%% 1.17
\section{Problem 1.17}
Suppose $X$ is a geometric random variable with mass function
$$ p(x) = P(X=x) = \theta(1-\theta)^x, ~~~ x= 0,1,\dots, $$
where $\theta \in (0,1)$ is a constant. Find the probability that $X$ is even.
\begin{proof}[\underline{\textbf{Solution}}]
\begin{align*}
P(X\text{ is even}) &= P(X=0) + P(X=2) + P(X=4) \cdots \\
			     &= \theta + \theta(1-\theta)^2 + \theta(1-\theta)^4 + \cdots \\
			     &= \frac{\theta}{1-(1-\theta)^2} \\
			     &= \frac{1}{2-\theta}
\end{align*}
\end{proof}


%% 1.18
\section{Problem 1.18}
Let $X$ be a function mapping $\calE$ into $\bbR$.
Recall that if $B$ is a subset of $\bbR$, then $X^{-1}(B) = \{e \in \calE : X(e) \in B\}.$
Use this definition to prove that
$$ X^{-1}(A \cap B) = X^{-1}(A) \cap X^{-1}(B), $$
$$ X^{-1}(A \cup B) = X^{-1}(A) \cup X^{-1}(B), $$
and
$$ X^{-1}\left(\union_{i=0}^\infty A_i \right) = \union_{i=0}^\infty X^{-1}(A_i). $$
\begin{proof}[\underline{\textbf{Solution}}] $\newline$
\begin{itemize}
\item[(i)] \begin{align*}
    e \in X^{-1}(A \cap B) &\Leftrightarrow X(e) \in A \cap B \\
    				&\Leftrightarrow X(e) \in A \text{ and } X(e) \in B \\
    				&\Leftrightarrow e \in X^{-1}(A) \text{ and } e \in X^{-1}(B) \\
    				&\Leftrightarrow e \in X^{-1}(A) \cap X^{-1}(B).
    \end{align*}

\item[(ii)] Similarly,
    \begin{align*}
    e \in X^{-1}(A \cup B) &\Leftrightarrow X(e) \in A \cup B \\
    				&\Leftrightarrow X(e) \in A \text{ or } X(e) \in B \\
    				&\Leftrightarrow e \in X^{-1}(A) \text{ or } e \in X^{-1}(B) \\
    				&\Leftrightarrow e \in X^{-1}(A) \cup X^{-1}(B).
    \end{align*}

\item[(iii)] \begin{align*}
    e \in X^{-1}(\union_{i=0}^\infty A_i) &\Leftrightarrow X(e) \in \union_{i=0}^\infty A_i \\
    				&\Leftrightarrow X(e) \in A_i \text{ for some } i \\
    				&\Leftrightarrow e \in X^{-1}(A_i) \text{ for some } i \\
    				&\Leftrightarrow e \in  \union_{i=0}^\infty X^{-1}(A_i).
    \end{align*}

\end{itemize}
\end{proof}





% Chapter 2. Exponential Families
% Chapter 2. Exponential Families
\chapter{Exponential Families}

%% # 2.1
\section{Problem 2.1}
Consider independent Bernoulli trials with success probability $p$ and let $X$ be the number of failures before the first success.
Then $P(X=x) = p(1-p)^x$, for $x = 0,1,\dots $, and $X$ has the geometric distribution with parameter $p$, introduced in Problem 1.17.
\begin{itemize}
	\item[a)] Show that the geometric distributions form an exponential family.
            	\begin{proof}[\underline{\textbf{Solution}}]
                		$$ P(X=x) = p(1-p)^x = \exp\big\{ x\log(1-p) - (-\log p) \big\}. $$
			$\therefore$ the geometric distribution is an exponential family.
            	\end{proof}
	
	\item[b)] Write the densities for the family in canonical form, identifying the canonical parameter $\eta$, and the function $A(\eta)$.
            	\begin{proof}[\underline{\textbf{Solution}}] $\newline$
                		Let $\eta(p) = \log(1-p).$ Then, $ P(X=x) = \exp\big\{ \eta x - \big(-\log(1-e^\eta)\big) \big\}. $ \\
			$\therefore A(\eta) = -\log(1-e^\eta)$ with $T(x) = x$.
            	\end{proof}
	
	\item[c)] Find the mean of the geometric distribution using a differential identity.
            	\begin{proof}[\underline{\textbf{Solution}}] 
                		$$ E_\eta(T) = A'(\eta) = \frac{e^\eta}{1-e^\eta} = \frac{1-p}{p}. $$
            	\end{proof}	
	
	\item[d)] Suppose $X_1, \dots, X_n$ are i.i.d. from a geometric distribution. Show that the joint distributions form an exponential family, and find the mean and variance of $T$.
            	\begin{proof}[\underline{\textbf{Solution}}] $\newline$
                		$$ P(X_1 = x_1,\dots, X_n = x_n) = \prod_{i=1}^n P(X_i = x_i) = p^n(1-p)^{\sum_{i=1}^n x_i} = \exp\big\{ \sum_ix_i\log(1-p) - (-n\log p) \big\}. $$
		Therefore, the joint distribution is also exponential family.\\
		Now, let $\eta = \log(1-p)$. Then, the canonical exponential family is obtained with $A(\eta) = -n\log(1-e^\eta)$ and $T=\sum_i x_i$. 
		$$ ET = \kappa_1 = A'(\eta) = \frac{ne^\eta}{1-e^\eta} = \frac{n(1-p)}{p}, $$
		$$ Var(T) = \kappa_2 = A''(\eta) = \frac{ne^\eta(1-e^\eta)+ne^{2\eta}}{(1-e^\eta)^2} = \frac{np(1-p)+n(1-p)^2}{p^2} = \frac{n(1-p)}{p^2}. $$
            	\end{proof}
\end{itemize}


%% # 2.2
\section{Problem 2.2}
Determine the canonical parameter space $\Xi$, and find densities for the one-parameter exponential family with $\mu$ Lebesgue measure on $\bbR^2$, $h(x,y) = \exp\big[ -(x^2+y^2)/2 \big] / (2\pi)$, and $T(x,y) = xy$.

\begin{proof}[\underline{\textbf{Solution}}] $\newline$
	By definition of the canonical exponential family, the pdf be represented as
	$$ p(x,y) = \exp\big(\eta T(x,y) - A(\eta)\big) h(x,y) = \exp\big\{\eta xy - A(\eta) - (x^2+y^2)/2 \big\} / (2\pi).$$
	Thus,
	\begin{align*}
		\int p(x,y)d\mu(x,y) = 1 &\Longleftrightarrow \int \int \exp\big\{\eta xy - A(\eta) - (x^2+y^2)/2 \big\} / (2\pi) dxdy = 1 \\
						 &\Longleftrightarrow e^{A(\eta)} = \int\int \frac{1}{\sqrt{2\pi}} \exp\left\{-\frac{1}{2}(x-\eta y)^2\right\}dx \cdot \frac{1}{\sqrt{2\pi}} e^{-\frac{1}{2}(y^2-\eta^2y^2)}dy \\
						 &\Longleftrightarrow e^{A(\eta)} = \frac{1}{\sqrt{2\pi}}\int e^{-\frac{1}{2} y^2(1-\eta^2)}dy \underset{\underset{\text{the integral is finite } \Leftrightarrow |\eta|<1}{\uparrow}}{=} (1-\eta^2)^{-1/2}.
	\end{align*}
	$$ \therefore A(\eta) = -\frac{1}{2} \log(1-\eta^2) \text{ for } \eta \in \Xi = (-1, 1).$$
	$ \therefore $ The canonical exponential density is
	$$ \exp\big\{\eta xy + \frac{1}{2} \log(1-\eta^2) - (x^2+y^2)/2 \big\} / (2\pi). $$
\end{proof}


%% # 2.4
\section{Problem 2.4}
Find the natural parameter space $\Xi$ and densities $p_\eta$ for a canonical one-parameter exponential family with $\mu$ Lebesgue measure on $\bbR$, $T_1(x) = \log x$, and $h(x) = (1-x)^2, ~ x\in (0,1)$, and $h(x) = 0, ~ x\not\in (0, 1)$.

\begin{proof}[\underline{\textbf{Solution}}]
	\begin{align*}
		 e^{A(\eta)} &= \int_0^1 e^{\eta\log x} (1-x)^2 dx \\
		 		  & \text{Let } t = \log x \Rightarrow dt = 1/x dx = e^{-t}dx, ~ -\infty < t < 0 \\
		 		  &= \int_{-\infty}^0 e^{\eta t}(1-e^t)^2e^t dt \\
				  &= \int_{-\infty}^0 \left( e^{t(\eta+1)} - 2e^{t(\eta+2)} + e^{t(\eta+3)} \right) dt \\
				  &= \left( \frac{1}{\eta+1} e^{t(\eta+1)}   -\frac{2}{\eta+2}e^{t(\eta+2)} +\frac{1}{\eta+3} e^{t(\eta+3)} \right) \Big|_{-\infty}^0 \\
				  &= \frac{1}{\eta+1}  -\frac{2}{\eta+2} +\frac{1}{\eta+3}\\
				  &= \frac{2}{(\eta+1)(\eta+2)(\eta+3)}, ~~~ \eta \in \Xi = (-1, \infty).
	\end{align*}
	($\because$ For a finite integral, $\eta+1 >0, ~ \eta+2 > 0,$ and $\eta + 3>0$ $\Rightarrow \eta > -1$.)
	$$ \therefore p_\eta(x) =  \exp\big(\eta T(x) - A(\eta)\big) h(x) =  \frac{1}{2}(\eta+1)(\eta+2)(\eta+3)(1-x)^2x^{\eta}, ~~~ x \in (0,1). $$
\end{proof}


%% # 2.5
\section{Problem 2.5}
Find the natural parameter space $\Xi$ and densities $p_\eta$ for a canonical one-parameter exponential family with $\mu$ Lebesgue measure on $\bbR$, $T_1(x) = -x$, and $h(x) = e^{-2\sqrt{x}}/\sqrt{x}, ~ x > 0$, and $h(x) = 0, ~ x \le 0$. (\underline{Hint}: After a change of variables, relevant integrals will look like integrals against a normal density. You should be able to express the answer using $\Phi$, the standard normal cumulative distribution function.)
Also, determine the mean and variance for a variable $X$ with this density.

\begin{proof}[\underline{\textbf{Solution}}] $\newline$
	\begin{align*}
		 e^{A(\eta)} &= \int_0^\infty e^{-\eta x - 2\sqrt{x}}\frac{1}{\sqrt{x}} dx \\
		 		  & \text{To the integral is finite, } \eta \in \Xi = [0, \infty), \\
		 		  & \text{Let } t = \sqrt x \Rightarrow t^2=x, ~ 2tdt = dx \\
				  &= \int_0^\infty \frac{1}{t} e^{-\eta t^2-2t}2t dt \\
				  &= 2\int_0^\infty e^{-\eta\left(t^2+\frac{2}{\eta} t\right)}dt \\
				  &= 2\int_0^\infty e^{-\frac{2\eta}{2}\left(t+\frac{1}{\eta}\right)^2} dt \cdot e^{-\eta (-\frac{1}{\eta^2})} \\
				  & \text{Since inner term of integration } \sim N( -1/\eta, 1/2\eta ), \\
				  &= 2\sqrt{2\pi}/\sqrt{2\eta} \cdot e^{\frac{1}{\eta}}\Phi\left(\frac{-1/\eta}{\sqrt{1/(2\eta)}} \right) \\
				  &= \frac{2\sqrt{\pi}}{\sqrt\eta}e^{1/\eta}\Phi( -\sqrt{2/\eta} ) \\
		\Rightarrow A(\eta) &= \log 2\sqrt\pi -\frac{1}{2}\log\eta + \frac{1}{\eta} + \log\Phi\left(-\sqrt{\frac{2}{\eta}}\right).
	\end{align*}
	$$ \therefore p_\eta(x) = \exp\big\{ -\eta x -2\sqrt{x} - A(\eta) \big\} \frac{1}{\sqrt x}, ~~~ x > 0. $$
	The mean of $X$ is
	$$ E_\eta(X) = -E_\eta T = -A'(\eta) = \frac{1}{2\eta} + \frac{1}{\eta^2} - \frac{\frac{\sqrt{2}}{2}\eta^{-3/2} \phi\left(-\sqrt{\frac{2}{\eta}}\right)}{\Phi\left(-\sqrt{\frac{2}{\eta}}\right)}. $$
\end{proof}


%% # 2.6
\section{Problem 2.6}
Find the natural parameter space $\Xi$ and densities $p_\eta$ for a canonical two-parameter exponential family with $\mu$ counting measure on $\{0,1,2\}$, $T_1(x) = x, ~ T_2(x) = x^2$, and $h(x) = 1$ for $x \in \{0,1,2\}$.

\begin{proof}[\underline{\textbf{Solution}}]
	\begin{align*}
		e^{A(\eta)} &= \sum_{x=0}^2 e^{\eta_1x+\eta_2x^2} \\
				 &= 1 + e^{\eta_1 +\eta_2} + e^{2\eta_1 + 4\eta_2}.
	\end{align*}
	To the sum is finite, $\eta_1 + \eta_2 < \infty ~\& ~  2\eta_1+4\eta_2 < \infty \Rightarrow \eta \in \Xi = \bbR^2.$
	$$ \therefore p_\eta(x) = \exp\left[ \eta_1x + \eta_2x^2 -\left( 1+e^{\eta_1+\eta_2} + e^{2\eta_1+4\eta_2} \right) \right]. $$
\end{proof}


%% # 2.7
\section{Problem 2.7}
Suppose $X_1, \dots, X_n$ are independent geometric variables with $p_i$ the success probability for $X_i$.
Suppose these success probabilities are related to a sequence of "independent" variables $t_1,\dots, t_n$, viewed as known constants, through
$$ p_i = 1- \exp(\alpha + \beta t_i), ~~~~ i=1,\dots, n. $$
Show that the joint densities for $X_1, \dots, X_n$ form a two-parameter exponential family, and identify the statistics $T_1$ and $T_2$.

\begin{proof}[\underline{\textbf{Solution}}] $\newline$
	Since $X_i \overset{\mathrm{ind}}{\sim} Geo(p_i)$,  then the joint density is represented by
	\begin{align*}
	 	\prod_{i=1}^n p(x_i) &= \prod_{i=1}^n (1-p_i)^{x_i}p_i = \exp[\sum_{i=1}^nx_i\log(1-p_i) + \sum_{i=1}^n\log p_i] \\
					       &= \exp[\sum_{i=1}^nx_i(\alpha + \beta t_i) + \sum_{i=1}^n\log(1-\exp(\alpha+\beta t_i))].
	\end{align*}
	$\therefore T_1 = \sum_{i=1}^n X_i, ~ T_2 = \sum_{i=1}^nt_iX_i.$
\end{proof}


%% # 2.8
\section{Problem 2.8}
Assume that $X_1, \dots, X_n$ are independent random variables with $X_i \sim N(\alpha + \beta t_i, 1)$, where $t_1,\dots, t_n$ are observed constants and $\alpha$ and $\beta$ are unknown parameters.
Show that the joint densities for $X_1, \dots, X_n$ form a two-parameter exponential family, and identify the statistics $T_1$ and $T_2$.

\begin{proof}[\underline{\textbf{Solution}}] $\newline$
	The joint density of $X_1,\dots, X_n$ is
	$$ \prod_{i=1}^n \frac{1}{\sqrt{2\pi}} \exp\left[ -\frac{1}{2}(x_i - \alpha-\beta t_i)^2 \right] = (2\pi)^{-\frac{n}{2}} \exp \left[ \sum_i x_i(\alpha+\beta t_i) - \frac{1}{2} \sum_i (\alpha + \beta t_i) \right] \exp\left( -\sum_i \frac{x_i^2}{2}\right). $$
	$\therefore T_1 = \sum_{i=1}^n X_i, ~ T_2 = \sum_{i=1}^n t_iX_i.$
\end{proof}


%% # 2.9
\section{Problem 2.9}
Suppose that $X_1, \dots, X_n$ are independent Bernoulli variables (a random variable is Bernoulli if it only takes on values 0 and 1) with
$$ P(X_i = 1) = \frac{\exp(\alpha+\beta t_i)}{1 + \exp(\alpha + \beta t_i)}. $$
Show that the joint distribution for $X_1, \dots, X_n$ form a two-parameter exponential family, and identify the statistics $T_1$ and $T_2$.

\begin{proof}[\underline{\textbf{Solution}}] $\newline$
	The joint density of $X_1, \dots, X_n$ is
	\begin{align*}
		\prod_{i=1}^n p_i^{x_i}(1-p_i)^{1-x_i} &= \exp\left[ \sum_i x_i\log p_i + \sum_i (1-x_i)\log (1-p_i) \right] \\
									&= \exp \left[ \sum_i x_i\left(\alpha + \beta t_i - \log\left(1+e^{\alpha+\beta t_i}\right)\right) + \sum_i (1-x_i)\left(1-\log\left(1+e^{\alpha+\beta t_i}\right)\right) \right] \\
									&= \exp\left[ \sum_i x_i(\alpha + \beta t_i) - \sum_i \log\left( 1+e^{\alpha+\beta t_i} \right) \right].
	\end{align*}
	$\therefore T_1 = \sum_{i=1}^n X_i, ~ T_2 = \sum_{i=1}^n t_iX_i.$
\end{proof}


%% # 2.15
\section{Problem 2.15}
For an exponential family in canonical form, $ET_j = \partial A(\eta)/\partial \eta_j$.
This can be written in vector form as $ET = \nabla A(\eta)$.
Derive an analogous differential formula for $E_\theta T$ for an $s$-parameter exponential family that is not in canonical form.
Assume that $\Omega$ has dimension $s$. \\
\underline{Hint}: Differentiation under the integral sign should give a system of linear equations. Write these equations in matrix form.

\begin{proof}[\underline{\textbf{Solution}}] $\newline$
	Note that the exponential family form is
	\begin{align*}
		p_\eta(x) &= \exp\left[ \sum_{i=1}^s \eta_i T_i(x) - A(\eta) \right]h(x) \\
			       &= \exp\left[ \sum_{i=1}^s \eta_i(\theta)T_i(x) - B(\theta) \right]h(x) \\
			       &= p_\theta(x).
	\end{align*}
	Thus, 
	$$ e^{B(\theta)} = \int \exp\left[ \sum_{i=1}^s \eta_i(\theta)T_i(x) \right]h(x) d\mu(x). $$
	By differentiating with respect to $\theta_i$,
	$$ e^{B(\theta)} \frac{\partial B(\theta)}{\partial \theta_i} = \int \left( \sum_{j=1}^s \frac{\partial \eta_j(\theta)}{\partial \theta_i}T_j(x) \right) \exp\left[ \sum_{j=1}^s \eta_j(\theta)T_j(x) \right]h(x) d\mu(x). $$
	\begin{align*}
		\frac{\partial B(\theta)}{\partial \theta_i} &= \int \left( \sum_{j=1}^s \frac{\partial \eta_j(\theta)}{\partial \theta_i}T_j(x) \right) p_\theta (x) d\mu(x) \\
									   &= \sum_{j=1}^s \frac{\partial \eta_j(\theta)}{\partial \theta_i} E_\theta T_j, ~~~~ i = 1, \dots, s.
	\end{align*}
\end{proof}


%% # 2.17
\section{Problem 2.17}
Let $\mu$ denote counting measure on $\{1,2,\dots\}$.
One common definition for $\sum_{k=1}^\infty f(k)$ is $\limn \sum_{k=1}^n f(k)$, and another definition is $\int f d\mu$.
\begin{itemize}
	\item[a)] Use the dominated convergence theorem to show that the two definitions give the same answer when $\int |f| d\mu < \infty$.\\
		\underline{Hint}: Find functions $f_n, ~ n=1,2,\dots$, so that $\sum_{k=1}^n f(k) = \int f_n d\mu$.
            	\begin{proof}[\underline{\textbf{Solution}}] $\newline$
            		Let $f_n(k) = f(k), ~ \forall k \le n$, $f_n(k) = 0, ~ \forall k > n$.
			Then, $f_n \to f$ pointwise and $|f_n| \le |f|$.
			Also $f_n$ is simple and $\int f_n d\mu = \sum_{k=1}^n f(k)$. \\
			By D.C.T., $\int f_n d\mu \to \int f d\mu.$
            	\end{proof}

	\item[b)] Use the monotone convergence theorem, give in Problem 1.25, to show the definitions agree if $f(k) \ge 0$ for all $1,2,\dots$.
            	\begin{proof}[\underline{\textbf{Solution}}] $\newline$
            		Define $f_n$ and $f$ like part (a).
			Then, $0 \le f_1 \le f_2 \le \cdots$. \\
			By M.C.T., $\sum_{k=1}^n f(k) = \int f_nd\mu \to \int f d\mu$.
            	\end{proof}
	
	\item[c)] Suppose $\limn f(n) = 0$ and that $\int f^+d\mu = \int f^-d\mu = \infty$ (so that $\int f d\mu$ is undefined.)
		Let $K$ be an arbitrary constant. Show that the list $f(1), f(2), \dots$ can be rearranged to form a new list $g(1), g(2), \dots$ so that
		$$ \limn \sum_{k=1}^n g(k) = K. $$
            	\begin{proof}[\underline{\textbf{Solution}}] $\newline$
			Take positive parts of the sequence $f$ until the sum is exceed $K$.
			And then, take negative parts of the sequence $f$ until the sum is below $K$.
			Repeat this procedure like the figure below. {\color{blue} (In the figure, $g(k)$ is typo. $\sum g(k)$ is correct.) }
			\begin{figure}[!h]
				\center
				\includegraphics[width = 0.3 \textwidth]{figure/2-17.jpeg}
			\end{figure}
			Then, the sum of the sequence goes to 0 for $k > k'$ where $k'$ is the number when the sum of the rearranged sequence exceed $K$, first.
			Let this sequence as $g$.
			Then, $\limn \sum_{k=1}^n g(k) \to K.$
            	\end{proof}
\end{itemize}


%% # 2.19
\section{Problem 2.19}
Let $p_n, n = 1, 2, \dots,$ and $p$ be probability densities with respect to a measure $\mu$, and let $P_n, n=1,2,\dots$, and $P$ be the corresponding probability measures.
\begin{itemize}
	\item[a)] Show that if $p_n(x) \rightarrow p(x)$ as $n \to \infty$, then $\int |p_n-p| d\mu \to 0$. \\
		\underline{Hint}: First use the fact that $\int(p_n-p)d\mu = 0$ to argue that $\int |p_n-p|d\mu = 2\int(p-p_n)^+d\mu$.
		Then use dominated convergence.
            	\begin{proof}[\underline{\textbf{Solution}}] $\newline$
            		Since $p_n$ and $p$ be probability densities, $\int (p_n-p)d\mu = \int p_n d\mu - \int p d\mu = 1-1= 0$.
			Since $p_n - p = (p-p_n)^+  - (p-p_n)^-$,
			$$ \int (p_n-p) d\mu = \int (p-p_n)^+ d\mu  - \int (p-p_n)^- d\mu = 0, $$
			$$\therefore \int (p-p_n)^+ d\mu  = \int (p-p_n)^- d\mu. $$
			
			Also $|p_n - p| = (p-p_n)^+ + (p-p_n)^-$, 
			$$ \int |p_n - p| d\mu = \int (p-p_n)^+ d\mu + \int (p-p_n)^- d\mu = 2\int (p-p_n)^+ d\mu. $$
			Also note that $|(p-p_n)^+| \le p \Rightarrow |(p-p_n)^+|$ is integrable. ($\because$ $p$ is a probability density $\Rightarrow p$ is integrable.)
			From the condition, $(p_n \to p) \Rightarrow \big(p(x) - p_n(x)\big)^+ \to 0$. \\
			By D.C.T., 
			$$ \int |p_n - p| d\mu = 2\int (p-p_n)^+ d\mu \to 0.$$
            	\end{proof}

	\item[b)] Show that $|P_n(A)-P(A)| \le \int|p_n - p|d\mu$. \\
		\underline{Hint}: Use indicators and the bound $|\int f d\mu| \le \int |f|d\mu$.
            	\begin{proof}[\underline{\textbf{Solution}}]
			\begin{align*}
				LHS &= \left| \int 1_A dP_n - \int 1_A dP \right| \\
				       &= \left|\int 1_A(p_n - p) d\mu\right| \\
				       &\le \int \left| 1_A(p_n - p) \right| d\mu \\
				       &\le \int \left| p_n - p \right| d\mu.
			\end{align*}
            	\end{proof}
\end{itemize}
Remark: Distributions $P_n, ~ n \ge 1$, are said to {\em converge strongly} to $P$ if $\sup_A|P_n(A) - P(A)| \to 0$.
The two parts above show that pointwise convergence of $p_n$ to $p$ implies strong convergence.
This was discovered by Scheff\'{e}.


%% # 2.22
\section{Problem 2.22}
Suppose $X$ is absolutely continuous with density
$$ p_\theta(x) = \begin{cases}
\frac{e^{-(x-\theta)^2/2}}{\sqrt{2\pi}\Phi(\theta)}, ~~~ x > 0, \\
0, ~~~~~~~~~~~~~~ \text{otherwise.}
\end{cases} $$
Find the moment generating function of $X$. Compute the mean and variance of $X$.

\begin{proof}[\underline{\textbf{Solution}}]
	$$ p_\theta(x) = \exp\left[ -\frac{1}{2}(x^2-2x\theta+\theta^2) - \log\Phi(\theta) \right]\frac{1}{\sqrt{2\pi}}, ~~ x > 0 $$
	is the exponential family with
	$T = X, ~ A(\theta) = \frac{\theta^2}{2} + \log\Phi(\theta).$
	Therefore, the moment generating function of $T=X$ is
	\begin{align*}
		M_X(u) &= \exp\left[ A(\theta + u) - A(\theta) \right] \\
			     &= \exp\left[ \frac{1}{2}(\theta+u)^2 - \frac{1}{2}\theta^2 \right] \Phi(\theta+u) / \Phi(\theta) \\
			     &= \exp\left( \theta u + \frac{1}{2}u^2 \right) \Phi(\theta+u) / \Phi(\theta).
	\end{align*}
	Meanwhile, the {\em c.g.f.} of $T=X$ for exponential family is $K_X(u) = A(\theta + u) - A(\theta)$.
	Thus, 
	$$ EX = K_X'(u) = A'(\theta) = \theta + \frac{\phi(\theta)}{\Phi(\theta)}, $$
	$$ \Var(X) = K_X''(u) = A''(\theta) = 1 + \frac{\phi '(\theta)\Phi(\theta) - \phi(\theta)^2}{\Phi(\theta)^2}. $$
	$$ \left( \because \phi(x) = e^{-x^2/2} / \sqrt{2\pi} \Rightarrow \phi'(x) = -x\phi(x). \right) $$
\end{proof}


%% # 2.23
\section{Problem 2.23}
Suppose $Z \sim N(0,1)$. Find the first four cumulants of $Z^2$. \\
\underline{Hint}: Consider the exponential family $N(0,\sigma^2)$.

\begin{proof}[\underline{\textbf{Solution}}] $\newline$
	Let $X \sim N(0, \sigma^2)$. Then, its density is
	$$ p_\sigma(x) = \frac{1}{\sqrt{2\pi}} e^{-\frac{x^2}{s\sigma^2}} = \frac{1}{\sqrt{2\pi}}\exp\left[ -\frac{1}{2\sigma^2}x^2-\log\sigma \right], $$
	and also be the exponential family. \\
	Reparameterize $\eta = -\frac{1}{2\sigma^2}$, then $T = X^2$ and $A(\eta) = \frac{1}{2}\log -\frac{1}{2\eta} = -\frac{1}{2} \log(-2\eta)$.
	Thus, the cumulative generating function of $T=X^2$ is
	\begin{align*}
		K_{X^2}(u) &= A(\eta+u) - A(\eta) \\
				  &= -\frac{1}{2}\left[ \log\big(-2(\eta+u)\big) + \log(-2\eta) \right].
	\end{align*}
	If $\sigma = 1$, then $\eta = -\frac{1}{2}$ and $X^2 = Z^2$.
	Therefore, $$ K_{Z^2}(u) = -\frac{1}{2}\left[ \log(-2u + 1)\right]. $$
	$$ K_{Z^2}'(u) = \frac{1}{1-2u}, ~~~ K_{Z^2}''(u) = \frac{2}{(1-2u)^2}, $$
	$$ K_{Z^2}^{(3)}(u) = 8(1-2u)^{-3}, ~~~ K_{Z^2}^{(4)}(u) = 48(1-2u)^{-4}.$$
	$\therefore$ the cumulants are 1, 2, 8, and 48, respectively.
\end{proof}


%% # 2.24
\section{Problem 2.24}
Find the first four cumulants of $T=XY$ when $X$ and $Y$ are independent standard normal variates.

\begin{proof}[\underline{\textbf{Solution}}] $\newline$
	Since $X$ and $Y$ are independent, the function of these r.v.s are also independent. (i.e. $g(X)$ and $g(Y)$ are independent for any function $g$.)
	Then, first 2 cumulants are
	$$ \kappa_1= ET = EXY = EXEY = 0, $$
	$$ \kappa_2 = E(T-ET)^2 = EX^2Y^2 = EX^2EY^2 = 1. $$
	
	For third cumulants, we need to obtain $EX^3$.
	\begin{align*}
		EX^3 &= \int x^3 \frac{1}{\sqrt{2\pi}} e^{-\frac{x^2}{2}}dx \\
			 & \text{Let } t = x^2/2 \Rightarrow xdx = dt \\
			 &= \frac{2}{\sqrt{2\pi}} \int_0^\infty te^{-t}dt \\
			 &= \frac{2}{\sqrt{2\pi}} \left[ -te^{-t}\big|_0^\infty + \int_0^\infty e^{-t}dt \right] \\
			 &= 0.
	\end{align*}
	Thus, the third cumulants is
	$$ \kappa_3 = E(T - ET)^3 = EX^3Y^3 = EX^3EY^3 = 0. $$
	
	Similarly, we need to obtain $EX^4$.
	{\color{blue} 자꾸 적분이 틀림... 다시 해보기 Gamma dist 형태 사용}
	\begin{align*}
		EX^4 &= \int x^4 \frac{1}{\sqrt{2\pi}} e^{-\frac{x^2}{2}}dx \\
			 & \text{Let } t = x^2/2 \Rightarrow xdx = dt \\
	\end{align*}

	$$ \kappa_4 = E(T-ET)^4 - 3\Var(T)^2 = EX^4EY^4 - 3\cdot 1^2 = 4. $$
\end{proof}


%% # 2.25
\section{Problem 2.25}
Find the third and fourth cumulants of the geometric distribution.

\begin{proof}[\underline{\textbf{Solution}}] 
	$$f_p(x) = (1-p)^xp = \exp\left[ x\log(1-p) + \log p \right]$$
	is the exponential family.
	Reparameterize $\eta = -\log(1-p)$ ($\Rightarrow p = 1-e^\eta$), then the above form be the canonical exponential family with $T=X$ and $A(\eta) = -\log(1-e^\eta)$.
	
	Thus, the cumulative generating function of $T=X$ is
	$$ K_X(u) = A(\eta + u) - A(\eta) = -\log(1-e^{\eta+u}) + \log(1-e^\eta).$$
	Then, the first cumulants is
	$$ K_X'(0) = A'(\eta) = \frac{e^\eta}{1-e^\eta} = \frac{1-p}{p}.$$
	The higher order cumulants are easily obtained by using the chain rule.
	By using the $p' = \frac{dp}{d\eta} = -e^\eta = p-1$,
	$$ K_X''(0) = A''(\eta) = \frac{d}{dp}A'(\eta) \frac{dp}{d\eta} = \frac{-p - (1-p)}{p^2}(p-1) = \frac{1}{p^2} - \frac{1}{p}, $$
	$$ K_X^{(3)}(0) = A^{(3)}(\eta) = \frac{d}{dp}A''(\eta) p' = -\frac{2p'}{p^3} + \frac{p'}{p^2} = -\frac{2-2p}{p^3} + \frac{p-1}{p^2} = \frac{2}{p^3} - \frac{3}{p^2} + \frac{1}{p}, $$
	$$ K_X^{(4)}(0) = A^{(4)}(\eta) = \frac{d}{dp}A^{(3)}(\eta) p' = -\frac{6p'}{p^4} + \frac{6p'}{p^3} - \frac{p'}{p^2} = \frac{6}{p^4}-\frac{12}{p^3}+\frac{7}{p^2}-\frac{1}{p}. $$
\end{proof}


%% # 2.26
\section{Problem 2.26}
Find the third cumulant and third moment of the binomial distribution with $n$ trials and success probability $p$.

\begin{proof}[\underline{\textbf{Solution}}] 
	$$ p(x) = {n \choose x} p^x(1-p)^{n-x} = {n \choose x}\left(\frac{p}{1-p}\right)^x(1-p)^n = \exp\left[  x\log\frac{p}{1-p} + n\log(1-p)\right] {n \choose x} $$
	is the exponential family for $\eta = \log\frac{p}{1-p}$ with $T=X$ and $A(\eta) = -n\log(1-p)$.
	To use the chain rule,
	$$ p = \frac{e^\eta}{1+e^\eta} \Rightarrow \frac{dp}{d\eta} = \frac{e^\eta(1+e^\eta)-(e^\eta)^2}{(1+e^\eta)^2} = p(1-p). $$
	By using the above fact,
	$$ \kappa_1 = A'(\eta) = \frac{d}{dp}A(\eta) \frac{dp}{d\eta} = \frac{n}{1-p}p(1-p) = np, $$
	$$ \kappa_2 = A''(\eta) = \frac{d}{dp}A'(\eta) \frac{dp}{d\eta} = np(1-p), $$
	$$ \kappa_3 = A'''(\eta) = \frac{d}{dp}A''(\eta) \frac{dp}{d\eta} = (n-2np)p(1-p) = np(1-p)(1-2p). $$
	
	The third moment for $T=X$ can be obtained by
	$$ EX^3 = \kappa_3 + 3\kappa_1\kappa_2 + \kappa_1^3. $$
\end{proof}


%% # 2.27
\section{Problem 2.27}
Let $T$ be a random vector in $\bbR^2$.
\begin{itemize}
	\item[a)] Express $\kappa_{2,1}$ as a function of the moments of $T$.
            	\begin{proof}[\underline{\textbf{Solution}}] $\newline$
            		To make simple derivation, we denote $f_{ij}(u) = \frac{\partial^{i+j}f(u)}{\partial u_1^i \partial u_2^j}$.
			Note that $K(u) = \log M(u)$, by taking derivative,
			$$ K_{10} = \frac{M_{10}}{M}, ~~~ K_{20} = \frac{M_{20}M-M_{10}^2}{M^2}, $$
			\begin{align*}
				K_{21} &= \frac{1}{M^4}\left[ (M_{21}M + M_{20}M_{01}-2M_{10}M_{11})M^2 - 2MM_{01}(M_{20}M-M_{10}^2) \right] \\
					   &= \frac{1}{M^3}[ M_{21}M^2 + M_{20}M_{01}M - 2M_{10}M_{11}M - 2M_{01}M_{20}M-2M_{01}M_{10}^2 ] \\
					   &= \frac{1}{M^3}[ M_{21}M^2 - M_{20}M_{01}M - 2M_{10}M_{11}M -2M_{01}M_{10}^2 ].
			\end{align*}
			Taking $u=0$, then
			$$ \kappa_{2,1} = K_{2,1}|_{u=0} = ET_1^2T_2 - (ET_2)(ET_1^2) - 2(ET_1)(ET_1T_2) -2(ET_2)(ET_1)^2.$$
            	\end{proof}

	\item[b)] Assume $ET_1 = ET_2 = 0$ and give an expression for $\kappa_{2,2}$ in terms of moments of $T$.
            	\begin{proof}[\underline{\textbf{Solution}}] $\newline$
            		$\kappa_{2,2}$ can be obtain by one more derivative for $K_{21}$. 
			{\color{blue} (Too complicated. See Keener page 462.) }
            	\end{proof}
\end{itemize}


%% # 2.28
\section{Problem 2.28}
Suppose $X \sim \Gamma(\alpha, 1/\lambda)$, with density
$$ \frac{\lambda^\alpha x^{\alpha-1}e^{-\lambda x}}{\Gamma(\alpha)}, ~~~~ x > 0. $$
Find the cumulants of $T=(X, \log X)$ of order 3 or less.
The answer will involve $\psi(\alpha) = d\log\Gamma(\alpha)/d\alpha = \Gamma'(\alpha)/\Gamma(\alpha)$.

\begin{proof}[\underline{\textbf{Solution}}] $\newline$
	Since Gamma distribution is the exponential family,
	\begin{align*}
		p(x) &= \exp\big[ \alpha \log\lambda + (\alpha -1)\log x - \lambda x - \log \Gamma(\alpha) \big] \\
		       &= \exp\big[ -\lambda x + \alpha \log x - \big(\log\Gamma(\alpha) - \alpha\log \lambda\big) - \log x \big].
	\end{align*}
	Let $\eta = (-\lambda, \alpha)$, and $A(\eta) = \log\Gamma(\eta_2) - \eta_2\log(-\eta_1)$. \\
	Then, the cumulants are obtained by taking derivatives:
	$$ \kappa_{1,0} = \frac{\partial A(\eta)}{\partial \eta_1} = -\frac{\eta_2}{\eta_1} = \frac{\alpha}{\lambda}, $$
	$$ \kappa_{0,1} = \frac{\partial A(\eta)}{\partial \eta_2} = \psi(\eta_2)-\log(-\eta_1) = \psi(\alpha)-\log \lambda, $$
	
	$$ \kappa_{2,0} = \frac{\partial^2 A(\eta)}{\partial \eta_1^2} = \frac{\eta_2}{\eta_1^2} = \frac{\alpha}{\lambda^2}, $$
	$$ \kappa_{1,1} = \frac{\partial^2 A(\eta)}{\partial \eta_1 \partial\eta_2} = -\frac{1}{\eta_1} = \frac{1}{\lambda}, $$
	$$ \kappa_{0,2} = \frac{\partial^2 A(\eta)}{\partial \eta_2^2} = \psi'(\eta_2) = \psi'(\alpha), $$
	
	$$ \kappa_{3,0} = \frac{\partial^3 A(\eta)}{\partial \eta_1^3} = -\frac{2\eta_2}{\eta_1^3} = \frac{2\alpha}{\lambda^3}, $$
	$$ \kappa_{2,1} = \frac{\partial^3 A(\eta)}{\partial \eta_1^2 \partial\eta_2} = \frac{1}{\eta_1^2} = frac{1}{\lambda^2}, $$
	$$ \kappa_{1,2} = \frac{\partial^3 A(\eta)}{\partial \eta_1 \partial\eta_2^2} = 0, $$
	$$ \kappa_{0,3} = \frac{\partial^3 A(\eta)}{\partial \eta_2^3} = \psi''(\eta_2) = \psi''(\alpha). $$
\end{proof}


% Chapter 3. Risk, Sufficiency, Completeness, and Ancillarity
\chapter{Risk, Sufficiency, Completeness, and Ancillarity}

%% # 3.2
\section{Problem 3.2}
Suppose data $X_1, \dots, X_n$ are independent with 
$$ P_\theta(X_i \le x) = x^{t_i\theta}, ~~~~~ x \in (0, 1). $$
where $\theta > 0$ is the unknown parameter, and $t_1, \dots, t_n$ are known positive constants.
Find a one-dimensional sufficient statistic $T$.

\begin{proof}[\underline{\textbf{Solution}}] $\newline$

\end{proof}


%% # 3.3
\section{Problem 3.3}
An object with weight $\theta$ is weighted on scales with different precision.
The data $X_1,\dots,X_n$ are independent, with $X_i \sim N(\theta, \sigma_i^2), ~ i=1,\dots, n$, with the standard deviations $\sigma_1,\dots, \sigma_n$ known constants.
Use sufficiency to suggest a weighted average of $X_1,\dots,X_n$ to estimate $\theta$.
(A weighted average would have form $\sum_{i=1}^n w_iX_i$, where the $w_i$ are positive and sum to one.)

\begin{proof}[\underline{\textbf{Solution}}] $\newline$

\end{proof}


%% # 3.4
\section{Problem 3.4}
Let $X_1,\dots,X_n$ be a random sample from an arbitrary discrete distribution $P$ on $\{1,2,3\}$.
Find a two-dimensional sufficient statistic.

\begin{proof}[\underline{\textbf{Solution}}] $\newline$

\end{proof}


%% # 3.6
\section{Problem 3.6}
The beta distribution with parameters $\alpha > 0$ and $\beta > 0$ has density
$$ f_{\alpha,\beta}(x) = \begin{cases}
\frac{\Gamma(\alpha+\beta)}{\Gamma(\alpha)\Gamma(\beta)} x^{\alpha-1}(1-x)^{\beta-1}, ~~~x\in(0,1), \\
0,~~~~~~~~~~~~~~~~~~~~~~~~~~~~~~~~\text{otherwise.}
\end{cases} $$
Suppose $X_1, \dots, X_n$ are i.i.d. from a beta distribution.
\begin{enumerate}
	\item[a)] Determine a minimal sufficient statistic (for the family of joint distributions) if $\alpha$ and $\beta$ vary freely.
		\begin{proof}[\underline{\textbf{Solution}}] $\newline$

		\end{proof}
		
	\item[b)] Determine a minimal sufficient statistic if $\alpha = 2\beta$.
		\begin{proof}[\underline{\textbf{Solution}}] $\newline$

		\end{proof}
		
	\item[c)] Determine a minimal sufficient statistic if $\alpha = \beta^2$.
		\begin{proof}[\underline{\textbf{Solution}}] $\newline$

		\end{proof}
\end{enumerate}


%% # 3.7
\section{Problem 3.7}
{\em Logistic regression}.
Let $X_1, \dots, X_n$ be independent Bernoulli variables, with $p_i = P(X_i=1), ~ i=1,\dots,n$.
Let $t_1,\dots,t_n$ be a sequence of known constants that are related to the $p_i$ via
$$ \log\frac{p_i}{1-p_i} = \alpha + \beta t_i, $$
where $\alpha$ and $\beta$ are unknown parameters.
Determine a minimal sufficient statistic for the family of joint distributions.

\begin{proof}[\underline{\textbf{Solution}}] $\newline$

\end{proof}


%% # 3.8
\section{Problem 3.8}
The multinomial distribution, derived later in Section 5.3, is a discrete distribution with mass function
$$ \frac{n!}{x_1!\times \cdots \times x_s!}p_1^{x_1}\times \cdots \times p_s^{x_s},$$
where $x_0,\dots, x_s$ are non-negative integers summing to $n$, where $p_1,\dots, p_s$ are non-negative probabilities summing to one, and $n$ is the sample size.
Let $N_{11},N_{21},N_{22}$ have a multinomial distribution with $n$ trials and success probabilities $p_{11},p_{12},p_{21},p_{22}$.
(A common model for a two-by-two contingency table.)
\begin{enumerate}
	\item[a)] Give a minimal sufficient statistic if the success probabilities vary freely over the unit simplex in $\bbR^4$.
		(The unit simplex in $\bbR^p$ is the set of all vectors with non-negative entries summing to one.)
		\begin{proof}[\underline{\textbf{Solution}}] $\newline$

		\end{proof}
		
	\item[b)] Give a minimal sufficient statistic if the success probabilities are constrained so that $p_{11}p_{22} = p_{12}p_{21}$.
		\begin{proof}[\underline{\textbf{Solution}}] $\newline$

		\end{proof}
\end{enumerate}


%% # 3.9
\section{Problem 3.9}
Let $f$ be a positive integrable function on $(0,\infty)$. Define
$$ c(\theta) = 1/\int_\theta^\infty f(x)dx, $$
and take $p_\theta(x) = c(\theta)f(x)$ for $x > \theta$, and $p_\theta(x) = 0$ for $x \le \theta$.
Let $X_1,\dots,X_n$ be i.i.d. with common density $p_\theta$.
\begin{enumerate}
	\item[a)] Show that $M=\min\{X_1,\dots,X_n\}$ is sufficient.
		\begin{proof}[\underline{\textbf{Solution}}] $\newline$

		\end{proof}
		
	\item[b)] Show that $M$ is minimal sufficient.
		\begin{proof}[\underline{\textbf{Solution}}] $\newline$

		\end{proof}
\end{enumerate}


%% # 3.10
\section{Problem 3.10}
Suppose $X_1,\dots,X_n$ are i.i.d. with common density $f_\theta(x) = (1+\theta x)/2, ~ |x| < 1; ~ f_\theta(x) = 0,$ otherwise, where $\theta \in [-1,1]$ is an unknown parameter.
Show that the order statistics are minimal sufficient.
(Hint: A polynomial of degree $n$ is uniquely determined by its value on a grid of $n+1$ points.)

\begin{proof}[\underline{\textbf{Solution}}] $\newline$

\end{proof}


%% # 3.16
\section{Problem 3.16}
Let $X_1,\dots, X_n$ be a random sample from an absolutely continuous distribution with density
$$ f_\theta(x) = \begin{cases}
2x/\theta^2, ~~~x \in (0,\theta), \\
0, ~~~~~~~~~ \text{otherwise}.
\end{cases} $$
\begin{enumerate}
	\item[a)] Find a one-dimensional sufficient statistic $T$.
		\begin{proof}[\underline{\textbf{Solution}}] $\newline$

		\end{proof}
		
	\item[b)] Determine the density of $T$.
		\begin{proof}[\underline{\textbf{Solution}}] $\newline$

		\end{proof}
		
	\item[c)] Show directly that $T$ is complete.
		\begin{proof}[\underline{\textbf{Solution}}] $\newline$

		\end{proof}
\end{enumerate}


%% # 3.17
\section{Problem 3.17}
Let $X, X_1,X_2,\dots$ be i.i.d. from an exponential distribution with failure rate $\lambda$ (introduced in Problem 1.30).
\begin{enumerate}
	\item[a)] Find the density of $Y=\lambda X$.
		\begin{proof}[\underline{\textbf{Solution}}] $\newline$

		\end{proof}
		
	\item[b)] Let $\overline{X} = (X_1 + \cdots + X_n) / n$. Show that $\overline X$ and $(X_1^2 + \cdots + X_n^2)/\overline X^2$ are independent.
		\begin{proof}[\underline{\textbf{Solution}}] $\newline$

		\end{proof}
\end{enumerate}


%% # 3.29
\section{Problem 3.29}
Find a function on $(0,\infty)$ that is bounded and strictly convex.

\begin{proof}[\underline{\textbf{Solution}}] $\newline$

\end{proof}


%% # 3.30
\section{Problem 3.30}
Use convexity to show that the canonical parameter space $\Xi$ of a one-parameter exponential family must be an interval.
Specifically, show that if $\eta_0 <\eta < \eta_1$, and if $\eta_0$ and $\eta_1$ both lie in $\Xi$, then $\eta$ must lie in $\Xi$.

\begin{proof}[\underline{\textbf{Solution}}] $\newline$

\end{proof}


%% # 3.31
\section{Problem 3.31}
Let $f$ and $g$ be positive probability densities on $\bbR$. Use Jensen's inequality to show that
$$ \int \log \left( \frac{f(x)}{g(x)} \right) f(x)dx > 0, $$
unless $f = g$ a.e. (If $f=g$, the integral equals zero.)
This integral is called the {\em Kullback-Liebler information}.

\begin{proof}[\underline{\textbf{Solution}}] $\newline$

\end{proof}














\end{document}
