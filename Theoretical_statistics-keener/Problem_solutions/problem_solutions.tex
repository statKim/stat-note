\documentclass[10pt]{book}

\usepackage{fancyhdr}
\usepackage{extramarks}
\usepackage{amsmath}
\usepackage{amsthm}
\usepackage{amsfonts}
\usepackage{tikz}
%\usepackage[plain]{algorithm}
%\usepackage{algpseudocode}

\usetikzlibrary{automata,positioning}

%%%%%%%%%%%%%%%%%%%%%%%%
%%% Useful packages and commands
%%%%%%%%%%%%%%%%%%%%%%%%
% packages
\usepackage{amsmath,amssymb,amsthm}
%\usepackage{times}
%\usepackage{setspace}
\usepackage{indentfirst}
\usepackage{epsfig}
\usepackage{graphicx}
\usepackage{latexsym}
\usepackage{amscd}
\usepackage{multirow}
\usepackage{array}
\usepackage{caption}
\usepackage{rotating}
\usepackage{subfig}
\usepackage{color}
\usepackage{natbib}
\usepackage{lscape}
\usepackage{graphics}
\usepackage{enumerate}
%\usepackage{fancyvrb}
%\usepackage{mathtools}
\usepackage{verbatim}
\usepackage{afterpage}
%\usepackage[ruled,vlined]{algorithm2e}
\usepackage{hyperref}
\usepackage[flushleft]{threeparttable}
\usepackage{rotating}
\usepackage{kotex}   % for Korean

% new commands
%\DeclarePairedDelimiter\abs{\lvert}{\rvert}
%\DeclarePairedDelimiter\norm{lVert}{\rVert}
\long\def\comment#1{}

%\newtheorem*{thm}{Theorem}
\newtheorem{thm}{Theorem}[section]
\newtheorem{cor}[thm]{Corollary}
\newtheorem{lem}[thm]{Lemma}
\newcommand{\rb}[1]{\raisebox{-.5em}[0pt]{#1}}
% \renewcommand{\baselinestretch}{1.8}
\renewcommand{\mid}{\, | \ }
\newcommand{\eighth}{{\textstyle \frac{1}{8}}}

\def \bY { \mathbf{ Y } }
\def \bX { \mathbf{ X } }
\def \bU { \mathbf{ U } }
\def \bmu { \boldsymbol{ \mu } }
\def \bSigma { \boldsymbol{ \Sigma } }
\def \bphi { \boldsymbol{ \phi } }
\def \bepsilon { \boldsymbol{ \epsilon } }
\def \bD { \boldsymbol{\mathcal{D}} }

\newcommand{\eqdis}{\overset{\mathrm{d}}{=\joinrel=}}
\newcommand{\ba}{\mbox{\boldmath $a$}}
\newcommand{\bg}{\mbox{\boldmath $g$}}
\newcommand{\bx}{\mbox{\boldmath $x$}}
\newcommand{\by}{\mbox{\boldmath $y$}}
\newcommand{\bd}{\mbox{\boldmath $d$}}
\newcommand{\bff}{\mbox{\boldmath $f$}}
\newcommand{\bz}{\mbox{\boldmath $z$}}
\newcommand{\bu}{\mbox{\boldmath $u$}}
\newcommand{\bv}{\mbox{\boldmath $v$}}
\newcommand{\bW}{\mbox{\boldmath $W$}}
\newcommand{\bI}{\mbox{\boldmath $I$}}
\newcommand{\bJ}{\mbox{\boldmath $J$}}
\newcommand{\bL}{\mbox{\boldmath $L$}}
\newcommand{\bQ}{\mbox{\boldmath $Q$}}
\newcommand{\bZ}{\mbox{\boldmath $Z$}}
\newcommand{\bV}{\mbox{\boldmath $V$}}
\newcommand{\bG}{\mbox{\boldmath $G$}}
\newcommand{\bdm}{\begin{displaymath}}
\newcommand{\edm}{\end{displaymath}}
\newcommand{\bnu}{\mbox{\boldmath $\nu$}}
\newcommand{\btau}{\mbox{\boldmath $\tau$}}
\newcommand{\biota}{\mbox{\boldmath $\iota$}}
\newcommand{\bbeta}{\mbox{\boldmath $\beta$}}
\newcommand{\bomega}{\mbox{\boldmath $\omega$}}
\newcommand{\btheta}{\mbox{\boldmath $\theta$}}
\newcommand{\bep}{\mbox{\boldmath $\epsilon$}}
\newcommand{\bdelta}{\mbox{\boldmath $\delta$}}
\newcommand{\balpha}{\mbox{\boldmath $\alpha$}}
\newcommand{\bxi}{\mbox{\boldmath $\xi$}}
\newcommand{\bgamma}{\mbox{\boldmath $\gamma$}}
\newcommand{\bOmega}{\mbox{\boldmath $\Omega$}}
\newcommand{\bPi}{\mbox{\boldmath $\Pi$}}
\newcommand{\bzeta}{\mbox{\boldmath $\zeta$}}
\newcommand{\bpsi}{\mbox{\boldmath $\psi$}}
\newcommand{\bPsi}{\mbox{\boldmath $\Psi$}}
\newcommand{\bl}{\mbox{\boldmath $l$}}
\newcommand{\C}{{\rm Cov}}
\newcommand{\bH}{\bold H}
\newcommand{\blambda}{\mbox{\boldmath $\lambda$}}
\newcommand{\bbh}{\bld h}
\newcommand{\calA}{\mathcal{A}}
\newcommand{\calB}{\mathcal{B}}
\newcommand{\calE}{\mathcal{E}}
\newcommand{\calX}{\mathcal{X}}
\newcommand{\calY}{\mathcal{Y}}
\newcommand{\bbR}{\mathbb{R}}
\newcommand{\union}{\bigcup}
\newcommand{\intersect}{\bigcap}
\newcommand{\eqdef}{\overset{\mathrm{def}}{=}}
\newcommand{\limn}{\lim_{n \to \infty}}


%
% Basic Document Settings
%

\topmargin=-0.45in
\evensidemargin=0in
\oddsidemargin=0in
\textwidth=6.5in
\textheight=9.0in
\headsep=0.25in

\linespread{1.2}

\pagestyle{fancy}
\lhead{\leftmark}
%\chead{\hmwkClass\ (\hmwkClassInstructor\ \hmwkClassTime): \hmwkTitle}
\rhead{\authorName}
%\lfoot{\lastxmark}
\cfoot{\thepage}

\renewcommand\headrulewidth{0.4pt}
\renewcommand\footrulewidth{0.4pt}

\setlength\parindent{0pt}

%%
%% Create Problem Sections
%%
%
%\newcommand{\enterProblemHeader}[1]{
%    \nobreak\extramarks{}{Problem \arabic{#1} continued on next page\ldots}\nobreak{}
%    \nobreak\extramarks{Problem \arabic{#1} (continued)}{Problem \arabic{#1} continued on next page\ldots}\nobreak{}
%}
%
%\newcommand{\exitProblemHeader}[1]{
%    \nobreak\extramarks{Problem \arabic{#1} (continued)}{Problem \arabic{#1} continued on next page\ldots}\nobreak{}
%    \stepcounter{#1}
%    \nobreak\extramarks{Problem \arabic{#1}}{}\nobreak{}
%}
%
%\setcounter{secnumdepth}{0}
%\newcounter{partCounter}
%\newcounter{homeworkProblemCounter}
%\setcounter{homeworkProblemCounter}{1}
%\nobreak\extramarks{Problem \arabic{homeworkProblemCounter}}{}\nobreak{}
%
%%
%% Homework Problem Environment
%%
%% This environment takes an optional argument. When given, it will adjust the
%% problem counter. This is useful for when the problems given for your
%% assignment aren't sequential. See the last 3 problems of this template for an
%% example.
%%
%\newenvironment{homeworkProblem}[1][-1]{
%    \ifnum#1>0
%        \setcounter{homeworkProblemCounter}{#1}
%    \fi
%    \section{Problem \arabic{homeworkProblemCounter}}
%    \setcounter{partCounter}{1}
%    \enterProblemHeader{homeworkProblemCounter}
%}{
%    \exitProblemHeader{homeworkProblemCounter}
%}

%
% Homework Details
%   - Title
%   - Due date
%   - Class
%   - Section/Time
%   - Instructor
%   - Author
%

%\newcommand{\hmwkTitle}{}
%\newcommand{\hmwkClass}{Theoretical Statistics: Topics for a Core Course\\ Solution}
%\newcommand{\hmwkClassTime}{}
%\newcommand{\hmwkClassInstructor}{Robert W. Keener}

\newcommand{\courseTitle}{Theoretical Statistics: Topics for a Core Course\\ Problem Solutions}
\newcommand{\bookAuthor}{Robert W. Keener}
\newcommand{\authorName}{Hyunsung Kim}



%
% Title Page
%

\title{
    \vspace{1.5in}
%    \textmd{\textbf{\hmwkClass:\ \hmwkTitle}}\\
    \textmd{\textbf{\courseTitle}}\\
%    \normalsize\vspace{0.1in}\hmwkDueDate\\
    \vspace{0.1in}\large{\textit{\bookAuthor}}
    \vspace{2in}
}
\date{Update: \today}
\author{
    {\sc \authorName} \\
    Department of Statistics\\
    Chung-Ang University
}



\renewcommand{\part}[1]{\textbf{\large Part \Alph{partCounter}}\stepcounter{partCounter}\\}

%
% Various Helper Commands
%

% Useful for algorithms
\newcommand{\alg}[1]{\textsc{\bfseries \footnotesize #1}}

% For derivatives
\newcommand{\deriv}[1]{\frac{\mathrm{d}}{\mathrm{d}x} (#1)}

% For partial derivatives
\newcommand{\pderiv}[2]{\frac{\partial}{\partial #1} (#2)}

% Integral dx
\newcommand{\dx}{\mathrm{d}x}
\newcommand{\dmu}{\mathrm{d}\mu}
\newcommand{\dP}{\mathrm{d}P}

% Alias for the Solution section header
\newcommand{\solution}{\textbf{\large Solution}}

% Probability commands: Expectation, Variance, Covariance, Bias
\newcommand{\E}{\mathrm{E}}
\newcommand{\Var}{\mathrm{Var}}
\newcommand{\Cov}{\mathrm{Cov}}
\newcommand{\Bias}{\mathrm{Bias}}

% Specify level of toc(table of content)
\setcounter{tocdepth}{0}

% Specify level of Section number
\setcounter{secnumdepth}{0}

% Setting for equation breaks
\allowdisplaybreaks

% Setting for blank page appeared when begin new chapter
\let\cleardoublepage=\clearpage


\newtheorem{theorem}{Theorem}[section]
\newtheorem{proposition}[theorem]{Proposition}
\newtheorem{corollary}[theorem]{Corollary}
\newtheorem{lemma}[theorem]{Lemma}
\newtheorem{definition}[theorem]{Definition}

\theoremstyle{definition}
\newtheorem{remark}[theorem]{Remark}
\newtheorem{example}[theorem]{Example}
%\newtheorem{problem}[theorem]{Problem}


%%% Main part
\begin{document}

% Title page
\maketitle

% Table of Contents
\tableofcontents

%% Preface
%\addcontentsline{toc}{chapter}{Preface}
\chapter*{Preface}
%\section{Preface}
This note contains summaries of the textbook, {\em Theoretical Statistics: Topics for a Core Course}, and it was created by Hyunsung Kim, who is a Ph.D. student.
I made it when I study a theoretical statistics based on this textbook on my own.

\section*{Textbook}
\begin{itemize}
	\item Keener, {\em Theoretical Statistics: Topics for a Core Course}.
\end{itemize}

\section*{Reference}
\begin{itemize}
	\item Durrett, {\em Probability: Theory and Examples, 5th edition}.
	\item Royden, {\em Real Analysis, 4th edition}.
\end{itemize}

% Test file
%\chapter{Math symbol test}

\section{Theorem symbol}

\begin{theorem}[Pythagorean theorem]
\label{pythagorean}
This is a theorem about right triangles and can be summarised in the next 
equation 
\[ x^2 + y^2 = z^2 \]
\end{theorem}
\begin{proof}
	dkdkdk
\end{proof}

And a consequence of theorem \ref{pythagorean} is the statement in the next 
corollary.

\begin{corollary}
There's no right rectangle whose sides measure 3cm, 4cm, and 6cm.
\end{corollary}

You can reference theorems such as \ref{pythagorean} when a label is assigned.

\begin{proposition}[Consistnecy]
ddfafa
\end{proposition}

\begin{lemma}
Given two line segments whose lengths are \(a\) and \(b\) respectively there is a 
real number \(r\) such that \(b=ra\).
\end{lemma}

\begin{remark}
This statement is true, I guess.
\end{remark}

\begin{definition}[Fibration]
A fibration is a mapping between two topological spaces that has the homotopy lifting property for every space \(X\).
\end{definition}

\begin{example}[Continuous prob]
This statement is true, I guess.
\end{example}

% Chapter 1. Probability and Measure
\chapter{Probability and Measure}

%% # 1.1
\section{Problem 1.1}
Prove (1.1). If measurable sets $B_n, ~ n \ge 1$, are increasing, with $B = \union_{n=1}^\infty B_n$, called the limit of the sequence, then
$$ \mu(B) = \lim_{n \rightarrow \infty} \mu(B_n). $$
\begin{proof}[\underline{\textbf{Solution}}] $\newline$
    First, we showed that $A_n$'s are disjoint.
    If $j < k$, then $B_j \subseteq B_{k-1} $. \\
    Since $A_j \subset B_j \subseteq B_{k-1}$ and $A_k \subset B_{k-1}^c$, $A_j$ and $A_k$ are disjoint.
    
    Also $B_n = \union_{j=1}^n A_j$ and $\union_{n=1}^\infty A_n = B$,
    $$ \mu(B) = \sum_{i=1}^\infty \mu(A_i) = \lim_{n \rightarrow \infty} \sum_{i=1}^n \mu(A_i) =  \lim_{n \rightarrow \infty} \mu\left(\union_{i=1}^n A_i\right) = \lim_{n \rightarrow \infty} \mu(B_n). $$
\end{proof}


%% # 1.8
\section{Problem 1.8}
Prove {\em Boole's inequality}: For any events $B_1, B_2, \dots $,
$$ P\left(\union_{i \ge 1} B_i\right) \le \sum_{i \ge 1}P(B_i). $$
\begin{proof}[\underline{\textbf{Solution}}] $\newline$
    Let $B = \union_{i=1}^\infty B_i$, then $1_B \le \sum 1_{B_i}$. Then by Fubini's theorem,
    $$ P(B) = \int 1_B dP \le \int \sum 1_{B_i} dP = \sum \int 1_{B_i} dP = \sum P(B_i).$$
\end{proof}


%% # 1.10
\section{Problem 1.10}
Let $\mu$ and $\nu$ be measures on $(\calE, \calB)$.
\begin{itemize}
	\item[a)] Show that the sum $\eta$ defined by $\eta(B) = \mu(B) + \nu(B)$ is also a measure.
	\begin{proof}[\underline{\textbf{Solution}}] $\newline$
		We should check following 2 conditions.
		\begin{itemize}
			\item[(i)] For arbitrary set $A \in \calB$, $\mu(A) \ge 0$ and $\nu(A) \ge 0$. $\Rightarrow \eta(A) = \mu(A) + \nu(A) \ge 0.$ \\
            		$\therefore \eta: \calB \rightarrow [0, \infty].$
			\item[(ii)] For disjoint set $B_1, B_2, \dots \subset \calB$, then
            	\begin{align*}
            		\eta \left( \union_{i=1}^\infty B_i \right) &= \mu\left(\union B_i\right) + \nu\left(\union B_i\right) \\
            									  &= \sum \mu(B_i) + \sum \nu(B_i) \\
            									  &= \sum \left\{ \mu(B_i) + \nu(B_i) \right\} \\
            									  &= \sum \eta(B_i)
            	\end{align*}
		\end{itemize}
            	$\therefore \eta$ is a measure.
	\end{proof}
	
	\item[b)] If $f$ is a non-negative measurable function, show that
	$$\int f d\eta = \int f d\mu + \int f d\nu.$$
	\begin{proof}[\underline{\textbf{Solution}}] $\newline$
		We show it by 2 stage.
		\begin{itemize}
			\item[(i)] Let $f = \sum_{i=1}^n a_i 1_{A_i}$, the non-negative simple function. Then,
                    		\begin{align*}
                    			\int f d\eta &= \int \sum_{i=1}^n a_i 1_{A_i} d\eta = \sum a_i\eta(A_i) \\
                    					 &= \sum a_i \left\{ \mu(A_i) + \nu(A_i) \right\}  = \int f d\mu + \int f d\nu.
                    		\end{align*}
			\item[(ii)] For general case, let $f_n$ is the sequence of non-negative simple functions increasing to $f$. (i.e. $f_1 \le f_2 \le \dots \le f$)
                    		\begin{align*}
                    			\int f d\eta &= \lim_{n \to \infty} \int f_n d\eta = \lim_{n \to \infty} \left( \int f_n d\mu + \int f_n d\nu \right)  \\
                    					&= \lim_{n \to \infty} \int f_n d\mu + \lim_{n \to \infty} \int f_n d\nu = \int f d\mu + \int f d\nu.
                    		\end{align*}
		\end{itemize}
	\end{proof}
\end{itemize}


%% 1.11
\section{Problem 1.11}
Suppose $f$ is the simple function $1_{(1/2, \pi]} + 21_{(1, 2]}$, and let $\mu$ be a measure on $\bbR$ with $\mu\{(0,a^2]\} = a, ~ a > 0$. Evaluate $\int f d\mu.$
\begin{proof}[\underline{\textbf{Solution}}] $\newline$
By the integral of simple function and the finite additivity, it can be simply computed as
\begin{align*}
	\int f d\mu &= \mu\{(1/2, \pi] \} + 2\mu\{(1,2]\} \\
			&= \left[ \mu\{(0, \pi]\} - \mu\{(1/2, \pi] \} \right] + 2\left[ \mu\{ (0, 2] \} -  \mu\{(1,2]\} \right]  ~~ \text{(finite additivity)}\\
			&= \left( \sqrt{\pi} - 1/\sqrt{2} \right) + 2\left( \sqrt{2} - 1 \right)
\end{align*}
\end{proof}


%% 1.12
\section{Problem 1.12}
Suppose that $\mu\{ (0, a) \} = a^2$ for $a > 0$ and that $f$ is defined by
$$ f(x) = \begin{cases}
    0, ~~~ x \le 0, \\
    1, ~~~ 0 < x < 2, \\
    \pi, ~~~ 2 \ge x < 5, \\
    0, ~~~ x \ge 5.
\end{cases}$$
Compute $\int f d\mu.$
\begin{proof}[\underline{\textbf{Solution}}] $\newline$
Since $f = 1_{(0, 2)} + 1_{[2, 5)}$ is simple, the integral can be computed as
\begin{align*}
	\int f d\mu &= \mu\{ (0, 2) \} + \pi \mu\{ [2, 5) \} \\
			&= \mu\{ (0, 2) \} + \pi \left[ \mu\{ (0, 5) \} - \mu\{ (0, 2) \} \right] \\
			&= 4 - \pi(25-4)
\end{align*}
\end{proof}


%% 1.13
\section{Problem 1.13}
Define the function $f$ by
$$ f(x) = \begin{cases}
    x, ~~~ 0 \le x \le 1, \\
    0, ~~~ \text{otherwise.}
\end{cases}$$
Find simple functions $f_1 \le f_2 \le \cdots$ increasing to $f$ (i.e. $f(x) = \limn f_n(x)$ for all $x \in \bbR$).
Let $\mu$ be Lebesgue measure on $\bbR$.
Using our formal definition of an integral and the faProblemct that $\mu\big( (a, b] \big) = b-a$ whenever $b > a$ (this might be used to formally define Lebesgue measure), show that $\int f d\mu = 1/2$.
\begin{proof}[\underline{\textbf{Solution}}] $\newline$
Let $f_n = \lfloor 2^nx \rfloor / 2^n$ for $0 < x \le 1$ and 0 otherwise. ($\lfloor y \rfloor$ is a floor function.)
Then,
\begin{align*}
    f_1(x) &= \lfloor 2x \rfloor / 2 = \begin{cases}
    0, ~~~ x < 1/2, \\
    1/2, ~~~ 1/2 \le x < 1, \\
    1, ~~~ x = 1
    \end{cases} \\
    f_2(x) &= \lfloor 2^2x \rfloor / 2^2 = \begin{cases}
    0, ~~~ x < 1/2^2, \\
    1/2^2, ~~~ 1/2^2 \le x < 2/2^2, \\
    2/2^2, ~~~ 2/2^2 \le x < 3/2^2, \\
    3/2^2, ~~~ 3/2^2 \le x < 1, \\
    1, ~~~ x = 1
    \end{cases}\\
    &\vdots \\
    f_n(x) &= \lfloor 2^nx \rfloor / 2^n = \begin{cases}
    0, ~~~ x < 1/2^n, \\
    1/2^n, ~~~ 1/2^n \le x < 2/2^n, \\
    \vdots \\
    (2^n-1)/2^n, ~~~ (2^n-1)/2^n \le x < 1, \\
    1, ~~~ x = 1
    \end{cases}\\
\end{align*}
\begin{align*}
\therefore \int f_n d\mu &= \frac{1}{2^n} \left( \frac{1}{2^n} + \frac{2}{2^n} + \cdots + \frac{2^n-1}{2^n} \right) \\
		    &= \frac{1+2+\cdots + (2^n-1)}{4^n} \\
		    &= \frac{2^n(2^n-1)}{2 \cdot 4^n} \\
		    &\longrightarrow \frac{1}{2}.
\end{align*}

{\color{blue} \underline{My solution} \\
Let the simple function $f_n = \sum_{i=1}^n \frac{i}{n}1_{(\frac{i-1}{n}, \frac{i}{n} ]}$. Then,
$$ \int f_n d\mu = \sum_{i=1}^n \frac{i}{n} \frac{1}{n} = \frac{1}{n^2} \frac{n(n+1)}{2} \longrightarrow \frac{1}{2}.$$
}
\end{proof}


%% 1.16
\section{Problem 1.16}
Define $F(a-) = \lim_{x\uparrow a}F(x)$. Then, if $F$ is non-decreasing, $F(a-) = \limn F(a-1/n)$.
Use (1.1)[Continuity of measure] to show that if a random variable $X$ has cumulative distribution function $F_X$,
$$ P(X < a) = F_X(a-). $$
Also, show that
$$ P(X=a) = F_X(a) - F_X(a-). $$
\begin{proof}[\underline{\textbf{Solution}}] $\newline$
\begin{itemize}
    \item[(i)] Let $B_n = \{ X \le a-1/n \}$ and $\union_{n=1}^\infty B_n = \{ X < a \}$.
    By continuity of measure,
    $$ P(B) = P(X < a) = \limn P(X \le a-1/n) = \limn F_X(a-1/n) = F_X(a-). $$
    \item[(ii)] Since $\{ X<a \}$ and $\{X=a\}$ are disjoint with union $\{X \le a\}$,
    $$ P(X < a) + P(X = a) = P(X \le a).$$
    $$\therefore P(X = a) = F_X(a) - F_X(a-).$$
\end{itemize}
\end{proof}


%% 1.17
\section{Problem 1.17}
Suppose $X$ is a geometric random variable with mass function
$$ p(x) = P(X=x) = \theta(1-\theta)^x, ~~~ x= 0,1,\dots, $$
where $\theta \in (0,1)$ is a constant. Find the probability that $X$ is even.
\begin{proof}[\underline{\textbf{Solution}}]
\begin{align*}
P(X\text{ is even}) &= P(X=0) + P(X=2) + P(X=4) \cdots \\
			     &= \theta + \theta(1-\theta)^2 + \theta(1-\theta)^4 + \cdots \\
			     &= \frac{\theta}{1-(1-\theta)^2} \\
			     &= \frac{1}{2-\theta}
\end{align*}
\end{proof}


%% 1.18
\section{Problem 1.18}
Let $X$ be a function mapping $\calE$ into $\bbR$.
Recall that if $B$ is a subset of $\bbR$, then $X^{-1}(B) = \{e \in \calE : X(e) \in B\}.$
Use this definition to prove that
$$ X^{-1}(A \cap B) = X^{-1}(A) \cap X^{-1}(B), $$
$$ X^{-1}(A \cup B) = X^{-1}(A) \cup X^{-1}(B), $$
and
$$ X^{-1}\left(\union_{i=0}^\infty A_i \right) = \union_{i=0}^\infty X^{-1}(A_i). $$
\begin{proof}[\underline{\textbf{Solution}}] $\newline$
\begin{itemize}
\item[(i)] \begin{align*}
    e \in X^{-1}(A \cap B) &\Leftrightarrow X(e) \in A \cap B \\
    				&\Leftrightarrow X(e) \in A \text{ and } X(e) \in B \\
    				&\Leftrightarrow e \in X^{-1}(A) \text{ and } e \in X^{-1}(B) \\
    				&\Leftrightarrow e \in X^{-1}(A) \cap X^{-1}(B).
    \end{align*}

\item[(ii)] Similarly,
    \begin{align*}
    e \in X^{-1}(A \cup B) &\Leftrightarrow X(e) \in A \cup B \\
    				&\Leftrightarrow X(e) \in A \text{ or } X(e) \in B \\
    				&\Leftrightarrow e \in X^{-1}(A) \text{ or } e \in X^{-1}(B) \\
    				&\Leftrightarrow e \in X^{-1}(A) \cup X^{-1}(B).
    \end{align*}

\item[(iii)] \begin{align*}
    e \in X^{-1}(\union_{i=0}^\infty A_i) &\Leftrightarrow X(e) \in \union_{i=0}^\infty A_i \\
    				&\Leftrightarrow X(e) \in A_i \text{ for some } i \\
    				&\Leftrightarrow e \in X^{-1}(A_i) \text{ for some } i \\
    				&\Leftrightarrow e \in  \union_{i=0}^\infty X^{-1}(A_i).
    \end{align*}

\end{itemize}
\end{proof}





% Chapter 2. Exponential Families
% Chapter 2. Exponential Families
\chapter{Exponential Families}

%% # 2.1
\section{Problem 2.1}
Consider independent Bernoulli trials with success probability $p$ and let $X$ be the number of failures before the first success.
Then $P(X=x) = p(1-p)^x$, for $x = 0,1,\dots $, and $X$ has the geometric distribution with parameter $p$, introduced in Problem 1.17.
\begin{itemize}
	\item[a)] Show that the geometric distributions form an exponential family.
            	\begin{proof}[\underline{\textbf{Solution}}]
                		$$ P(X=x) = p(1-p)^x = \exp\big\{ x\log(1-p) - (-\log p) \big\}. $$
			$\therefore$ the geometric distribution is an exponential family.
            	\end{proof}
	
	\item[b)] Write the densities for the family in canonical form, identifying the canonical parameter $\eta$, and the function $A(\eta)$.
            	\begin{proof}[\underline{\textbf{Solution}}] $\newline$
                		Let $\eta(p) = \log(1-p).$ Then, $ P(X=x) = \exp\big\{ \eta x - \big(-\log(1-e^\eta)\big) \big\}. $ \\
			$\therefore A(\eta) = -\log(1-e^\eta)$ with $T(x) = x$.
            	\end{proof}
	
	\item[c)] Find the mean of the geometric distribution using a differential identity.
            	\begin{proof}[\underline{\textbf{Solution}}] 
                		$$ E_\eta(T) = A'(\eta) = \frac{e^\eta}{1-e^\eta} = \frac{1-p}{p}. $$
            	\end{proof}	
	
	\item[d)] Suppose $X_1, \dots, X_n$ are i.i.d. from a geometric distribution. Show that the joint distributions form an exponential family, and find the mean and variance of $T$.
            	\begin{proof}[\underline{\textbf{Solution}}] $\newline$
                		$$ P(X_1 = x_1,\dots, X_n = x_n) = \prod_{i=1}^n P(X_i = x_i) = p^n(1-p)^{\sum_{i=1}^n x_i} = \exp\big\{ \sum_ix_i\log(1-p) - (-n\log p) \big\}. $$
		Therefore, the joint distribution is also exponential family.\\
		Now, let $\eta = \log(1-p)$. Then, the canonical exponential family is obtained with $A(\eta) = -n\log(1-e^\eta)$ and $T=\sum_i x_i$. 
		$$ ET = \kappa_1 = A'(\eta) = \frac{ne^\eta}{1-e^\eta} = \frac{n(1-p)}{p}, $$
		$$ Var(T) = \kappa_2 = A''(\eta) = \frac{ne^\eta(1-e^\eta)+ne^{2\eta}}{(1-e^\eta)^2} = \frac{np(1-p)+n(1-p)^2}{p^2} = \frac{n(1-p)}{p^2}. $$
            	\end{proof}
\end{itemize}


%% # 2.2
\section{Problem 2.2}
Determine the canonical parameter space $\Xi$, and find densities for the one-parameter exponential family with $\mu$ Lebesgue measure on $\bbR^2$, $h(x,y) = \exp\big[ -(x^2+y^2)/2 \big] / (2\pi)$, and $T(x,y) = xy$.

\begin{proof}[\underline{\textbf{Solution}}] $\newline$
	By definition of the canonical exponential family, the pdf be represented as
	$$ p(x,y) = \exp\big(\eta T(x,y) - A(\eta)\big) h(x,y) = \exp\big\{\eta xy - A(\eta) - (x^2+y^2)/2 \big\} / (2\pi).$$
	Thus,
	\begin{align*}
		\int p(x,y)d\mu(x,y) = 1 &\Longleftrightarrow \int \int \exp\big\{\eta xy - A(\eta) - (x^2+y^2)/2 \big\} / (2\pi) dxdy = 1 \\
						 &\Longleftrightarrow e^{A(\eta)} = \int\int \frac{1}{\sqrt{2\pi}} \exp\left\{-\frac{1}{2}(x-\eta y)^2\right\}dx \cdot \frac{1}{\sqrt{2\pi}} e^{-\frac{1}{2}(y^2-\eta^2y^2)}dy \\
						 &\Longleftrightarrow e^{A(\eta)} = \frac{1}{\sqrt{2\pi}}\int e^{-\frac{1}{2} y^2(1-\eta^2)}dy \underset{\underset{\text{the integral is finite } \Leftrightarrow |\eta|<1}{\uparrow}}{=} (1-\eta^2)^{-1/2}.
	\end{align*}
	$$ \therefore A(\eta) = -\frac{1}{2} \log(1-\eta^2) \text{ for } \eta \in \Xi = (-1, 1).$$
	$ \therefore $ The canonical exponential density is
	$$ \exp\big\{\eta xy + \frac{1}{2} \log(1-\eta^2) - (x^2+y^2)/2 \big\} / (2\pi). $$
\end{proof}


%% # 2.4
\section{Problem 2.4}
Find the natural parameter space $\Xi$ and densities $p_\eta$ for a canonical one-parameter exponential family with $\mu$ Lebesgue measure on $\bbR$, $T_1(x) = \log x$, and $h(x) = (1-x)^2, ~ x\in (0,1)$, and $h(x) = 0, ~ x\not\in (0, 1)$.

\begin{proof}[\underline{\textbf{Solution}}]
	\begin{align*}
		 e^{A(\eta)} &= \int_0^1 e^{\eta\log x} (1-x)^2 dx \\
		 		  & \text{Let } t = \log x \Rightarrow dt = 1/x dx = e^{-t}dx, ~ -\infty < t < 0 \\
		 		  &= \int_{-\infty}^0 e^{\eta t}(1-e^t)^2e^t dt \\
				  &= \int_{-\infty}^0 \left( e^{t(\eta+1)} - 2e^{t(\eta+2)} + e^{t(\eta+3)} \right) dt \\
				  &= \left( \frac{1}{\eta+1} e^{t(\eta+1)}   -\frac{2}{\eta+2}e^{t(\eta+2)} +\frac{1}{\eta+3} e^{t(\eta+3)} \right) \Big|_{-\infty}^0 \\
				  &= \frac{1}{\eta+1}  -\frac{2}{\eta+2} +\frac{1}{\eta+3}\\
				  &= \frac{2}{(\eta+1)(\eta+2)(\eta+3)}, ~~~ \eta \in \Xi = (-1, \infty).
	\end{align*}
	($\because$ For a finite integral, $\eta+1 >0, ~ \eta+2 > 0,$ and $\eta + 3>0$ $\Rightarrow \eta > -1$.)
	$$ \therefore p_\eta(x) =  \exp\big(\eta T(x) - A(\eta)\big) h(x) =  \frac{1}{2}(\eta+1)(\eta+2)(\eta+3)(1-x)^2x^{\eta}, ~~~ x \in (0,1). $$
\end{proof}


%% # 2.5
\section{Problem 2.5}
Find the natural parameter space $\Xi$ and densities $p_\eta$ for a canonical one-parameter exponential family with $\mu$ Lebesgue measure on $\bbR$, $T_1(x) = -x$, and $h(x) = e^{-2\sqrt{x}}/\sqrt{x}, ~ x > 0$, and $h(x) = 0, ~ x \le 0$. (\underline{Hint}: After a change of variables, relevant integrals will look like integrals against a normal density. You should be able to express the answer using $\Phi$, the standard normal cumulative distribution function.)
Also, determine the mean and variance for a variable $X$ with this density.

\begin{proof}[\underline{\textbf{Solution}}] $\newline$
	\begin{align*}
		 e^{A(\eta)} &= \int_0^\infty e^{-\eta x - 2\sqrt{x}}\frac{1}{\sqrt{x}} dx \\
		 		  & \text{To the integral is finite, } \eta \in \Xi = [0, \infty), \\
		 		  & \text{Let } t = \sqrt x \Rightarrow t^2=x, ~ 2tdt = dx \\
				  &= \int_0^\infty \frac{1}{t} e^{-\eta t^2-2t}2t dt \\
				  &= 2\int_0^\infty e^{-\eta\left(t^2+\frac{2}{\eta} t\right)}dt \\
				  &= 2\int_0^\infty e^{-\frac{2\eta}{2}\left(t+\frac{1}{\eta}\right)^2} dt \cdot e^{-\eta (-\frac{1}{\eta^2})} \\
				  & \text{Since inner term of integration } \sim N( -1/\eta, 1/2\eta ), \\
				  &= 2\sqrt{2\pi}/\sqrt{2\eta} \cdot e^{\frac{1}{\eta}}\Phi\left(\frac{-1/\eta}{\sqrt{1/(2\eta)}} \right) \\
				  &= \frac{2\sqrt{\pi}}{\sqrt\eta}e^{1/\eta}\Phi( -\sqrt{2/\eta} ) \\
		\Rightarrow A(\eta) &= \log 2\sqrt\pi -\frac{1}{2}\log\eta + \frac{1}{\eta} + \log\Phi\left(-\sqrt{\frac{2}{\eta}}\right).
	\end{align*}
	$$ \therefore p_\eta(x) = \exp\big\{ -\eta x -2\sqrt{x} - A(\eta) \big\} \frac{1}{\sqrt x}, ~~~ x > 0. $$
	The mean of $X$ is
	$$ E_\eta(X) = -E_\eta T = -A'(\eta) = \frac{1}{2\eta} + \frac{1}{\eta^2} - \frac{\frac{\sqrt{2}}{2}\eta^{-3/2} \phi\left(-\sqrt{\frac{2}{\eta}}\right)}{\Phi\left(-\sqrt{\frac{2}{\eta}}\right)}. $$
\end{proof}


%% # 2.6
\section{Problem 2.6}
Find the natural parameter space $\Xi$ and densities $p_\eta$ for a canonical two-parameter exponential family with $\mu$ counting measure on $\{0,1,2\}$, $T_1(x) = x, ~ T_2(x) = x^2$, and $h(x) = 1$ for $x \in \{0,1,2\}$.

\begin{proof}[\underline{\textbf{Solution}}]
	\begin{align*}
		e^{A(\eta)} &= \sum_{x=0}^2 e^{\eta_1x+\eta_2x^2} \\
				 &= 1 + e^{\eta_1 +\eta_2} + e^{2\eta_1 + 4\eta_2}.
	\end{align*}
	To the sum is finite, $\eta_1 + \eta_2 < \infty ~\& ~  2\eta_1+4\eta_2 < \infty \Rightarrow \eta \in \Xi = \bbR^2.$
	$$ \therefore p_\eta(x) = \exp\left[ \eta_1x + \eta_2x^2 -\left( 1+e^{\eta_1+\eta_2} + e^{2\eta_1+4\eta_2} \right) \right]. $$
\end{proof}


%% # 2.7
\section{Problem 2.7}
Suppose $X_1, \dots, X_n$ are independent geometric variables with $p_i$ the success probability for $X_i$.
Suppose these success probabilities are related to a sequence of "independent" variables $t_1,\dots, t_n$, viewed as known constants, through
$$ p_i = 1- \exp(\alpha + \beta t_i), ~~~~ i=1,\dots, n. $$
Show that the joint densities for $X_1, \dots, X_n$ form a two-parameter exponential family, and identify the statistics $T_1$ and $T_2$.

\begin{proof}[\underline{\textbf{Solution}}] $\newline$
	Since $X_i \overset{\mathrm{ind}}{\sim} Geo(p_i)$,  then the joint density is represented by
	\begin{align*}
	 	\prod_{i=1}^n p(x_i) &= \prod_{i=1}^n (1-p_i)^{x_i}p_i = \exp[\sum_{i=1}^nx_i\log(1-p_i) + \sum_{i=1}^n\log p_i] \\
					       &= \exp[\sum_{i=1}^nx_i(\alpha + \beta t_i) + \sum_{i=1}^n\log(1-\exp(\alpha+\beta t_i))].
	\end{align*}
	$\therefore T_1 = \sum_{i=1}^n X_i, ~ T_2 = \sum_{i=1}^nt_iX_i.$
\end{proof}


%% # 2.8
\section{Problem 2.8}
Assume that $X_1, \dots, X_n$ are independent random variables with $X_i \sim N(\alpha + \beta t_i, 1)$, where $t_1,\dots, t_n$ are observed constants and $\alpha$ and $\beta$ are unknown parameters.
Show that the joint densities for $X_1, \dots, X_n$ form a two-parameter exponential family, and identify the statistics $T_1$ and $T_2$.

\begin{proof}[\underline{\textbf{Solution}}] $\newline$
	The joint density of $X_1,\dots, X_n$ is
	$$ \prod_{i=1}^n \frac{1}{\sqrt{2\pi}} \exp\left[ -\frac{1}{2}(x_i - \alpha-\beta t_i)^2 \right] = (2\pi)^{-\frac{n}{2}} \exp \left[ \sum_i x_i(\alpha+\beta t_i) - \frac{1}{2} \sum_i (\alpha + \beta t_i) \right] \exp\left( -\sum_i \frac{x_i^2}{2}\right). $$
	$\therefore T_1 = \sum_{i=1}^n X_i, ~ T_2 = \sum_{i=1}^n t_iX_i.$
\end{proof}


%% # 2.9
\section{Problem 2.9}
Suppose that $X_1, \dots, X_n$ are independent Bernoulli variables (a random variable is Bernoulli if it only takes on values 0 and 1) with
$$ P(X_i = 1) = \frac{\exp(\alpha+\beta t_i)}{1 + \exp(\alpha + \beta t_i)}. $$
Show that the joint distribution for $X_1, \dots, X_n$ form a two-parameter exponential family, and identify the statistics $T_1$ and $T_2$.

\begin{proof}[\underline{\textbf{Solution}}] $\newline$
	The joint density of $X_1, \dots, X_n$ is
	\begin{align*}
		\prod_{i=1}^n p_i^{x_i}(1-p_i)^{1-x_i} &= \exp\left[ \sum_i x_i\log p_i + \sum_i (1-x_i)\log (1-p_i) \right] \\
									&= \exp \left[ \sum_i x_i\left(\alpha + \beta t_i - \log\left(1+e^{\alpha+\beta t_i}\right)\right) + \sum_i (1-x_i)\left(1-\log\left(1+e^{\alpha+\beta t_i}\right)\right) \right] \\
									&= \exp\left[ \sum_i x_i(\alpha + \beta t_i) - \sum_i \log\left( 1+e^{\alpha+\beta t_i} \right) \right].
	\end{align*}
	$\therefore T_1 = \sum_{i=1}^n X_i, ~ T_2 = \sum_{i=1}^n t_iX_i.$
\end{proof}


%% # 2.15
\section{Problem 2.15}
For an exponential family in canonical form, $ET_j = \partial A(\eta)/\partial \eta_j$.
This can be written in vector form as $ET = \nabla A(\eta)$.
Derive an analogous differential formula for $E_\theta T$ for an $s$-parameter exponential family that is not in canonical form.
Assume that $\Omega$ has dimension $s$. \\
\underline{Hint}: Differentiation under the integral sign should give a system of linear equations. Write these equations in matrix form.

\begin{proof}[\underline{\textbf{Solution}}] $\newline$
	Note that the exponential family form is
	\begin{align*}
		p_\eta(x) &= \exp\left[ \sum_{i=1}^s \eta_i T_i(x) - A(\eta) \right]h(x) \\
			       &= \exp\left[ \sum_{i=1}^s \eta_i(\theta)T_i(x) - B(\theta) \right]h(x) \\
			       &= p_\theta(x).
	\end{align*}
	Thus, 
	$$ e^{B(\theta)} = \int \exp\left[ \sum_{i=1}^s \eta_i(\theta)T_i(x) \right]h(x) d\mu(x). $$
	By differentiating with respect to $\theta_i$,
	$$ e^{B(\theta)} \frac{\partial B(\theta)}{\partial \theta_i} = \int \left( \sum_{j=1}^s \frac{\partial \eta_j(\theta)}{\partial \theta_i}T_j(x) \right) \exp\left[ \sum_{j=1}^s \eta_j(\theta)T_j(x) \right]h(x) d\mu(x). $$
	\begin{align*}
		\frac{\partial B(\theta)}{\partial \theta_i} &= \int \left( \sum_{j=1}^s \frac{\partial \eta_j(\theta)}{\partial \theta_i}T_j(x) \right) p_\theta (x) d\mu(x) \\
									   &= \sum_{j=1}^s \frac{\partial \eta_j(\theta)}{\partial \theta_i} E_\theta T_j, ~~~~ i = 1, \dots, s.
	\end{align*}
\end{proof}


%% # 2.17
\section{Problem 2.17}
Let $\mu$ denote counting measure on $\{1,2,\dots\}$.
One common definition for $\sum_{k=1}^\infty f(k)$ is $\limn \sum_{k=1}^n f(k)$, and another definition is $\int f d\mu$.
\begin{itemize}
	\item[a)] Use the dominated convergence theorem to show that the two definitions give the same answer when $\int |f| d\mu < \infty$.\\
		\underline{Hint}: Find functions $f_n, ~ n=1,2,\dots$, so that $\sum_{k=1}^n f(k) = \int f_n d\mu$.
            	\begin{proof}[\underline{\textbf{Solution}}] $\newline$
            		Let $f_n(k) = f(k), ~ \forall k \le n$, $f_n(k) = 0, ~ \forall k > n$.
			Then, $f_n \to f$ pointwise and $|f_n| \le |f|$.
			Also $f_n$ is simple and $\int f_n d\mu = \sum_{k=1}^n f(k)$. \\
			By D.C.T., $\int f_n d\mu \to \int f d\mu.$
            	\end{proof}

	\item[b)] Use the monotone convergence theorem, give in Problem 1.25, to show the definitions agree if $f(k) \ge 0$ for all $1,2,\dots$.
            	\begin{proof}[\underline{\textbf{Solution}}] $\newline$
            		Define $f_n$ and $f$ like part (a).
			Then, $0 \le f_1 \le f_2 \le \cdots$. \\
			By M.C.T., $\sum_{k=1}^n f(k) = \int f_nd\mu \to \int f d\mu$.
            	\end{proof}
	
	\item[c)] Suppose $\limn f(n) = 0$ and that $\int f^+d\mu = \int f^-d\mu = \infty$ (so that $\int f d\mu$ is undefined.)
		Let $K$ be an arbitrary constant. Show that the list $f(1), f(2), \dots$ can be rearranged to form a new list $g(1), g(2), \dots$ so that
		$$ \limn \sum_{k=1}^n g(k) = K. $$
            	\begin{proof}[\underline{\textbf{Solution}}] $\newline$
			Take positive parts of the sequence $f$ until the sum is exceed $K$.
			And then, take negative parts of the sequence $f$ until the sum is below $K$.
			Repeat this procedure like the figure below. {\color{blue} (In the figure, $g(k)$ is typo. $\sum g(k)$ is correct.) }
			\begin{figure}[!h]
				\center
				\includegraphics[width = 0.3 \textwidth]{figure/2-17.jpeg}
			\end{figure}
			Then, the sum of the sequence goes to 0 for $k > k'$ where $k'$ is the number when the sum of the rearranged sequence exceed $K$, first.
			Let this sequence as $g$.
			Then, $\limn \sum_{k=1}^n g(k) \to K.$
            	\end{proof}
\end{itemize}


%% # 2.19
\section{Problem 2.19}
Let $p_n, n = 1, 2, \dots,$ and $p$ be probability densities with respect to a measure $\mu$, and let $P_n, n=1,2,\dots$, and $P$ be the corresponding probability measures.
\begin{itemize}
	\item[a)] Show that if $p_n(x) \rightarrow p(x)$ as $n \to \infty$, then $\int |p_n-p| d\mu \to 0$. \\
		\underline{Hint}: First use the fact that $\int(p_n-p)d\mu = 0$ to argue that $\int |p_n-p|d\mu = 2\int(p-p_n)^+d\mu$.
		Then use dominated convergence.
            	\begin{proof}[\underline{\textbf{Solution}}] $\newline$
            		Since $p_n$ and $p$ be probability densities, $\int (p_n-p)d\mu = \int p_n d\mu - \int p d\mu = 1-1= 0$.
			Since $p_n - p = (p-p_n)^+  - (p-p_n)^-$,
			$$ \int (p_n-p) d\mu = \int (p-p_n)^+ d\mu  - \int (p-p_n)^- d\mu = 0, $$
			$$\therefore \int (p-p_n)^+ d\mu  = \int (p-p_n)^- d\mu. $$
			
			Also $|p_n - p| = (p-p_n)^+ + (p-p_n)^-$, 
			$$ \int |p_n - p| d\mu = \int (p-p_n)^+ d\mu + \int (p-p_n)^- d\mu = 2\int (p-p_n)^+ d\mu. $$
			Also note that $|(p-p_n)^+| \le p \Rightarrow |(p-p_n)^+|$ is integrable. ($\because$ $p$ is a probability density $\Rightarrow p$ is integrable.)
			From the condition, $(p_n \to p) \Rightarrow \big(p(x) - p_n(x)\big)^+ \to 0$. \\
			By D.C.T., 
			$$ \int |p_n - p| d\mu = 2\int (p-p_n)^+ d\mu \to 0.$$
            	\end{proof}

	\item[b)] Show that $|P_n(A)-P(A)| \le \int|p_n - p|d\mu$. \\
		\underline{Hint}: Use indicators and the bound $|\int f d\mu| \le \int |f|d\mu$.
            	\begin{proof}[\underline{\textbf{Solution}}]
			\begin{align*}
				LHS &= \left| \int 1_A dP_n - \int 1_A dP \right| \\
				       &= \left|\int 1_A(p_n - p) d\mu\right| \\
				       &\le \int \left| 1_A(p_n - p) \right| d\mu \\
				       &\le \int \left| p_n - p \right| d\mu.
			\end{align*}
            	\end{proof}
\end{itemize}
Remark: Distributions $P_n, ~ n \ge 1$, are said to {\em converge strongly} to $P$ if $\sup_A|P_n(A) - P(A)| \to 0$.
The two parts above show that pointwise convergence of $p_n$ to $p$ implies strong convergence.
This was discovered by Scheff\'{e}.


%% # 2.22
\section{Problem 2.22}
Suppose $X$ is absolutely continuous with density
$$ p_\theta(x) = \begin{cases}
\frac{e^{-(x-\theta)^2/2}}{\sqrt{2\pi}\Phi(\theta)}, ~~~ x > 0, \\
0, ~~~~~~~~~~~~~~ \text{otherwise.}
\end{cases} $$
Find the moment generating function of $X$. Compute the mean and variance of $X$.

\begin{proof}[\underline{\textbf{Solution}}]
	$$ p_\theta(x) = \exp\left[ -\frac{1}{2}(x^2-2x\theta+\theta^2) - \log\Phi(\theta) \right]\frac{1}{\sqrt{2\pi}}, ~~ x > 0 $$
	is the exponential family with
	$T = X, ~ A(\theta) = \frac{\theta^2}{2} + \log\Phi(\theta).$
	Therefore, the moment generating function of $T=X$ is
	\begin{align*}
		M_X(u) &= \exp\left[ A(\theta + u) - A(\theta) \right] \\
			     &= \exp\left[ \frac{1}{2}(\theta+u)^2 - \frac{1}{2}\theta^2 \right] \Phi(\theta+u) / \Phi(\theta) \\
			     &= \exp\left( \theta u + \frac{1}{2}u^2 \right) \Phi(\theta+u) / \Phi(\theta).
	\end{align*}
	Meanwhile, the {\em c.g.f.} of $T=X$ for exponential family is $K_X(u) = A(\theta + u) - A(\theta)$.
	Thus, 
	$$ EX = K_X'(u) = A'(\theta) = \theta + \frac{\phi(\theta)}{\Phi(\theta)}, $$
	$$ \Var(X) = K_X''(u) = A''(\theta) = 1 + \frac{\phi '(\theta)\Phi(\theta) - \phi(\theta)^2}{\Phi(\theta)^2}. $$
	$$ \left( \because \phi(x) = e^{-x^2/2} / \sqrt{2\pi} \Rightarrow \phi'(x) = -x\phi(x). \right) $$
\end{proof}


%% # 2.23
\section{Problem 2.23}
Suppose $Z \sim N(0,1)$. Find the first four cumulants of $Z^2$. \\
\underline{Hint}: Consider the exponential family $N(0,\sigma^2)$.

\begin{proof}[\underline{\textbf{Solution}}] $\newline$
	Let $X \sim N(0, \sigma^2)$. Then, its density is
	$$ p_\sigma(x) = \frac{1}{\sqrt{2\pi}} e^{-\frac{x^2}{s\sigma^2}} = \frac{1}{\sqrt{2\pi}}\exp\left[ -\frac{1}{2\sigma^2}x^2-\log\sigma \right], $$
	and also be the exponential family. \\
	Reparameterize $\eta = -\frac{1}{2\sigma^2}$, then $T = X^2$ and $A(\eta) = \frac{1}{2}\log -\frac{1}{2\eta} = -\frac{1}{2} \log(-2\eta)$.
	Thus, the cumulative generating function of $T=X^2$ is
	\begin{align*}
		K_{X^2}(u) &= A(\eta+u) - A(\eta) \\
				  &= -\frac{1}{2}\left[ \log\big(-2(\eta+u)\big) + \log(-2\eta) \right].
	\end{align*}
	If $\sigma = 1$, then $\eta = -\frac{1}{2}$ and $X^2 = Z^2$.
	Therefore, $$ K_{Z^2}(u) = -\frac{1}{2}\left[ \log(-2u + 1)\right]. $$
	$$ K_{Z^2}'(u) = \frac{1}{1-2u}, ~~~ K_{Z^2}''(u) = \frac{2}{(1-2u)^2}, $$
	$$ K_{Z^2}^{(3)}(u) = 8(1-2u)^{-3}, ~~~ K_{Z^2}^{(4)}(u) = 48(1-2u)^{-4}.$$
	$\therefore$ the cumulants are 1, 2, 8, and 48, respectively.
\end{proof}


%% # 2.24
\section{Problem 2.24}
Find the first four cumulants of $T=XY$ when $X$ and $Y$ are independent standard normal variates.

\begin{proof}[\underline{\textbf{Solution}}] $\newline$
	Since $X$ and $Y$ are independent, the function of these r.v.s are also independent. (i.e. $g(X)$ and $g(Y)$ are independent for any function $g$.)
	Then, first 2 cumulants are
	$$ \kappa_1= ET = EXY = EXEY = 0, $$
	$$ \kappa_2 = E(T-ET)^2 = EX^2Y^2 = EX^2EY^2 = 1. $$
	
	For third cumulants, we need to obtain $EX^3$.
	\begin{align*}
		EX^3 &= \int x^3 \frac{1}{\sqrt{2\pi}} e^{-\frac{x^2}{2}}dx \\
			 & \text{Let } t = x^2/2 \Rightarrow xdx = dt \\
			 &= \frac{2}{\sqrt{2\pi}} \int_0^\infty te^{-t}dt \\
			 &= \frac{2}{\sqrt{2\pi}} \left[ -te^{-t}\big|_0^\infty + \int_0^\infty e^{-t}dt \right] \\
			 &= 0.
	\end{align*}
	Thus, the third cumulants is
	$$ \kappa_3 = E(T - ET)^3 = EX^3Y^3 = EX^3EY^3 = 0. $$
	
	Similarly, we need to obtain $EX^4$.
	{\color{blue} 자꾸 적분이 틀림... 다시 해보기 Gamma dist 형태 사용}
	\begin{align*}
		EX^4 &= \int x^4 \frac{1}{\sqrt{2\pi}} e^{-\frac{x^2}{2}}dx \\
			 & \text{Let } t = x^2/2 \Rightarrow xdx = dt \\
	\end{align*}

	$$ \kappa_4 = E(T-ET)^4 - 3\Var(T)^2 = EX^4EY^4 - 3\cdot 1^2 = 4. $$
\end{proof}


%% # 2.25
\section{Problem 2.25}
Find the third and fourth cumulants of the geometric distribution.

\begin{proof}[\underline{\textbf{Solution}}] 
	$$f_p(x) = (1-p)^xp = \exp\left[ x\log(1-p) + \log p \right]$$
	is the exponential family.
	Reparameterize $\eta = -\log(1-p)$ ($\Rightarrow p = 1-e^\eta$), then the above form be the canonical exponential family with $T=X$ and $A(\eta) = -\log(1-e^\eta)$.
	
	Thus, the cumulative generating function of $T=X$ is
	$$ K_X(u) = A(\eta + u) - A(\eta) = -\log(1-e^{\eta+u}) + \log(1-e^\eta).$$
	Then, the first cumulants is
	$$ K_X'(0) = A'(\eta) = \frac{e^\eta}{1-e^\eta} = \frac{1-p}{p}.$$
	The higher order cumulants are easily obtained by using the chain rule.
	By using the $p' = \frac{dp}{d\eta} = -e^\eta = p-1$,
	$$ K_X''(0) = A''(\eta) = \frac{d}{dp}A'(\eta) \frac{dp}{d\eta} = \frac{-p - (1-p)}{p^2}(p-1) = \frac{1}{p^2} - \frac{1}{p}, $$
	$$ K_X^{(3)}(0) = A^{(3)}(\eta) = \frac{d}{dp}A''(\eta) p' = -\frac{2p'}{p^3} + \frac{p'}{p^2} = -\frac{2-2p}{p^3} + \frac{p-1}{p^2} = \frac{2}{p^3} - \frac{3}{p^2} + \frac{1}{p}, $$
	$$ K_X^{(4)}(0) = A^{(4)}(\eta) = \frac{d}{dp}A^{(3)}(\eta) p' = -\frac{6p'}{p^4} + \frac{6p'}{p^3} - \frac{p'}{p^2} = \frac{6}{p^4}-\frac{12}{p^3}+\frac{7}{p^2}-\frac{1}{p}. $$
\end{proof}


%% # 2.26
\section{Problem 2.26}
Find the third cumulant and third moment of the binomial distribution with $n$ trials and success probability $p$.

\begin{proof}[\underline{\textbf{Solution}}] 
	$$ p(x) = {n \choose x} p^x(1-p)^{n-x} = {n \choose x}\left(\frac{p}{1-p}\right)^x(1-p)^n = \exp\left[  x\log\frac{p}{1-p} + n\log(1-p)\right] {n \choose x} $$
	is the exponential family for $\eta = \log\frac{p}{1-p}$ with $T=X$ and $A(\eta) = -n\log(1-p)$.
	To use the chain rule,
	$$ p = \frac{e^\eta}{1+e^\eta} \Rightarrow \frac{dp}{d\eta} = \frac{e^\eta(1+e^\eta)-(e^\eta)^2}{(1+e^\eta)^2} = p(1-p). $$
	By using the above fact,
	$$ \kappa_1 = A'(\eta) = \frac{d}{dp}A(\eta) \frac{dp}{d\eta} = \frac{n}{1-p}p(1-p) = np, $$
	$$ \kappa_2 = A''(\eta) = \frac{d}{dp}A'(\eta) \frac{dp}{d\eta} = np(1-p), $$
	$$ \kappa_3 = A'''(\eta) = \frac{d}{dp}A''(\eta) \frac{dp}{d\eta} = (n-2np)p(1-p) = np(1-p)(1-2p). $$
	
	The third moment for $T=X$ can be obtained by
	$$ EX^3 = \kappa_3 + 3\kappa_1\kappa_2 + \kappa_1^3. $$
\end{proof}


%% # 2.27
\section{Problem 2.27}
Let $T$ be a random vector in $\bbR^2$.
\begin{itemize}
	\item[a)] Express $\kappa_{2,1}$ as a function of the moments of $T$.
            	\begin{proof}[\underline{\textbf{Solution}}] $\newline$
            		To make simple derivation, we denote $f_{ij}(u) = \frac{\partial^{i+j}f(u)}{\partial u_1^i \partial u_2^j}$.
			Note that $K(u) = \log M(u)$, by taking derivative,
			$$ K_{10} = \frac{M_{10}}{M}, ~~~ K_{20} = \frac{M_{20}M-M_{10}^2}{M^2}, $$
			\begin{align*}
				K_{21} &= \frac{1}{M^4}\left[ (M_{21}M + M_{20}M_{01}-2M_{10}M_{11})M^2 - 2MM_{01}(M_{20}M-M_{10}^2) \right] \\
					   &= \frac{1}{M^3}[ M_{21}M^2 + M_{20}M_{01}M - 2M_{10}M_{11}M - 2M_{01}M_{20}M-2M_{01}M_{10}^2 ] \\
					   &= \frac{1}{M^3}[ M_{21}M^2 - M_{20}M_{01}M - 2M_{10}M_{11}M -2M_{01}M_{10}^2 ].
			\end{align*}
			Taking $u=0$, then
			$$ \kappa_{2,1} = K_{2,1}|_{u=0} = ET_1^2T_2 - (ET_2)(ET_1^2) - 2(ET_1)(ET_1T_2) -2(ET_2)(ET_1)^2.$$
            	\end{proof}

	\item[b)] Assume $ET_1 = ET_2 = 0$ and give an expression for $\kappa_{2,2}$ in terms of moments of $T$.
            	\begin{proof}[\underline{\textbf{Solution}}] $\newline$
            		$\kappa_{2,2}$ can be obtain by one more derivative for $K_{21}$. 
			{\color{blue} (Too complicated. See Keener page 462.) }
            	\end{proof}
\end{itemize}


%% # 2.28
\section{Problem 2.28}
Suppose $X \sim \Gamma(\alpha, 1/\lambda)$, with density
$$ \frac{\lambda^\alpha x^{\alpha-1}e^{-\lambda x}}{\Gamma(\alpha)}, ~~~~ x > 0. $$
Find the cumulants of $T=(X, \log X)$ of order 3 or less.
The answer will involve $\psi(\alpha) = d\log\Gamma(\alpha)/d\alpha = \Gamma'(\alpha)/\Gamma(\alpha)$.

\begin{proof}[\underline{\textbf{Solution}}] $\newline$
	Since Gamma distribution is the exponential family,
	\begin{align*}
		p(x) &= \exp\big[ \alpha \log\lambda + (\alpha -1)\log x - \lambda x - \log \Gamma(\alpha) \big] \\
		       &= \exp\big[ -\lambda x + \alpha \log x - \big(\log\Gamma(\alpha) - \alpha\log \lambda\big) - \log x \big].
	\end{align*}
	Let $\eta = (-\lambda, \alpha)$, and $A(\eta) = \log\Gamma(\eta_2) - \eta_2\log(-\eta_1)$. \\
	Then, the cumulants are obtained by taking derivatives:
	$$ \kappa_{1,0} = \frac{\partial A(\eta)}{\partial \eta_1} = -\frac{\eta_2}{\eta_1} = \frac{\alpha}{\lambda}, $$
	$$ \kappa_{0,1} = \frac{\partial A(\eta)}{\partial \eta_2} = \psi(\eta_2)-\log(-\eta_1) = \psi(\alpha)-\log \lambda, $$
	
	$$ \kappa_{2,0} = \frac{\partial^2 A(\eta)}{\partial \eta_1^2} = \frac{\eta_2}{\eta_1^2} = \frac{\alpha}{\lambda^2}, $$
	$$ \kappa_{1,1} = \frac{\partial^2 A(\eta)}{\partial \eta_1 \partial\eta_2} = -\frac{1}{\eta_1} = \frac{1}{\lambda}, $$
	$$ \kappa_{0,2} = \frac{\partial^2 A(\eta)}{\partial \eta_2^2} = \psi'(\eta_2) = \psi'(\alpha), $$
	
	$$ \kappa_{3,0} = \frac{\partial^3 A(\eta)}{\partial \eta_1^3} = -\frac{2\eta_2}{\eta_1^3} = \frac{2\alpha}{\lambda^3}, $$
	$$ \kappa_{2,1} = \frac{\partial^3 A(\eta)}{\partial \eta_1^2 \partial\eta_2} = \frac{1}{\eta_1^2} = frac{1}{\lambda^2}, $$
	$$ \kappa_{1,2} = \frac{\partial^3 A(\eta)}{\partial \eta_1 \partial\eta_2^2} = 0, $$
	$$ \kappa_{0,3} = \frac{\partial^3 A(\eta)}{\partial \eta_2^3} = \psi''(\eta_2) = \psi''(\alpha). $$
\end{proof}


% Chapter 3. Risk, Sufficiency, Completeness, and Ancillarity
\chapter{Risk, Sufficiency, Completeness, and Ancillarity}

%% # 3.2
\section{Problem 3.2}
Suppose data $X_1, \dots, X_n$ are independent with 
$$ P_\theta(X_i \le x) = x^{t_i\theta}, ~~~~~ x \in (0, 1). $$
where $\theta > 0$ is the unknown parameter, and $t_1, \dots, t_n$ are known positive constants.
Find a one-dimensional sufficient statistic $T$.

\begin{proof}[\underline{\textbf{Solution}}] $\newline$
	Take a first derivative for the above cdf, then the pdf of $X$ be $p_\theta(x) = t_i\theta x_i^{t_i\theta-1}$.
	Then, the joint density is
	$$ p_\theta(x_1,\dots,x_n) = \prod_{i=1}^n t_i\theta x_i^{t_i\theta-1} = \theta^n \left( \prod_{i=1}^n x_i^{t_i} \right)^\theta \frac{\prod_{i=1}^n t_i}{\prod_{i=1}^n x_i}. $$
	By factorization theorem, the sufficient statistic for $\theta$ is
	$$T = \prod_{i=1}^n x_i^{t_i}.$$	
\end{proof}


%% # 3.3
\section{Problem 3.3}
An object with weight $\theta$ is weighted on scales with different precision.
The data $X_1,\dots,X_n$ are independent, with $X_i \sim N(\theta, \sigma_i^2), ~ i=1,\dots, n$, with the standard deviations $\sigma_1,\dots, \sigma_n$ known constants.
Use sufficiency to suggest a weighted average of $X_1,\dots,X_n$ to estimate $\theta$.
(A weighted average would have form $\sum_{i=1}^n w_iX_i$, where the $w_i$ are positive and sum to one.)

\begin{proof}[\underline{\textbf{Solution}}] $\newline$
	The pdf of $X_i$ is 
	$$ p_\mu(x_i) = \frac{1}{\sqrt{2\pi}\sigma_i}e^{-\frac{1}{2\sigma_i^2}(x_i-\theta)^2} = \frac{1}{\sqrt{2\pi}\sigma_i} \exp\left[ \frac{\theta x_i}{\sigma_i^2} - \frac{x_i^2}{2\sigma_i^2} - \frac{\theta^2}{2\sigma_i^2} \right]. $$
	Then, the joint density is
	$$ p_\mu(x_1,\dots,x_n) = \frac{1}{(2\pi)^{n/2}\prod_i \sigma_i}\exp\left[ \sum_i\frac{\theta x_i}{\sigma_i^2} - \sum_i\frac{x_i^2}{2\sigma_i^2} - \sum_i\frac{\theta^2}{2\sigma_i^2} \right]. $$
	By factorization theorem, 
	$$T = \sum_{i=1}^n \frac{x_i}{\sigma_i^2}.$$
	To estimate $\theta$ using a weighted average form, let 
	$$w_i = \frac{T}{\sum_{i=1}^n \sigma_i^{-2}},$$
	then the weighted average of $X_1,\dots, X_n$ be an estimator for $\theta$ satisfied the sufficiency.
\end{proof}


%% # 3.4
\section{Problem 3.4}
Let $X_1,\dots,X_n$ be a random sample from an arbitrary discrete distribution $P$ on $\{1,2,3\}$.
Find a two-dimensional sufficient statistic.

\begin{proof}[\underline{\textbf{Solution}}] $\newline$
	Let $p_i = P(\{i\})$ and $n_i(x) = \sum_{j=1}^n  1_{\{ x_j = i \}}(x)$ for $i=1,2,3$.
	Then, $n_1(x) + n_2(x) + n_3(x) = n$.
	Since $X_i$s are i.i.d., the joint pdf is
	\begin{align*}
		P(X_1=x_1, \dots, X_n = x_n) &= p_1^{n_1(x)}p_2^{n_2(x)}p_3^{n_3(x)} \\
							      &= p_1^{n_1(x)}p_2^{n_2(x)}p_3^{n - n_1(x)-n_2(x)}.
	\end{align*}
	By factorization theorem, the sufficient statistic $T = (n_1,n_2)$.
\end{proof}


%% # 3.6
\section{Problem 3.6}
The beta distribution with parameters $\alpha > 0$ and $\beta > 0$ has density
$$ f_{\alpha,\beta}(x) = \begin{cases}
\frac{\Gamma(\alpha+\beta)}{\Gamma(\alpha)\Gamma(\beta)} x^{\alpha-1}(1-x)^{\beta-1}, ~~~x\in(0,1), \\
0,~~~~~~~~~~~~~~~~~~~~~~~~~~~~~~~~\text{otherwise.}
\end{cases} $$
Suppose $X_1, \dots, X_n$ are i.i.d. from a beta distribution.
\begin{enumerate}
	\item[a)] Determine a minimal sufficient statistic (for the family of joint distributions) if $\alpha$ and $\beta$ vary freely.
		\begin{proof}[\underline{\textbf{Solution}}] $\newline$
			The joint density
			\begin{align*}
				f_{\alpha,\beta}(x_1,\dots,x_n) &= \exp\left[ \sum_i (\alpha-1)\log x_i + \sum_i (\beta-1)\log(1-x_i) + n\log \frac{\Gamma(\alpha+\beta)}{\Gamma(\alpha)\Gamma(\beta)} \right] \\
									       &= \exp\left[ \sum_i \alpha \log x_i + \sum_i \beta \log(1-x_i) + n\log \frac{\Gamma(\alpha+\beta)}{\Gamma(\alpha)\Gamma(\beta)} - \sum_i\log x_i(1-x_i) \right],
			\end{align*}
			is the exponential family and full-rank.
			Then,
			$$ T = \left( \sum_{i=1}^n \log X_i, \sum_{i=1}^n \log(1-X_i) \right) $$
			is complete. \\
			By factorization theorem, $T$ is also sufficient. \\
			Therefore, $T$ is the minimal sufficient statistic for $(\alpha, \beta)$. (See Theorem 3.19) 
		\end{proof}
		
	\item[b)] Determine a minimal sufficient statistic if $\alpha = 2\beta$.
		\begin{proof}[\underline{\textbf{Solution}}] $\newline$
			Plug into the joint density,
			\begin{align*}
				f_{\beta}(x_1,\dots,x_n) &= \exp\left[ \sum_i 2\beta \log x_i + \sum_i \beta \log(1-x_i) + n\log \frac{\Gamma(2\beta+\beta)}{\Gamma(2\beta)\Gamma(\beta)} - \sum_i\log x_i(1-x_i) \right] \\
									       &= \exp\left[ \beta \sum_i \left( \log x_i^2 + \log(1-x_i) \right) + n\log \frac{\Gamma(3\beta)}{\Gamma(2\beta)\Gamma(\beta)} - \sum_i\log x_i(1-x_i) \right]
			\end{align*}
			is the one-parameter exponential family and full-rank.
			Thus,
			$$T = \sum_i \left( \log x_i^2 + \log(1-x_i) \right) = 2T_1 +T_2$$
			is complete, and also sufficient by factorization theorem. \\
			Therefore, $T$ is the minimal sufficient statistic for $\beta$.
		\end{proof}
		
	\item[c)] Determine a minimal sufficient statistic if $\alpha = \beta^2$.
		\begin{proof}[\underline{\textbf{Solution}}] $\newline$
			Now, the joint density is
			\begin{align*}
				p_{\beta}(x_1,\dots,x_n) &= \exp\left[ \sum_i (\beta^2-1) \log x_i + \sum_i (\beta-1) \log(1-x_i) + n\log \frac{\Gamma(2\beta+\beta)}{\Gamma(2\beta)\Gamma(\beta)}\right] \\
								     &= \exp\left[ (\beta^2-1) T_1(x) + (\beta-1) T_2(x) + n\log \frac{\Gamma(\beta^2+\beta)}{\Gamma(\beta^2)\Gamma(\beta)}\right].
			\end{align*}
			Suppose $p_\theta(x) \propto_\theta p_\theta(y)$. Then W.L.O.G. we consider 2 cases as follows:
			$$ \frac{p_1(y)}{p_1(x)} = \frac{p_2(y)}{p_2(x)} \Longrightarrow \frac{p_2(x)}{p_1(x)} = \frac{p_2(y)}{p_1(y)}, $$
			$$ \frac{p_1(y)}{p_1(x)} = \frac{p_3(y)}{p_3(x)} \Longrightarrow \frac{p_3(x)}{p_1(x)} = \frac{p_3(y)}{p_1(y)}. $$
			Since $p_1(x) = p_1(y) = 1$, from the above equation,
			$$ 3T_1(x) + T_2(x) + n\log\frac{\Gamma(2^2+2)}{\Gamma(2^2)\Gamma(2)} = 3T_1(y) + T_2(y) + n\log\frac{\Gamma(2^2+2)}{\Gamma(2^2)\Gamma(2)}, $$
			$$ 8T_1(x) + 2T_2(x) + n\log\frac{\Gamma(3^2+3)}{\Gamma(3^2)\Gamma(3)} = 8T_1(y) + 2T_2(y) + n\log\frac{\Gamma(3^2+3)}{\Gamma(3^2)\Gamma(3)}. $$
			These 2 equations imply $T(x) = T(y)$. \\
			Therefore, $T$ is the minimal sufficient statistic by Theorem 3.11.
		\end{proof}
\end{enumerate}


%% # 3.7
\section{Problem 3.7}
{\em Logistic regression}.
Let $X_1, \dots, X_n$ be independent Bernoulli variables, with $p_i = P(X_i=1), ~ i=1,\dots,n$.
Let $t_1,\dots,t_n$ be a sequence of known constants that are related to the $p_i$ via
$$ \log\frac{p_i}{1-p_i} = \alpha + \beta t_i, $$
where $\alpha$ and $\beta$ are unknown parameters.
Determine a minimal sufficient statistic for the family of joint distributions.

\begin{proof}[\underline{\textbf{Solution}}] $\newline$
	From the above equation, 
	$$p_i = \frac{e^{\alpha+\beta t_i}}{1+e^{\alpha+\beta t_i}}.$$
	Then, the density of $X_i$ is
	\begin{align*}
		 p_{\alpha,\beta}(x_i) &= p_i^{x_i}(1-p_i)^{1-x_i} \\
		 				 &= \left( \frac{e^{\alpha+\beta t_i}}{1+e^{\alpha+\beta t_i}} \right)^{x_i} \left( \frac{1}{1+e^{\alpha+\beta t_i}} \right)^{1-x_i} \\
						 &= \exp\left[ x_i(\alpha+\beta t_i) - x_i\log(1+e^{\alpha+\beta t_i}) - (1-x_i)\log(1+e^{\alpha+\beta t_i}) \right] \\
						 &= \exp\left[ \alpha x_i + \beta x_it_i - \log(1+e^{\alpha+\beta t_i}) \right].
	\end{align*}
	Therefore, we can easily obtain the following sufficient statistic by factorization theorem for the joint density,
	$$T = \left( \sum_{i=1}^n X_i, \sum_{i=1}^n X_it_i \right).$$
	And also, the joint density is exponential family and full-rank. Thus $T$ is complete. \\
	$\therefore$ $T$ is minimal sufficient.
\end{proof}


%% # 3.8
\section{Problem 3.8}
The multinomial distribution, derived later in Section 5.3, is a discrete distribution with mass function
$$ \frac{n!}{x_1!\times \cdots \times x_s!}p_1^{x_1}\times \cdots \times p_s^{x_s},$$
where $x_1,\dots, x_s$ are non-negative integers summing to $n$, where $p_1,\dots, p_s$ are non-negative probabilities summing to one, and $n$ is the sample size.
Let $N_{11},N_{21},N_{22}$ have a multinomial distribution with $n$ trials and success probabilities $p_{11},p_{12},p_{21},p_{22}$.
(A common model for a two-by-two contingency table.)
\begin{enumerate}
	\item[a)] Give a minimal sufficient statistic if the success probabilities vary freely over the unit simplex in $\bbR^4$.
		(The unit simplex in $\bbR^p$ is the set of all vectors with non-negative entries summing to one.)
		\begin{proof}[\underline{\textbf{Solution}}] $\newline$
			The density be the exponential family as follows:
			\begin{align*}
				p(\mathbf{N}) &= \frac{n!}{N_{11}! N_{12}! N_{21}! N_{22}!} p_{11}^{N_{11}}p_{12}^{N_{12}}p_{21}^{N_{21}} p_{22}^{N_{22}} \\
				       &= \frac{n!}{N_{11}! N_{12}! N_{21}! N_{22}!} \exp\left[ N_{11}\log p_{11} + N_{12}\log p_{12} + N_{21}\log p_{21} + N_{22}\log p_{22}  \right] \tag{$*$} \\
				       &= \frac{n!}{N_{11}! N_{12}! N_{21}! N_{22}!} \exp\left[ N_{11}\log p_{11} + N_{12}\log p_{12} + N_{21}\log p_{21} \right. \\
				       &~~~~~~~~~~~~~~~~~~~~~~~~~~~~~~~~~~~\left. + (n- N_{11} - N_{12} - N_{21})\log(1-p_{11} - p_{12}- p_{21})  \right] \\
				       &= \frac{n!}{N_{11}! N_{12}! N_{21}! N_{22}!} \exp\left[ N_{11}\log \left(\frac{p_{11}}{1-p_{11} - p_{12}- p_{21}}\right) + N_{12}\log \left(\frac{p_{12}}{1-p_{11} - p_{12}- p_{21}}\right) \right. \\
				       &~~~~~~~~~~~~~~~~~~~~~~~~~~~~~~~~~~~\left. + N_{21}\log \left(\frac{p_{21}}{1-p_{11} - p_{12}- p_{21}}\right) + n\log(1-p_{11} - p_{12}- p_{21})  \right].
			\end{align*}
			Since it is the exponential family of full-rank, $T=(N_{11}, N_{12}, N_{21})$ is complete, and also sufficient by factorization theorem.
			Thus $T=(N_{11}, N_{12}, N_{21})$ is minimal sufficient. \\
			If we use the equation ($*$), the minimal sufficient statistic is  $T=(N_{11}, N_{12}, N_{21}, N_{22})$.
		\end{proof}
		
	\item[b)] Give a minimal sufficient statistic if the success probabilities are constrained so that $p_{11}p_{22} = p_{12}p_{21}$.
		\begin{proof}[\underline{\textbf{Solution}}] $\newline$
			{\color{blue} 확실하지 않음...} \\
			From the constraint, plug in $p_{11} = (p_{12}p_{21})/p_{22}$ and denote $C$ is the term which is not depend on parameters,
			\begin{align*}
				p(\mathbf{N}) &= C p_{11}^{N_{11}}p_{12}^{N_{12}}p_{21}^{N_{21}} p_{22}^{N_{22}} \\
						     &= C \left( \frac{p_{12}p_{21}}{p_{22}} \right)^{N_{11}}p_{12}^{N_{12}}p_{21}^{N_{21}} p_{22}^{N_{22}} \\
						     &= C p_{12}^{N_{11}+N_{12}} p_{21}^{N_{11}+N_{21}} p_{22}^{N_{22}-N_{11}} \\
						     &= C p_{12}^{N_{11}+N_{12}} p_{21}^{N_{11}+N_{21}} (1-p_{11}-p_{12} - p_{21})^{n-N_{12} - N_{21}-2N_{11}} \\
				       		     &= C \exp\left[ (N_{11} + N_{12})\log \left(\frac{p_{12}}{1-p_{11} - p_{12}- p_{21}}\right) \right. \\
				      		     &~~~~~~~~~~~~~\left. + (N_{11}+N_{21})\log \left(\frac{p_{21}}{1-p_{11} - p_{12}- p_{21}}\right) + n\log(1-p_{11} - p_{12}- p_{21})  \right].
			\end{align*}
			Since it is the exponential family of full-rank, $T=(N_{11} + N_{12}, N_{11}+N_{21})$ is complete, and also sufficient by factorization theorem.
			Thus $T=(N_{11} + N_{12}, N_{11}+N_{21})$ is minimal sufficient.
		\end{proof}
\end{enumerate}


%% # 3.9
\section{Problem 3.9}
Let $f$ be a positive integrable function on $(0,\infty)$. Define
$$ c(\theta) = 1/\int_\theta^\infty f(x)dx, $$
and take $p_\theta(x) = c(\theta)f(x)$ for $x > \theta$, and $p_\theta(x) = 0$ for $x \le \theta$.
Let $X_1,\dots,X_n$ be i.i.d. with common density $p_\theta$.
\begin{enumerate}
	\item[a)] Show that $M=\min\{X_1,\dots,X_n\}$ is sufficient.
		\begin{proof}[\underline{\textbf{Solution}}] $\newline$
			The joint density is
			\begin{align*}
				p_\theta(x_1,\dots, x_n) &= c(\theta)^n \prod_i f(x_i) \\
								     &= c(\theta)^n \prod_i f(x_i) \cdot 1_{\min\{x_1,\dots, x_n\} > \theta }.
			\end{align*}
			By factorization theorem, $M = \min\{X_1,\dots,X_n\}$ is sufficient.
		\end{proof}
		
	\item[b)] Show that $M$ is minimal sufficient.
		\begin{proof}[\underline{\textbf{Solution}}] $\newline$
			Suppose $p_\theta(x) \propto_\theta p_\theta(y)$.
			Then, the region of 0 value should be equal.
			Thus $T(x) = T(y)$ should be satisfied, therefore $T=M$ is minimal sufficient.
		\end{proof}
\end{enumerate}


%% # 3.10
\section{Problem 3.10}
Suppose $X_1,\dots,X_n$ are i.i.d. with common density $f_\theta(x) = (1+\theta x)/2, ~ |x| < 1; ~ f_\theta(x) = 0,$ otherwise, where $\theta \in [-1,1]$ is an unknown parameter.
Show that the order statistics are minimal sufficient.
(Hint: A polynomial of degree $n$ is uniquely determined by its value on a grid of $n+1$ points.)

\begin{proof}[\underline{\textbf{Solution}}] $\newline$
	The joint density is
	\begin{align*}
		f_\theta(x_1,\dots,x_n) &= 2^{-n}(1+\theta x_1)\cdots(1+\theta x_n) \\
						   &= 2^{-n}(1+\theta x_{(1)})\cdots(1+\theta x_{(n)}) 1_{\{-1 < x_{(1)} <\cdots < x_{(n)} < 1\}}.
	\end{align*}
	By factorization theorem, the order statistic $T = (X_{(1)}, \dots, X_{(n)})$ is sufficient. \\
	Now, suppose $p_\theta(x) \propto_\theta p_\theta(y)$.
	Since the joint density is the $n$-th order polynomials(다항식) with root(근) $-1/x_{(i)}$, the roots of $p_\theta(x)$ and $p_\theta(x)$ should be equal which implies $T(x) = T(y)$.
	Therefore, $T = (X_{(1)}, \dots, X_{(n)})$  is minimal sufficient.
\end{proof}


%% # 3.16
\section{Problem 3.16}
Let $X_1,\dots, X_n$ be a random sample from an absolutely continuous distribution with density
$$ f_\theta(x) = \begin{cases}
2x/\theta^2, ~~~x \in (0,\theta), \\
0, ~~~~~~~~~ \text{otherwise}.
\end{cases} $$
\begin{enumerate}
	\item[a)] Find a one-dimensional sufficient statistic $T$.
		\begin{proof}[\underline{\textbf{Solution}}] $\newline$
			The joint density is factorized as follows:
			$$ f_\theta(x_1,\dots,x_n) = 2^n\frac{x_1\cdots x_n}{\theta^{2n}} = \frac{2^n}{\theta^{2n}}x_{(1)}\cdots x_{(n)} 1_{x_{(1)} > 0} 1_{x_{(n)} < \theta}. $$
			By factorization theorem, $T = \max\{ X_1, \dots, X_n \}$ is the sufficient statistic for $\theta$.
		\end{proof}
		
	\item[b)] Determine the density of $T$.
		\begin{proof}[\underline{\textbf{Solution}}] $\newline$
			Since $T = \max\{ X_1, \dots, X_n \}$,
			\begin{align*}
				P(T \le t) &= \prod_{i=1}^n P(X_i \le t) \\
					      &~~\text{Note that } P(X_i \le t) = \int_0^t 2x/\theta^2 dx = t^2/\theta^2. \\
					      &= \frac{t^{2n}}{\theta^{2n}}
			\end{align*}
			Taking a derivative with respect to $t$, the density of $T$ is
			$$ p_\theta(t) = \frac{2n}{\theta^{2n}}t^{2n-1}, ~~ t \in (0,\theta). $$
		\end{proof}
		
	\item[c)] Show directly that $T$ is complete.
		\begin{proof}[\underline{\textbf{Solution}}] $\newline$
			Suppose $E_\theta(T) = c, ~ \forall \theta > 0$. Then,
			\begin{align*}
				&E_\theta\left[ f(T)-c \right] = \frac{2n}{\theta^{2n}}\int_0^\theta [f(t)-c] t^{2n-1} dt = 0 \\
				&\Rightarrow ~~ [f(t)-c]t^{2n-1}=0 ~~\text{ a.e.} ~ 0<t<\theta. \\
				&\Rightarrow ~~ f(T) = c.
			\end{align*}
			Therefore, $T$ is complete.
			 
			 \underline{Other solution} \\
			 \begin{align*}
				&E_\theta f(T) = \frac{2n}{\theta^{2n}}\int_0^\theta f(t) t^{2n-1} dt = c \\
				&\Rightarrow ~~ \int_0^\theta f(t) t^{2n-1} dt = c \theta^{2n}/2n, ~~ \forall \theta > 0 \\
				&\Rightarrow ~~ f(\theta)\theta^{2n-1} = c\theta^{2n-1} ~~\text{ a.e. } \theta \\
				&\Rightarrow ~~ f(t) = c ~~ \text{ a.e. } t.
			\end{align*}
		\end{proof}
\end{enumerate}


%% # 3.17
\section{Problem 3.17}
Let $X, X_1,X_2,\dots$ be i.i.d. from an exponential distribution with failure rate $\lambda$ (introduced in Problem 1.30).
\begin{enumerate}
	\item[a)] Find the density of $Y=\lambda X$.
		\begin{proof}[\underline{\textbf{Solution}}] $\newline$
			Note that the density of $X$ is $p_\lambda(x) = \lambda e^{-\lambda x}, ~ \lambda > 0$, and its cdf is easily obtained by integration, $ P(X \le x) = 1-e^{-\lambda x}$.
			Thus the cdf of $Y$ is
			$$ P(Y \le y) = P(X \le y/\lambda) = 1-e^{-y}, ~ y > 0, $$
			and the density is
			$$ p(y) = e^{-y}, ~ y>0. $$
		\end{proof}
		
	\item[b)] Let $\overline{X} = (X_1 + \cdots + X_n) / n$. Show that $\overline X$ and $(X_1^2 + \cdots + X_n^2)/\overline X^2$ are independent.
		\begin{proof}[\underline{\textbf{Solution}}] $\newline$
			The joint density of $X_1, \dots, X_n$ is
			$$ p_\lambda(x_1,\dots,x_n) = \lambda^ne^{-\lambda\sum_i x_i} = \lambda^n e^{-n\lambda \bar x}, $$
			is the full rank exponential family, and by factorization theorem, $T= \overline X$ is the complete sufficient statistic. \\
			Now, let $Y=\lambda X$, then  
			$$ V = \frac{(Y_1^2 + \cdots + Y_n^2)}{\overline Y^2} = \frac{(X_1^2 + \cdots + X_n^2)}{\overline X^2}. $$
			Since $Y$ does not depend on parameter $\lambda$, $V$ is ancillary for $P_\lambda$. \\
			$\therefore$ By Basu's theorem, $T$ and $V$ are independent.
		\end{proof}
\end{enumerate}


%% # 3.29
\section{Problem 3.29}
Find a function on $(0,\infty)$ that is bounded and strictly convex.

\begin{proof}[\underline{\textbf{Solution}}] $\newline$
	$f(x) = e^{-x}$ is bounded on range $(0,1)$ and strictly convex. \\
	Or $f(x) = 1/(x+1)$ is bounded on range $(0,1)$ and strictly convex.
\end{proof}


%% # 3.30
\section{Problem 3.30}
Use convexity to show that the canonical parameter space $\Xi$ of a one-parameter exponential family must be an interval.
Specifically, show that if $\eta_0 <\eta < \eta_1$, and if $\eta_0$ and $\eta_1$ both lie in $\Xi$, then $\eta$ must lie in $\Xi$.

\begin{proof}[\underline{\textbf{Solution}}] $\newline$
	Note that if a random variable $X$ is a canonical exponential family, then it has the form,
	$$ p_\eta(x) = \exp[\eta T(x) - A(\eta)]h(x), ~~ \eta \in \Xi. $$
	Since $\eta_0 < \eta < \eta_1$, we define $\eta = \gamma \eta_0 + (1-\gamma)\eta_1$ for some $\gamma \in (0,1)$.
	Since the exponential function is convex, 
	$$ \exp\big[ (\gamma \eta_0 + (1-\gamma)\eta_1)T(x) \big] = e^{\eta T(x)} < \gamma e^{\eta_0 T(x)} + (1-\gamma)e^{\eta_1T(x)}. $$
	From the definition of exponential family, $h(x) \ge 0$, and for the positive function, the inequality sign does not change by integration against $\mu$. Thus, following inequality is satisfied:
	$$ \int e^{\eta T(x)}h(x) d\mu(x) < \gamma \int e^{\eta_0 T(x)}h(x)d\mu(x) + (1-\gamma) \int e^{\eta_1T(x)}h(x)d\mu(x). $$
	Because it should be satisfied the exponential family, RHS is finite. ($\because$ RHS $= \gamma e^{A(\eta_0)} + (1-\gamma)e^{A(\eta_1)} < \infty$.) 
	Therefore, $\eta$ must lie in $\Xi$.
\end{proof}


%% # 3.31
\section{Problem 3.31}
Let $f$ and $g$ be positive probability densities on $\bbR$. Use Jensen's inequality to show that
$$ \int \log \left( \frac{f(x)}{g(x)} \right) f(x)dx > 0, $$
unless $f = g$ a.e. (If $f=g$, the integral equals zero.)
This integral is called the {\em Kullback-Liebler information}.

\begin{proof}[\underline{\textbf{Solution}}] $\newline$
	Suppose $X$ be a absolutely continuous random variable with density $f$.
	Now, define $Y= \frac{g(X)}{f(X)}$ , then
	$$ EY = \int \frac{g(x)}{f(x)}f(x) dx = \int g(x)dx = 1. $$
	Next, let $h(y) = \log\frac{1}{y} = -\log y$, then $h$ is strictly convex on $(0,\infty)$.
	Then, by Jensen's inequality,
	$$  Eh(Y)  = \int \log \left( \frac{f(x)}{g(x)} \right) f(x)dx \ge h\left( EY \right) = \log1 = 0. $$
	The equality holds when $Y$ is a constant. (then $Y=EY=1 \Rightarrow f(x)=g(x)$ a.e.) 
\end{proof}

% Chapter 4. Unbiased Estimation
\chapter{Unbiased Estimation}

%% # 4.1
\section{Problem 4.1}
Let $X_1, \dots, X_m$ and $Y_1,\dots, Y_n$ be independent variables with the $X_i$ a random sample from an exponential distribution with failure rate $\lambda_x$, and the $Y_j$ a random sample from an exponential distribution with failure rate $\lambda_y$.
\begin{enumerate}
	\item[a)] Determine the UMVU estimator of $\lambda_x / \lambda_y$.
            	\begin{proof}[\underline{\textbf{Solution}}] $\newline$
            		Note that if $X \sim \text{Exp}(\lambda)$ with rate parameter $\lambda$, the denstiy $p_\lambda(x) = \lambda e^{-\lambda x}$. \\
			Since $X_i, \forall i$ and $Y_j, \forall j$ are independent, the joint density is
			$$ p_{\lambda_x, \lambda_y}(x,y) = \lambda_x^n\lambda_y^n e^{-\lambda_x\sum_ix_i - \lambda_y \sum_iy_i}, $$
			which is the 2-parameter full-rank exponential family.
			Thus, by factorization theorem, $T=(\sum_{i=1}^m X_i, \sum_{i=1}^n Y_i)$ is the complete sufficient. \\
			Since $X_1,\dots,X_m$ are i.i.d. Exp($\lambda_x$) = $\Gamma(1, \lambda_x)$, $T_1 = \sum_{i=1}^m X_i \sim \Gamma(m, \lambda_x)$. Then,
			$$ E\frac{1}{T_1} = \int \frac{1}{t} \frac{\lambda_x^m}{\Gamma(m)} t^{m-1} e^{-\lambda_x t} dt = \frac{\Gamma(m-1) \lambda_x^m}{\Gamma(m)\lambda_x^{m-1}} = \frac{\lambda_x}{m-1}. $$
			Similarly, we can obtain $E T_2 = n / \lambda_y$. ($\because EY_i = 1/\lambda_y$) \\
			From the fact of the expectation of independent random variables,
			$$ E\frac{T_2}{T_1} = E\frac{1}{T_1} \cdot ET_2 = \frac{n\lambda_x}{(m-1)\lambda_y}. $$
			Therefore, 
			$$ \delta(x,y) = \frac{(m-1)T_2}{nT_1} $$
			is the unbiased estimator of $\lambda_x/\lambda_y$, and also UMVU since it is the function of complete sufficient statistic $T$. (By Theorem 4.4 with Rao-Blackwellization)
            	\end{proof}
	
	\item[b)] Under squared error loss, find the best estimator of $\lambda_x / \lambda_y$ of form $\delta = c\overline Y / \overline X$.
            	\begin{proof}[\underline{\textbf{Solution}}] $\newline$
			Let $\delta = c\overline Y / \overline X = d T_2/T_1$.
            		Then, the risk of $\delta = c\overline Y / \overline X$ and $\lambda_x / \lambda_y$ is
			\begin{equation}\label{eq:4.1}
				R\left( \frac{\lambda_y}{\lambda_x}, \delta \right) = E\left[ d\frac{T_2}{T_1} - \frac{\lambda_x}{\lambda_y}  \right]^2 = d^2E\frac{T_2^2}{T_1^2} - 2 \frac{\lambda_x}{\lambda_y} d E\frac{T_2}{T_1} +  \frac{\lambda_x^2}{\lambda_y^2}. 
			\end{equation}
			Since $T_1 \sim \Gamma(m, \lambda_x)$ and $T_2 \sim \Gamma(n, \lambda_y)$,
			$$ ET_2^2 = \Var(T_2) + \{ET_2\}^2 = n/\lambda_y^2 + n^2/\lambda_y^2, $$
			$$ E\frac{1}{T_1^2} = \int \frac{1}{t^2} \frac{\lambda_x^m}{\Gamma(m)} t^{m-1} e^{-\lambda_x t} dt = \frac{\Gamma(m-2) \lambda_x^2}{\Gamma(m)} = \frac{\lambda_x^2}{(m-1)(m-2)}. $$
			Thus, using the fact that $T_1$ and $T_2$ are independent,
			$$ E\frac{T_2^2}{T_1^2} = E\frac{1}{T_1^2} \cdot ET_2^2 = \frac{n(n+1)\lambda_x^2}{(m-1)(m-2)\lambda_y^2}. $$
			By using the result from a), we know $ET_2/T_1 = \big(n\lambda_x\big) / \big((m-1)\lambda_y\big)$, then the risk \eqref{eq:4.1} be
			\begin{align*}
				R\left( \frac{\lambda_y}{\lambda_x}, \delta \right) &= d^2E\frac{T_2^2}{T_1^2} - 2 \frac{\lambda_x}{\lambda_y} d E\frac{T_2}{T_1} +  \frac{\lambda_x^2}{\lambda_y^2} \\
				&= d^2\frac{n(n+1)\lambda_x^2}{(m-1)(m-2)\lambda_y^2} - 2\frac{\lambda_x}{\lambda_y}d\frac{n\lambda_x}{(m-1)\lambda_y} + \frac{\lambda_x^2}{\lambda_y^2} \\
				&= \frac{\lambda_x^2}{\lambda_y^2} \left( \frac{n(n+1)}{(m-1)(m-2)}d^2 - 2\frac{n}{(m-1)}d + 1 \right) \\
				&= \frac{\lambda_x^2}{\lambda_y^2} \frac{n(n+1)}{(m-1)(m-2)} \left[ \left( d- \frac{m-2}{n+1}\right)^2 - \left(\frac{m-2}{n+1}\right)^2 + 1 \right].
			\end{align*}
			Thus, it has the minimum when $d = \frac{m-1}{n+1}$. \\
			Therefore, the best estimator of $\lambda_x/\lambda_y$ is $c\overline Y/\overline X = dT_2/T_1 = \left[(m-2)n \overline Y\right] / \left[(n+1)m \overline X\right]$.
			$$ \therefore  c =  \frac{(m-2)n}{(n+1)m}. $$
            	\end{proof}
	
	\item[c)] Find the UMVU estimator of $e^{-\lambda_x} = P(X_1 > 1)$.
            	\begin{proof}[\underline{\textbf{Solution}}] $\newline$
            		Let $\delta$ is the unbiased estimator of $P(X_1 > 1)$. Then, 
			$$ E\delta = P(X_1 > 1) \Rightarrow \delta = 1_{\{X_1 > 1\}}.$$
			From the previous steps, we know $T_1$ is C.S.S., thus $\eta(T_1) =E(\delta | T_1) = P(X_1 > 1|T_1)$ is UMVU by Theorem 4.4. 
			We follow by 2 steps.
			\begin{enumerate}
			\item[(i)] \underline{Find the joint density of $X$ and $T_1$, where $X = 1_{\{X_1 > 1\}}$ and $T_1 = \sum_{i=1}^m X_i$.}	 \\
				First, we should find the joint density of $X$ and $S$ where $S = \sum_{i=2}^m X_i$.
				Since $X$ and $S$ are independent, the joint density is easily obtained as
				$$ f(x,s) = f(x)f(s) = \lambda_x e^{-\lambda_x x} \frac{\lambda_x^{m-1}}{\Gamma(m-1)} s^{m-2}e^{-\lambda_x s} = \frac{\lambda_x^m}{\Gamma(m-1)} s^{m-2}e^{-\lambda_x(x+2)}, ~~ x>0, s>0. $$
				By transformation of variables $X = X, T_1 = S+X$, the jacobian $|J| = \begin{vmatrix}
				1 & 0 \\
				-1 & 1
				\end{vmatrix}•= 1$.
				Then, the joint density of $X$ and $T_1$ is
				$$ f(x,t) = \frac{\lambda_x^m}{\Gamma(m-2)} (t-x)^{m-2}e^{-\lambda_x t}, ~~ 0 < x <t. $$
			\item[(ii)] \underline{Derive $E(\delta|T_1)$.} \\
				To obtain the expectation, we derive the conditional density of $X$ given $T_1$.
				Since $T_1 \sim \Gamma(m, \lambda_x)$, the conditional density be
				$$ f(x|t) = \frac{f(x,t)}{f(t)} = \frac{\Gamma(m)}{\Gamma(m-2)} (t-x)^{m-2}t^{-(m-1)} = (m-1)\frac{1}{t}\left(1-\frac{x}{t}\right)^{m-2}, ~~ 0 < x < t. $$
				Then, 
				\begin{align*}
					 E(\delta|T_1) &= P(X_1 > 1|T_1) = P(X|T_1) \\
					 		    &= \int_1^t f(x|t) dx \\
							    &= \int_1^t (m-1)\frac{1}{t}\left(1-\frac{x}{t}\right)^{m-2} dx \\
							    &= -\left(1-\frac{x}{t}\right)^{m-1} \bigg|_1^t \\
							    &= \left(1-\frac{1}{t}\right)^{m-1} 1_{\{T_1 > 1\}}.							    
				\end{align*}
			\end{enumerate}

            	\end{proof}
\end{enumerate}


%% # 4.2
\section{Problem 4.2}
Let $X_1,\dots,X_n$ be a random sample from $N(\mu_x, \sigma^2)$, and let $Y_1,\dots,Y_m$ be an independent random sample from $N(\mu_y, 2\sigma^2)$, with $\mu_x, ~ \mu_y$, and $\sigma^2$ all unknown parameters..
\begin{enumerate}
	\item[a)] Find a complete sufficient statisic.
            	\begin{proof}[\underline{\textbf{Solution}}] $\newline$
            		Since $X_1,\dots,X_n, Y_1,\dots,Y_m$ are mutually independent, the joint density,
			\begin{align*}
				f(x,y) &= f(x)f(y) \\
					&= (2\pi \sigma^2)^{-n/2}(4\pi\sigma^2)^{-m/2} \exp\left[ -\frac{1}{2\sigma^2}\sum_{i=1}^n(x_i - \mu_x)^2 - \frac{1}{4\sigma^2}\sum_{j=1}^m(y_j - \mu_y)^2 \right] \\
					&= \exp\left[ -\frac{\sum_{i=1}^nx_i^2}{2\sigma^2} - \frac{\sum_{j=1}^m y_j^2}{4\sigma^2} + \frac{\mu_x\sum_ix_i}{\sigma^2} + \frac{\mu_y \sum_jy_j}{2\sigma^2} - \frac{n\mu_x^2}{2\sigma^2} - \frac{m\mu_y^2}{4\sigma^2} \right]  (2\pi \sigma^2)^{-n/2}(4\pi\sigma^2)^{-m/2} \\
					&= \exp\left[ \frac{\mu_x\sum_ix_i}{\sigma^2} + \frac{\mu_y \sum_jy_j}{2\sigma^2} -\frac{2\sum_{i=1}^nx_i^2 + \sum_{j=1}^m y_j^2}{4\sigma^2} - \frac{n\mu_x^2}{2\sigma^2} - \frac{m\mu_y^2}{4\sigma^2} \right] (2\pi \sigma^2)^{-n/2}(4\pi\sigma^2)^{-m/2},
			\end{align*}
			is the full-rank exponential family. Therefore, 
			$$ T = \left(\sum_{i=1}^n X_i, \sum_{j=1}^m Y_j, 2\sum_{i=1}^nX_i^2 + \sum_{j=1}^m Y_j^2\right) $$
			is the complete sufficient statistic for $(\mu_x,\mu_y, \sigma^2)$.
            	\end{proof}
	
	\item[b)] Determine the UMVU estimator of $\sigma^2$.\\
	 	\underline{Hint}: Find a linear combination $L$ of $S_x^2 = \sum_{i=1}^n (X_i - \overline X)^2/(n-1)$ and $S_y^2 = \sum_{j=1}^m (Y_j-\overline Y)^2/(m-1)$ so that $(\overline X, \overline Y, L)$ is complete sufficient.
            	\begin{proof}[\underline{\textbf{Solution}}] $\newline$
			To use the C.S.S. obtained from a), let 
			$$ 2(n-1)S_x^2 + (m-1)S_y^2 = 2\left(\sum_iX_i^2 -n\overline X^2\right) + \left( \sum_jY_j^2 - m\overline Y^2 \right) $$
			is the linear combination of $T$. 
			Since the sample variance is unbiased, then its expectation is
			$$ E\left[2(n-1)S_x^2 + (m-1)S_y^2 \right] = (2n+2m-4)\sigma^2. $$
			Now, we let
			$$ S_p^2 = \frac{1}{2n+2m-4}\Big[ 2(n-1)S_x^2 + (m-1)S_y^2 \Big], $$
			then it is unbiased for $\sigma^2$ based on complete sufficient $T$. \\
			By theorem 4.4, it is UMVU of $\sigma^2$.
            	\end{proof}
	
	\item[c)] Find a UMVU estimator of $(\mu_x - \mu_y)^2$.
            	\begin{proof}[\underline{\textbf{Solution}}]
            		\begin{align*}
				E(\overline X - \overline Y)^2 &= \Var(\overline X - \overline Y) + \big[E(\overline X - \overline Y)\big]^2 \\
									     &= \Var(\overline X) + \Var(\overline Y) + \big[ E\overline X - E\overline Y \big]^2 \\
									     &= \frac{\sigma^2}{n} + \frac{2\sigma^2}{m} + (\mu_x - \mu_y)^2.
			\end{align*}
			We obtained the unbiased estimator of $\sigma^2$ in b). Thus, we let $\delta$ as
			$$ \delta = (\overline X - \overline Y)^2 - S_p^2\left( \frac{1}{n} - \frac{2}{m} \right), $$
			then it is the unbiased estimator of $(\mu_x - \mu_y)^2$ based on complete sufficient $T$. \\
			By Theorem 4.4, $\delta$ is UMVU of $(\mu_x - \mu_y)^2$.
            	\end{proof}
	
	\item[d)] Suppose we know the $\mu_y = 3\mu_x$. What is the UMVU estimator of $\mu_x$?
            	\begin{proof}[\underline{\textbf{Solution}}] $\newline$
            		With restriction $\mu_y = 3\mu_x$, the joint density of exponential family is changed as
			$$ f(x,y) = \cdots = \exp\left[ \frac{\mu_x(2\sum_ix_i + 3\sum_jy_j)}{2\sigma^2} - \cdots \right]\cdots. $$
			Thus, $T = \left(2\sum_iX_i + 3\sum_jY_j,~ 2\sum_{i=1}^nX_i^2 + \sum_{j=1}^m Y_j^2\right)$ is complete sufficient.\\
			Take expectation for $T_1$, then 
			$$ ET_1 = 2n\mu_x + 3m\mu_y = (2n+9m)\mu_x,$$
			Now we let $\delta$ as
			$$ \delta = \frac{2\sum_ix_i + 3\sum_jy_j}{2n+9m}, $$
			then,  it is unbiased estimator based on $T$. \\
			By Theorem 4.4, $\delta$ is UMVU of $\mu_x$.
            	\end{proof}
\end{enumerate}


%% # 4.3
\section{Problem 4.3}
Let $X_1,\dots,X_n$ be a random sample from the Poisson distribution with mean $\lambda$.
Find the UMVU for $\cos \lambda$.
(\underline{Hint}: For Taylor expansion, the identity $\cos \lambda = (e^{i\lambda} + e^{-i\lambda})/2$ may be useful.)

\begin{proof}[\underline{\textbf{Solution}}] $\newline$
	The joint density is
	$$ p_\lambda(x) = \frac{\lambda^{\sum_ix_i}e^{-n\lambda}}{x_1!\cdots x_n!}, $$
	with $T = \sum_iX_i$, that is complete sufficient. ($\because$ full-rank exponential family, and by factorization theorem)
	
	Note that $T \sim Poisson(n\lambda)$.
	Let $\delta(t)$ is an unbiased estimator of $\cos \lambda$ based on $T$.
	Then, its expectation should satisfy follows:
	$$ E\delta(T) = \sum_{t=0}^\infty \delta(t)\frac{(n\lambda)^t e^{-n\lambda}}{t!} = \cos\lambda $$
	$$ \Leftrightarrow \sum_{t=0}^\infty \delta(t)\frac{(n\lambda)^t }{t!} = e^{n\lambda}\cos\lambda $$
	Apply Talyor expansion for RHS,
	\begin{align*}
		e^{n\lambda}\cos\lambda &= e^{n\lambda}\frac{1}{2}\left( e^{i\lambda} + e^{-i\lambda} \right) \\
						        &= \frac{1}{2}\left( e^{(n+i)\lambda} + e^{(n-i)\lambda} \right) \\
						        &= \frac{1}{2}\sum_{t=0}^\infty \frac{\big((n+i)\lambda\big)^t + \big((n-i)\lambda\big)^t}{t!} \\
						        &= \sum_{t=0}^\infty \frac{(n\lambda)^t}{t!} \left[ \frac{\left(1+\frac{i}{n}\right)^t + \left(1-\frac{i}{n}\right)^t}{2} \right].
	\end{align*}
	Therefore, 
	$$ \delta(t) = \frac{1}{2}\left\{ \left(1+\frac{i}{n}\right)^t + \left(1-\frac{i}{n}\right)^t \right\} $$	
	is unbiased estimator of $\cos\lambda$ based on $T$. \\
	By Theorem 4.4, $\delta(t)$ is UMVU of $\cos\lambda$.
\end{proof}


%% # 4.4
\section{Problem 4.4}
Let $X_1,\dots,X_n$ be independent normal variables, each with unit variance, and with $EX_i=\alpha t_i + \beta t_i^2, ~ i=1,\dots, n$, where $\alpha$ and $\beta$ are unknown parameters and $t_1,\dots,t_n$ are known constants.
Find UMVU estimators of $\alpha$ and $\beta$.

\begin{proof}[\underline{\textbf{Solution}}] $\newline$
	Since $X_i \sim N(\alpha t_i + \beta t_i^2, 1)$, the joint density is
	\begin{align*}
	f(x) &= (2\pi)^{-\frac{n}{2}} \exp\left[ -\frac{1}{2}\sum_i (x_i - \alpha t_i - \beta t_i^2)^2 \right] \\
	      &= (2\pi)^{-\frac{n}{2}} \exp\left[ -\frac{1}{2}\sum_i \left( x_i^2 + \alpha^2t_i^2 + \beta^2t_i^4-2\alpha x_it_i - 2\beta x_it_i^2 + \alpha\beta t_i^3 \right) \right].
	\end{align*}
	Since these density is full-rank exponential family, 
	$$ T = \left( \sum_i X_it_i, ~ \sum_i X_it_i^2 \right) $$
	 is complete sufficient and its expected value is
	$$ ET_1 = \alpha \sum t_i^2 + \beta \sum t_i^3, ~~~ ET_2 = \alpha \sum t_i^3 + \beta \sum t_i^4. $$
	By using this,
	$$ \widehat\alpha = \frac{T_1\sum t_i^4 - T_2 \sum t_i^3}{\sum t_i^2\sum t_i^4 - \left(\sum t_i^3\right)^2}, ~~~ \widehat\beta = \frac{T_1\sum t_i^3 - T_2 \sum t_i^2}{\left(\sum t_i^3\right)^2 - \sum t_i^4\sum t_i^2} $$
	are unbiased estimator of $\alpha$ and $\beta$.
	Since these are the function of $T$, also UMVU.
\end{proof}


%% # 4.5
\section{Problem 4.5}
Let $X_1,\dots,X_n$ be i.i.d. from some distribution $Q_\theta$, and let $\overline X=(X_1 + \cdots + X_n) / n$ be the sample average.
\begin{enumerate}
	\item[a)] Show that $S^2 = \sum(X_i-\overline X)^2 / (n-1)$ is unbiased for $\sigma^2 = \sigma^2(\theta) = \Var_\theta(X_i)$.
	       	\begin{proof}[\underline{\textbf{Solution}}] $\newline$
            		Denote $\mu = EX_i$. Also note that $(n-1)S^2 = \sum_iX_i^2 - n\overline X^2$. Then,
			\begin{align*}
				E\left((n-1)S^2\right) &= \sum_i EX_i^2 - nE\overline X^2 \\
								&= \sum_i\left[\Var X_i + (EX_i)^2\right] - n\left[\Var\overline X + (E\overline X)^2\right] \\
								&= n(\sigma^2 + \mu^2) - n(\sigma^2/n + \mu^2) \\
								&= (n-1)\sigma^2.
			\end{align*}
			Therefore $S^2$ is unbiased for $\sigma^2$.
            	\end{proof}
	
	\item[b)] If $Q_\theta$ is the Bernoulli distribution with success probability $\theta$, show that $S^2$ from (a) is UMVU.
            	\begin{proof}[\underline{\textbf{Solution}}] $\newline$
			Note that the density of $X_i$ is $\theta^{x_i}(1-\theta)^{1-x_i} = \left(\frac{\theta}{1-\theta}\right)^x(1-\theta)$. \\
            		The joint density,
			$$ p(x) = \exp\left[ \sum_ix_i\log\frac{\theta}{1-\theta} + n\log(1-\theta) \right], $$
			is the full-rank exponential family.
			Thus, $T=\overline X$ (or $\sum_i X_i$) is complete sufficient. \\
			Meanwhile, 
			\begin{align*}
				S^2 &= \frac{1}{n-1}\left(\sum X_i^2 - n\overline X^2\right) \\
				       &= \frac{1}{n-1}\left(\sum X_i - n\overline X^2\right) ~~~(\because X_i^2 = X_i) \\
				       &= \frac{n\overline X(1-\overline X)}{n-1}
			\end{align*}
			is the function of $T=\overline X$.
			Therefore, $S^2$ is UMVU of $\sigma^2$ by Theorem 4.4.
            	\end{proof}
	
	\item[c)] If $Q_\theta$ is the exponential distribution with failure rate $\theta$, find the UMVU estimator of $\sigma^2 = 1/\theta^2$.
		Give a formula for $E_\theta[X_i^2|\overline X=c]$ in this case.
            	\begin{proof}[\underline{\textbf{Solution}}] $\newline$
            		Note that the density of $X_i$ is $\theta e^{-\theta x}$ with mean $1/\theta$ and variance $1/\theta^2$. \\
			The joint density,
			$$ p(x) = \exp\left[ -\theta\sum x_i + n\log\theta \right], $$
			is the full-rank exponential family.
			Thus, $T=\overline X$ (or $\sum_i X_i$) is complete sufficient with $ET = 1/\theta$ and $\Var T = 1/(n\theta^2)$. \\
			Meanwhile,
			$$ E\overline X^2 = \Var\overline X + (E\overline X)^2 = \frac{1}{n\theta^2} + \frac{1}{\theta^2} = \frac{n+1}{n}\frac{1}{\theta^2} = \frac{n+1}{n}\sigma^2.$$
			Thus, $\delta = \frac{n}{n+1}\overline X^2$ is the unbiased estimator of $\sigma^2$, and also function of $T=\overline X$. \\
			By Theorem 4.4, $\delta$ is UMVU for $\sigma^2$.
			
			Next, since $X_i$'s are i.i.d., $E_\theta[X_1^2|\overline X=c] = \cdots = E_\theta[X_n^2|\overline X=c]$. Then,
			\begin{align*}
				E_\theta[S^2|\overline X=c] &= \frac{nc^2}{n+1} \\
									   &= E_\theta\left[\frac{\sum_i X_i^2-n\overline X^2}{n-1}  \bigg|\overline X=c\right] \\
									   &= E_\theta\left[\frac{\sum_i X_i^2}{n-1} - \frac{nc^2}{n-1} \bigg|\overline X=c\right] \\
									   &= \frac{n}{n-1}E_\theta[X_1^2|\overline X=c] - \frac{nc^2}{n-1}.
			\end{align*}
			$$ \therefore E_\theta[X_i^2|\overline X=c] = nc^2\left( \frac{1}{n+1} + \frac{1}{n-1} \right)\frac{n-1}{n} = c^2\left( \frac{n-1}{n+1} + 1 \right) = \frac{2nc^2}{n+1}. $$
			
            	\end{proof}
\end{enumerate}


%% # 4.6
\section{Problem 4.6}
Suppose $\delta$ is a UMVU estimator of $g(\theta)$; $U$ is an unbiased estimator of zero, $E_\theta U=0, ~\theta \in \Omega$; and that $\delta$ and $U$ both have finite variances for all $\theta \in \Omega$.
Show that $U$ and $\delta$ are uncorrelated, $E_\theta U\delta = 0, ~ \theta \in \Omega$.

\begin{proof}[\underline{\textbf{Solution}}] $\newline$
For the convenience, we drop $\theta$ in the notation of $E$, $\Var$ and $\Cov$.\\
Let $\delta^* = \delta + cU$ where $c$ is a constant. Then, $\delta^*$ is an unbiased estimator for $g(\theta)$ ($\because ~ E\delta^* = E(\delta + cU) = E\delta + cEU = g(\theta)$). \\
Since $\delta$ is UMVU, 
$$ \Var(\delta + cU) = \Var(\delta) + c^2\Var(U) + 2c\Cov(\delta, U) \ge \Var(\delta), $$
$$ \Rightarrow c^2\Var(U) + 2c\Cov(\delta, U) \ge 0. $$
Denote LHS as $h(c)$. Since $h$ is convex and has minimum $h(0) = 0$, $h$ should have a local minimum at $c=0$.
Since $h$ is differentiable, we can obtain the following result by taking a first derivative at $c=0$:
$$ h'(0) = 2\Cov(\delta, U) = 0. $$
Since $EU=0$, we finally obtain  $E_\theta U\delta = 0$ for all $\theta \in \Omega$.
\end{proof}


%% # 4.7
\section{Problem 4.7}
Suppose $\delta_1$ is a UMVU estimator of $g_1(\theta)$, $\delta_2$ is UMVU estimator of $g_2(\theta)$, and that $\delta_1$ and $\delta_2$ both have finite variance for all $\theta$.
Show that $\delta_1 + \delta_2$ is UMVU for $g_1(\theta) + g_2(\theta)$.\\
\underline{Hint}: Use the result in the previous problem.

\begin{proof}[\underline{\textbf{Solution}}] $\newline$
We can easily show that $\delta_1 + \delta_2$ is unbiased for $g_1(\theta)+g_2(\theta)$. \\
Now, suppose $\delta$ is any unbiased estimator of $g_1(\theta)+g_2(\theta)$ with $\Var(\delta) < \infty$.
Let $U=\delta - \delta_1-\delta_2$, then $U$ is an unbiased estimator of 0 ($\because ~ EU = E\delta - E\delta_1-E\delta_2= 0$). \\
By Problem 4.6, 
$$ \Cov(U,\delta_1 + \delta_2) = \Cov(U,\delta_1) + \Cov(U,\delta_2) = 0.$$
Since $\delta = U+\delta_1+\delta_2$, we obtain the following result:
\begin{align*}
	\Var(\delta) &= \Var(U) + \Var(\delta_1+\delta_2) \\
			  &\ge \Var(\delta_1 + \delta_2).
\end{align*}
Because $\delta$ is arbitrary, $\delta_1 +\delta_2$ is UMVU.
\end{proof}


%% # 4.8
\section{Problem 4.8}
Let $X_1,\dots,X_n$ be i.i.d. absolutely continuous variables with common density $f_\theta, ~ \theta > 0$, given by
$$ f_\theta(x) = \begin{cases}
\theta/x^2, ~~~~~ x > \theta, \\
0,~~~~~~~~~~ x \le \theta.
\end{cases} $$
Find the UMVU estimator for $g(\theta)$ if $g(\theta)/\theta^n \to 0$ as $\theta \to \infty$ and $g$ is differentiable.

\begin{proof}[\underline{\textbf{Solution}}] $\newline$
The joint density is
$$ \theta^n/(x_1\cdots x_n)^2 1_{\{\min x_i\}}. $$
By factorization theorem, $T=\min X_i$ is sufficient. \\
To obtain the density of $T$, the cumulative distribution function is
$$ P(T \le t) = 1- P(\min X_i > t) = 1-\prod_i P(X_i > t) = 1-\left(\frac{\theta}{t}\right)^n, ~~~ t > \theta. $$
Take a derivative, we can obtain the density,
$$ f(t) = \frac{n\theta^n}{t^{n+1}}, ~~~ t >\theta. $$

Now, suppose $\delta(T)$ is the unbiased estimator of $g(\theta)$, then
\begin{align*}
	\int_\theta^\infty \delta(t)\frac{n\theta^n}{t^{n+1}}dt = g(\theta) &\Longleftrightarrow n\int_\theta^\infty \delta(t)t^{-n-1}dt = \frac{g(\theta)}{\theta^n} \\
			&~~ \text{Taking a derivative w.r.t. } \theta, \\
			&\Longleftrightarrow -n\delta(t)t^{-n-1} = g'(t)t^{-n}-ng(t)t^{-n-1}.
\end{align*}
$$ \therefore ~ \delta(t) = g(t) - \frac{g'(t)}{nt}. $$
From the above result, $g(t) = c$ for any constant $c$ implies $\delta(t) = c$.
Therefore $T$ is complete.\\
In summary, $T$ is complete sufficient, and $\delta(T)$ is the function of $T$.
Therefore, by Theorem 4.4, $\delta(T)$ is UMVU.
\end{proof}


%% # 4.10
\section{Problem 4.10}
Suppose $X$ is an exponential variable with density $p_\theta(x) = \theta e^{-\theta x}, ~ x > 0; ~ p_\theta (x) = 0,$ otherwise.
Find the UMVU estimator for $1/(1+\theta)$.

\begin{proof}[\underline{\textbf{Solution}}] $\newline$
Since $X$ is a full-rank exponential family, $T=X$ is complete sufficient.\\
Let $\delta(X)$ be an unbiased estimator of $1/(1+\theta)$, and represent as $\delta(x) = \sum_{n=0}^\infty c_nx^n$ by power series. Then,
\begin{align*}
	E\delta(X) &= \int_0^\infty \left( \sum_{n=0}^\infty c_nx^n \right) \theta e^{-\theta x}dx \\
			& \text{By Fubini theorem,} \\
			&= \sum_{n=0}^\infty \int_0^\infty c_n\theta x^n e^{-\theta x}dx \\
			& \text{The inner integration follows } \Gamma(n+1, 1/\theta) \text{ with rate parameter.} \\
			&= \sum_{n=0}^\infty \frac{c_n \Gamma(n+1)}{\theta^n} \\
			&= \sum_{n=0}^\infty \frac{c_n n!}{\theta^n}. \tag{$*$}
\end{align*}
Meanwhile, $1/(1+\theta)$ can be written as the geometric series,
\begin{equation*}
	\frac{1}{1+\theta} = \frac{1/\theta}{1+1/\theta} = -\sum_{n=1}^\infty \left(-\frac{1}{\theta}\right)^n = \sum_{n=1}^\infty \frac{(-1)^{n+1}}{\theta^n} ~~\text{ for } \theta >1. \tag{**} 
\end{equation*}
Since $\delta$ is unbiased, ($*$) and ($**$) should be identical. Thus,
$$ c_0 = 0, ~ c_n = \frac{(-1)^{n+1}}{n!}, ~n=1,2,\dots. $$
Therefore, 
$$ \delta(x) = \sum_{n=1}^\infty \frac{(-1)^{n+1}}{n!}x^n = -\sum_{n=1}^\infty \frac{(-x)^n}{n!} = 1-e^{-x} ~~\text{ for } \theta > 1.$$
Since $\delta$ is the function of the complete sufficient statistic $X$, $\delta(X)$ is UMVU by Theorem 4.4.

{\color{blue} If $\theta \le 1$, direct하게 unbiased임을 보일 수 있다?? 어떻게?? }
\end{proof}


%% # 4.11
\section{Problem 4.11}
Let $X_1,\dots,X_3$ be i.i.d. geometric variables with common mass function $f_\theta(x) = P_\theta(X_i=x)=\theta(1-\theta)^x, ~ x=0,1,\dots$.
Find the UMVU estimator of $\theta^2$.

\begin{proof}[\underline{\textbf{Solution}}] $\newline$
Since the joint mass, $\theta^3(1-\theta)^{x_1+x_2+x_3}$, is the full-rank exponential family, $T=X_1+X_2+X_3$ is complete sufficient. \\
Denote $g(\theta) = \theta^2$, and let $\delta(X) = 1_{\{X_1= 0, X_2=0\}}$, then $\delta(X)$ be an unbiased estimator of $g(\theta)$ ($\because ~ E\delta(X) = P(X_1=0,X_2=0) = \theta^2$). \\
Define $\eta(T) = E[\delta(X)|T] = P(X_1=0, X_2=0|T)$. Then it is UMVU by Theorem 4.4.
To obtain $\eta(t)$, we first find the density of $T$ by using the transformation $T=X_1+X_2+X_3$ from the joint density,
$$ P(T=t) = {t+2 \choose 2} \theta^3(1-\theta)^t = {t+2 \choose 2} \theta^2 \Big[\theta(1-\theta)^t\Big]. $$
Because we already have 2 successes in first $t+2$ trials of total $t+3$ trials, we should add the combination term. (See RHS!!) \\
Next, the joint mass of $X_1,X_2,T$ is
$$ P(X_1=0, X_2 =0, T=t) = P(X_1=0, X_2 =0, X_3 = t) = \theta^3(1-\theta)^t, $$
from the joint mass of $X_1,X_2,X_3$. \\
Therefore, 
$$ \eta(t) = \frac{P(X_1=0,X_2=0,T=t)}{P(T=t)} = 1 \bigg/ {t+2 \choose 2} = \frac{2}{(t+2)(t+1)}, $$
$$ \therefore ~ \eta(T) = \frac{2}{(T+2)(T+1)} $$
is the UMVU estimator of $\theta^2$.
\end{proof}


%% # 4.12
\section{Problem 4.12}
Let $X$ be a single observation, absolutely continuous with density
$$ p_\theta(x) = \begin{cases}
\frac{1}{2}(1+\theta x), ~~~ |x| < 1, \\
0, ~~~~~~~~~~~~~~ |x| \ge 1.
\end{cases}$$
Here $\theta \in [-1,1]$ is an unknown parameter.
\begin{enumerate}
	\item[a)] Find a constant $a$ so that $aX$ is unbiased for $\theta$.
		The expected value is
	       	\begin{proof}[\underline{\textbf{Solution}}] $\newline$
            	$$ EX = \int_{-1}^1 \frac{x}{2}(1+\theta x)dx = \frac{1}{4}x^2 + \frac{\theta}{6}x^3 \bigg|_{-1}^1 = \frac{\theta}{3}. $$
		Therefore, when $a=3$, $aX$ be unbiased.
            	\end{proof}
	
	\item[b)] Show that $b=E_\theta|X|$ is independent of $\theta$.
            	\begin{proof}[\underline{\textbf{Solution}}] $\newline$
		By using the procedure in a),
            	$$ E|X| = \int_{-1}^1 \frac{|x|}{2}(1+\theta x)dx = \frac{1}{2}. $$
		Thus, $b=E_\theta|X|$ does not depend on $\theta$.
            	\end{proof}
	
	\item[c)] Let $\theta_0$ be a fixed parameter value in $[-1,1]$.
		Determine the constant $c=c_{\theta_0}$ that minimizes the variance of the unbiased estimator $aX+c\big(|X|-b\big)$ when $\theta=\theta_0$.
		Is $aX$ uniformly minimum variance unbiased?
            	\begin{proof}[\underline{\textbf{Solution}}] $\newline$
            	To obtain the variance of $aX+c\big(|X|-b\big)$, we first derive the second moments.
		$$ EX^2 = E|X|^2 = \int_{-1}^1\frac{x^2}{2}(1+\theta x)dx = \frac{x^3}{6}+\frac{\theta x^4}{8}\bigg|_{-1}^1 = \frac{1}{3}, $$
		$$ EX|X| = \int_{-1}^1 \frac{x|x|}{2}(1+\theta x)dx = \frac{\theta}{4}. $$
		From the previous steps, we obtain $a=3$ and $b=E|X|$.
		Under $\theta=\theta_0$, the variance be
		\begin{align*}
			\Var_{\theta_0}\Big(3X+c\big(|X| - E|X|\big)\Big) &= 9\Var_{\theta_0}(X) + 6c\Cov_{\theta_0}\big(X,|X|-E|X|\big) + c^2\Var_{\theta_0}\big(|X| - E|X|\big) \\
										&= 9\Var_{\theta_0}(X) + 6c\Cov_{\theta_0}\big(X,|X|-E|X|\big) + c^2\Var_{\theta_0}\big(|X|\big) \\
										&= 9\left(\frac{1}{3}-\frac{\theta_0^2}{9}\right) + 6c\left(\frac{\theta_0}{4}-\frac{1}{2}\cdot\frac{\theta_0}{3}\right) + c^2\left(\frac{1}{3}-\frac{1}{4}\right)  \\
										&= 3-\theta_0^2 + \frac{\theta_0}{2}c +\frac{1}{12}c^2. \tag{$*$}
		\end{align*}
		Since ($*$) is convex with respect to $c$, take a derivative, 
		$$ \frac{\theta_0}{2} + \frac{c}{6} = 0 $$
		Therefore, ($*$) has minimum when $c=-3\theta_0$. \\
		Meanwhile, $\Var_{\theta_0}(3X) = 3-\theta_0^2$ which is greater than ($*$).
		Therefore, $3X$ is not UMVU.
            	\end{proof}
\end{enumerate}


%% # 4.24
\section{Problem 4.24}
In the normal one-sample problem, the statistic $t = \sqrt{n} \overline X/S$ has the non-central $t$-distribution on $n-1$ degrees of freedom and non-centrality parameter $\delta = \sqrt{n}\mu/\sigma$.
Use our results on distribution theory for the one-sample problem to find the mean and variance of $t$.

\begin{proof}[\underline{\textbf{Solution}}] $\newline$
	Note that $E\overline X = \mu = \sigma\delta/\sqrt{n}, ~ \Var(\overline X) = \sigma^2/n$, and the formula $ES^r$ in (4.10) page 70. \\
	Since $\overline X$ and $S^2$ are independent, 
	$$ Et = E\left[\sqrt{n}\frac{\overline X}{S}\right] = \sqrt{n}(E\overline X) (E S^{-1}) = \sqrt{n}\frac{\sigma\delta}{\sqrt{n}} \frac{\sigma^{-1}2^{-\frac{1}{2}}\Gamma\big[(-1+n-1)/2\big]}{(n-1)^{-\frac{1}{2}}\Gamma\big[(n-1)/2\big]} = \delta \frac{\sqrt{n-1}~\Gamma\big[(n-2)/2\big]}{\sqrt{2}~\Gamma\big[(n-1)/2\big]}, $$
	\begin{align*}
	Et^2 &= n(E\overline X^2)(ES^2) = n\left( \frac{\sigma^2}{n}+\frac{\sigma^2\delta^2}{n} \right) \frac{\sigma^{-2}2^{-1}~\Gamma\big[(-2+n-1)/2\big]}{(n-1)^{-1}~\Gamma\big[(n-1)/2\big]}  \\
		&= (1+\delta^2)\frac{n-1}{2} \frac{\Gamma\big[(n-3)/2\big]}{\Gamma\big[(n-3)/2+1\big]} = (1+\delta^2)\frac{n-1}{n-3}.
	\end{align*}
	$$ \Var(t) = Et^2-(Et)^2 = \frac{n-1}{n-3} + \delta^2\left[ \frac{n-1}{n-3} - \frac{(n-1)\left(\Gamma\big[(n-2)/2\big]\right)^2}{2\left(\Gamma\big[(n-1)/2\big]\right)^2} \right]. $$
\end{proof}


%% # 4.28
\section{Problem 4.28}
Let $X_1,\dots,X_n$ be i.i.d. from the uniform distribution on $(0,\theta)$.
\begin{enumerate}
	\item[a)] Use the Hammersley-Chapman-Robbins inequality to find a lower bound for the variance of an unbiased estimator of $\theta$.
		This bound will depend on $\Delta$. Note that $\Delta$ cannot vary freely but must lie in a suitable set.
	       	\begin{proof}[\underline{\textbf{Solution}}] $\newline$
            		Since $\delta$ is the unbiased estimator of $\theta$, $g(\theta) = \theta$ and $g(\theta+\Delta) - g(\theta) = \Delta$. 
		By Hammersley-Chapman-Robbins inequality, the lower bound for the variance of $\delta$ be
		$$ LB = \frac{\Delta^2}{E_\theta\left[\frac{p_{\theta+\Delta}(X)}{p_\theta(X)} - 1 \right]^2}. $$
		Since $X_i \sim Uniform(0,\theta)$, $p_\theta(X) = 0 \Rightarrow p_{\theta+\Delta}(X) = 0$.
		Thus $\Delta$ should be negative and $\theta+\Delta > 0$, therefore $\Delta \in (-\theta, 0)$.
		$$ \frac{p_{\theta+\Delta}(X)}{p_\theta(X)} = \begin{cases}
			\frac{\theta^n}{(\theta+\Delta)^n}, ~~~ \max\{X_1,\dots,X_n\} < \theta+\Delta \text{ with probability } \frac{(\theta+\Delta)^n}{\theta^n} \text{ under } P_\theta, \\
			0, ~~~~~~~~~~~\text{otherwise}.
		\end{cases} $$
		For simplicity, denote $Y = p_{\theta+\Delta}(X)/p_\theta(X)$, then $EY=1$ and $EY^2 = \frac{\theta^n}{(\theta+\Delta)^n}$.
		Thus, the expectation in the denominator is 
		$$ E(Y-1)^2 = Var(Y) = EY^2 - [EY]^2 = \frac{\theta^n}{(\theta+\Delta)^n} - 1. $$
		Therefore, the lower bound be
		\begin{equation*}
		 LB = \frac{\Delta^2}{\frac{\theta^n}{(\theta+\Delta)^n} - 1}. \tag{$*$}
		\end{equation*}
            	\end{proof}
	
	\item[b)] In principle, the best lower bound can be found taking the supremum over $\Delta$.
		This calculation cannot be done explicitly, but an approximation is possible.
		Suppose $\Delta = -c\theta/n$. Show that the lower bound for the variance can be written as $\theta^2g_n(c)/n^2$.
		Determine $g(c) = \limn g_n(c)$.
            	\begin{proof}[\underline{\textbf{Solution}}] $\newline$
            		Plug in $\Delta = -c\theta/n$ to ($*$), then
			$$ LB = \frac{c^2\theta^2 / n^2}{\left( 1-\frac{c}{n}\right)^{-n} - 1}. $$
			Therefore,
			$$ g_n(c) = \frac{c^2}{\left( 1-\frac{c}{n}\right)^{-n} - 1}, $$
			$$ g(c) = \limn g_n(c) = \frac{c^2}{e^c-1}. $$
            	\end{proof}
	
	\item[c)] Find the value $c_0$ that maximizes $g(c)$ over $c \in (0,1)$ and give an approximate lower bound for the variance of $\delta$. (The value $c_0$ cannot be found explicitly, but you should be able to come up with a numerical value.)
            	\begin{proof}[\underline{\textbf{Solution}}] $\newline$
            		Taking a derivative,
			$$ \frac{\partial g(c)}{\partial c} = \frac{\partial }{\partial c} \left( \frac{c^2}{e^c-1} \right) = \frac{2c(e^c-1)-c^2e^c}{(e^c-1)^2} = 0 $$
			\begin{align*}
            			&\Longleftrightarrow 2c(e^c-1) = c^2e^c \\
            			&\Longleftrightarrow 2e^c-2 = ce^c \\
            			&\Longleftrightarrow 1-e^c = \frac{c}{2} \\
            			&\Longleftrightarrow e^{-c} = 1-\frac{c}{2}.
			\end{align*}
			By numerical method, $c_0 = 1.59362$.
			Therefore, an approximated lower bound of $g(c_0)\theta^2/n^2 = 0.64761 \theta^2/n^2$.
            	\end{proof}
\end{enumerate}


%% # 4.30
\section{Problem 4.30}
Suppose $X_1,\dots,X_n$ are independent with $X_i \sim N(\alpha+\beta t_i, 1), ~ i=1,\dots, n$, where $t_1,\dots,t_n$ are known constants and $\alpha, \beta$ are unknown parameters.
\begin{enumerate}
	\item[a)] Find the Fisher information matrix $I(\alpha,\beta)$.
	       	\begin{proof}[\underline{\textbf{Solution}}] $\newline$
            		The log likelihood of $\theta = (\alpha, \beta)$ is
			$$ l(\beta) = \frac{n}{2}\log(2\pi) - \frac{1}{2}\sum_{i=1}^n (x_i-\alpha-\beta t_i)^2. $$
			Take derivatives,
			\begin{align*}
				&\frac{\partial l}{\partial \alpha} = \sum_i (x_i-\alpha-\beta t_i),  &&\frac{\partial l}{\partial \beta} = \sum_i (x_i-\alpha-\beta t_i)t_i, \\
				&\frac{\partial^2 l}{\partial \alpha^2} = -n,  &&\frac{\partial^2 l}{\partial \beta^2} = -\sum_i t_i^2, \\
				&\frac{\partial^2 l}{\partial \alpha \beta} = -\sum_i t_i 
			\end{align*}
			Therefore, the Fisher information matrix be
			$$ I(\alpha, \beta) = -E \left[\nabla l(\alpha,\beta)\right]^2 =  \begin{pmatrix}
			n & \sum_i t_i \\
			\sum_i t_i & \sum_i t_i^2
			\end{pmatrix}. $$
            	\end{proof}
	
	\item[b)] Give a lower bound for the variance of an unbiased estimator of $\alpha$.
            	\begin{proof}[\underline{\textbf{Solution}}] $\newline$
            		Since $g(\theta) = \alpha$, by Cramer-Rao lower bound,
			\begin{align*}
				LB &= \nabla g(\theta)^T I^{-1}(\alpha,\beta) \nabla g(\theta) \\
				     &= (1,0)\frac{\begin{pmatrix}
                    			\sum_i t_i^2 & -\sum_it_i \\
                    			-\sum_i t_i & n
                    			\end{pmatrix}}{n\sum_it_i^2 - (\sum_it_i)^2} {1 \choose 0} = \frac{1}{n\sum_it_i^2 - (\sum_it_i)^2}\left(\sum_it_i^2, ~ -\sum_i t_i\right) {1 \choose 0} \\
				     &= \frac{\sum_it_i^2}{n\sum_it_i^2 - (\sum_it_i)^2} \\
				     &= \frac{1/n}{1-(\sum_it_i)^2/\sum_i t_i^2}.
			\end{align*}
            	\end{proof}
	
	\item[c)] Suppose we know the value of $\beta$. Give a lower bound for the variance of an unbiased estimator of $\alpha$ in this case.
            	\begin{proof}[\underline{\textbf{Solution}}] $\newline$
            		Since $\beta$ is known, $\theta = \alpha$.
			Then the lower bound be
			$$ LB = \frac{g'(\theta)}{I(\theta)} = \frac{1}{n}. $$
            	\end{proof}
	
	\item[d)] Compare the estimators in parts (b) and (c). When are the bounds the same?
		If the bounds are different, which is larger?
            	\begin{proof}[\underline{\textbf{Solution}}] $\newline$
            		If $\sum_i t_i = 0$, two lower bounds be same. And by comparing the lower bounds,
			$$ LB_{b)} = \frac{1/n}{1-(\sum_it_i)^2/\sum_i t_i^2} >  \frac{1}{n} = LB_{c)}. $$
            	\end{proof}
	
	\item[e)] Give a lower bound for the variance of an unbiased estimator of the product $\alpha\beta$.
            	\begin{proof}[\underline{\textbf{Solution}}] $\newline$
            		Since $g(\theta) = \alpha \beta$, $\nabla g(\theta) = {\beta \choose \alpha}$.
			Then the lower bound be
			\begin{align*}
				LB &= \nabla g(\theta)^T I^{-1}(\alpha, \beta) \nabla g(\theta) \\
				      &= (\beta, ~ \alpha) \frac{\begin{pmatrix}
                    			\sum_i t_i^2 & -\sum_it_i \\
                    			-\sum_i t_i & n
                    			\end{pmatrix}}{n\sum_it_i^2 - (\sum_it_i)^2} {\beta \choose \alpha} \\
				      &= \frac{1}{n\sum_it_i^2 - (\sum_it_i)^2} \left( \beta \sum_i t_i^2-\alpha\sum_it_i, ~ -\beta\sum_it_i + n\alpha \right) {\beta \choose \alpha} \\
				      &= \frac{\beta^2\sum_it_i^2 - 2\alpha\beta\sum_it_i + n\alpha^2}{n\sum_it_i^2 - (\sum_it_i)^2}.
			\end{align*}
            	\end{proof}
\end{enumerate}


%% # 4.31
\section{Problem 4.31}
Find the Fisher information for the Cauchy location family with densities $p_\theta$ given by
$$ p_\theta(x) = \frac{1}{\pi\big[(x-\theta)^2+1\big]}. $$
Also, what is the Fisher information for $\theta^3$?

\begin{proof}[\underline{\textbf{Solution}}] $\newline$
	Since $p_\theta$ is the location family, by using Example 4.11 in page 75,
	\begin{align*}
		I(\theta) &= \int \frac{[f'(x)]^2}{f(x)}dx = \int \frac{\left[ -\frac{2\pi x}{\pi(x^2+1)} \right]^2}{1/[\pi(x^2+1)]} = \int \frac{4x^2}{\pi(x^2+1)^3}dx \\
			     & \text{Change of variable } x = \tan t \Rightarrow dx = \frac{\cos^2t + \sin^2t}{\cos^2t} = \frac{1}{\cos^2t}dt \\
			     &= \frac{4}{\pi}\int_{-\frac{\pi}{2}}^{\frac{\pi}{2}} \frac{\tan^2t}{(\tan^2+1)^3}\frac{1}{\cos^2t}dt \\
			     &= \frac{4}{\pi}\int_{-\frac{\pi}{2}}^{\frac{\pi}{2}} \frac{\sin^2t}{\cos^2t}\cos^6t\frac{1}{\cos^2t}dt ~~ (\because \tan^2t+1 = 1/\cos^2t) \\
			     &= \frac{4}{\pi}\int_{-\frac{\pi}{2}}^{\frac{\pi}{2}} \sin^2t\cos^2t dt \\
			     &= \frac{4}{\pi}\int_{-\frac{\pi}{2}}^{\frac{\pi}{2}} \frac{1-\cos 2x}{2}\frac{1+\cos 2x}{2} dt \\
			     &= \frac{4}{\pi}\int_{-\frac{\pi}{2}}^{\frac{\pi}{2}} \frac{1-\cos^2 2x}{4} dt \\
			     &= \frac{1}{\pi}\int_{-\frac{\pi}{2}}^{\frac{\pi}{2}} \frac{1-\cos 4x}{2} dt \\
			     &= \frac{1}{2\pi} \left[t - \frac{1}{4}\sin 4t\right]_{-\frac{\pi}{2}}^{\frac{\pi}{2}} \\
			     &= \frac{1}{2}.
	\end{align*}
	Let $\xi = \theta^3$, then $\theta = \xi^{1/3} := h(\xi)$.
	$$ \therefore I(\xi) = [h'(\xi)]^2 I\big( h(\xi) \big) = \left(\frac{1}{3}\xi^{-2/3}\right)^2 \frac{1}{2} = \frac{1}{18\xi^{4/3}}. $$
\end{proof}


%% # 4.32
\section{Problem 4.32}
Suppose $X$ has a Poisson distribution with mean $\theta^2$, so the parameter $\theta$ is the square root of the usual parameter $\lambda =EX$. Show that the Fisher information $I(\theta)$ is constant.

\begin{proof}[\underline{\textbf{Solution}}] $\newline$
	$p_\theta(X) = (\theta^2)^xe^{-\theta^2}/x!$ and $l(\theta) = \log 1/x! + 2x\log\theta-\theta^2$.
	$$ \frac{\partial l(\theta)}{\partial \theta} = \frac{2x}{\theta}-2\theta,~~~ \frac{\partial^2 l(\theta)}{\partial \theta^2} = -\frac{2x}{\theta^2}-2. $$
	$$ \therefore I(\theta) = E\left( -\frac{\partial^2 l(\theta)}{\partial \theta^2} \right) = \frac{2}{\theta^2}EX+2 = 4.$$
\end{proof}


%% # 4.33
\section{Problem 4.33}
Consider the exponential distribution with failure rate $\lambda$. Find a function $h$ defining a new parameter $\theta=h(\lambda)$ so that Fisher information $I(\theta)$ is constant.

\begin{proof}[\underline{\textbf{Solution}}] $\newline$
	Let the original parameter is $\theta$, and reparameterized parameter is $\lambda$. \\
	$p_\lambda(x) = \lambda e^{-\lambda x}$, $l(\lambda) = \log\lambda - \lambda x$.
	$$ \frac{\partial l(\lambda)}{\partial \lambda} = \frac{1}{\lambda}-x,~~~ \frac{\partial^2 l(\lambda)}{\partial \lambda^2} = -\frac{1}{\lambda^2}. $$
	$$ \tilde I(\lambda) = \frac{1}{\lambda^2}. $$
	From (4.18) in page 74,
	$$ I(\theta) = \frac{\tilde I(\lambda)}{[h'(\lambda)]^2} = \frac{1}{[\lambda h'(\lambda)]^2}. $$
	Thus, if $h(\lambda)=\log \lambda$, $I(\theta)$ be constant.
\end{proof}


%% # 4.34
\section{Problem 4.34}
Consider an autoregressive model in which $X_1 \sim N\big(\theta, \sigma^2/(1-\rho^2)\big)$ and the conditional distribution of $X_{j+1}$ given $X_1=x_1,\dots.X_j=x_j$, is $N\big(\theta+\rho(x_j-\theta), \sigma^2\big), ~ j=1,\dots,n-1$.
\begin{enumerate}
	\item[a)] Find the Fisher information matrix, $I(\theta,\sigma)$.
	       	\begin{proof}[\underline{\textbf{Solution}}] $\newline$
			The joint density is
            		$$ p(x_1,\dots,x_n) = p(x_1)p(x_2|x_1)\cdots(x_n|x_1,\dots,x_n), $$
			thus the likelihood function be
			\begin{align*}
				l(\theta,\sigma) &= l(x_1) + \sum_{j=1}^{n-1}l(x_{j+1}|x_1,\dots,x_j) \\
							&= \log\frac{1}{\sqrt{2\pi}\sqrt{1-\rho^2}}-\log\sigma - \frac{1-\rho^2}{2\sigma^2}(x_1-\theta)^2 \\
							&~~~~~ -\frac{1}{\sigma^2} \sum_{j=1}^{n-1} [x_{j+1}-\theta-\rho(x_j-\theta)]^2 + \log\left( \frac{1}{\sqrt{2\pi}\sigma} \right)^{n-1} \\
							&= - \frac{1-\rho^2}{2\sigma^2}(x_1-\theta)^2 -\frac{1}{2\sigma^2} \sum_{j=1}^{n-1} [x_{j+1} -\rho x_j -\theta(1-\rho)]^2 \\
							&~~~~~ -\frac{n}{2}\log(2\pi)-n\log\sigma+\log\sqrt{1-\rho^2}.
			\end{align*}
			Take derivatives,
			$$ \frac{\partial l(\theta,\sigma)}{\partial \theta} = \frac{1-\rho^2}{\sigma^2}(x_1-\theta) + \frac{1-\rho}{\sigma^2}\sum_{j=1}^{n-1} [x_{j+1}-\rho x_j - \theta(1-\rho)], $$
			$$ \frac{\partial^2 l(\theta,\sigma)}{\partial \theta^2} = -\frac{1-\rho^2}{\sigma^2} - \frac{(n-1)(1-\rho)^2}{\sigma^2}, $$
			$$ \frac{\partial^2 l(\theta,\sigma)}{\partial \theta \partial \sigma} = -\frac{2(1-\rho^2)\epsilon_1}{\sigma^3} + \frac{2(1-\rho)}{\sigma^3}\sum_{j=1}^{n-1}\eta_{j+1}, $$
			$$  \frac{\partial l(\theta,\sigma)}{\partial \sigma} = \frac{(1-\rho^2)(x_1-\theta)^2}{\sigma^3} + \frac{1}{\sigma^3}\sum_{j=1}^{n-1}\eta_{j+1}^2-\frac{n}{\sigma}, $$
			$$  \frac{\partial^2 l(\theta,\sigma)}{\partial \sigma^2} = -\frac{3(1-\rho^2)(x_1-\theta)^2}{\sigma^4} -\frac{3}{\sigma^4}\sum_{j=1}^{n-1}\eta_{j+1}^2 +\frac{n}{\sigma^2}, $$
			where $\epsilon_j = x_j-\theta$, and $\eta_{j+1} = \epsilon_{j+1}-\rho\epsilon_j$. (See \underline{Hint} in part c)) \\
			We can easily infer that $\epsilon_1 \sim N\big(0, \sigma^2/(1-\rho^2)\big)$ and $\eta_{j+1}|_{X_1=x_1,\dots,X_j=x_j} = \eta_{j+1}|_{\epsilon_1,\eta_2,\dots,\eta_j} \sim N(0,\sigma^2)$.
			{\color{blue} From this, $\eta_2,\dots,\eta_n \sim$ i.i.d. $N(0,\sigma^2)$ and these are independent of $\epsilon_1$. (왜???)} \\
			Therefore, the information matrix is given by taking expectation,
			$$ I(\theta,\sigma) = \begin{pmatrix}
			\frac{1-\rho^2+(n-1)(1-\rho)^2}{\sigma^2} & 0 \\
			0 & \frac{2n}{\sigma^2}
			\end{pmatrix}. $$
            	\end{proof}
	
	\item[b)] Give a lower bound for the variance of an unbiased estimator of $\theta$.
            	\begin{proof}[\underline{\textbf{Solution}}] $\newline$
			Since $g(\theta,\sigma) = \theta$,
            		$$ LB = \nabla g(\theta,\sigma)^T I^{-1}(\theta,\sigma) \nabla g(\theta,\sigma) = (1,0) I^{-1}(\theta,\sigma) {1 \choose 0} = \frac{\sigma^2}{1-\rho^2+(n-1)(1-\rho)^2}. $$
            	\end{proof}
	
	\item[c)] Show that the sample average $\overline X = (X_1 + \cdots + X_n)/n$ is an unbiased estimator of $\theta$, compute its variance, and compare its variance with the lower bound. \\
		\underline{Hint}: Define $\epsilon_j = X_j-\theta$ and $\eta_{j+1} = \epsilon_{j+1}-\epsilon_j$.
		Use smoothing to argue that $\eta_2,\dots,\eta_n$ are i.i.d. $N(0,\sigma^2)$ and are independent of $\epsilon_1$.
		Similarly, $X_i$ is independent of $\eta_{i+1},\eta_{i+2},\dots$.
		Use these facts to find first $ \Var (X_2) = \Var(\epsilon_2)$, then $\Var(X_3), \Var(X_4), \dots$.
		Finally find $\Cov(X_{i+1},X_i), n\Cov(X_{i+2},X_i)$, and so on.
            	\begin{proof}[\underline{\textbf{Solution}}] $\newline$
            		Let $\bar \epsilon = \frac{1}{n}\sum_j\epsilon_j = \overline X - \theta$.
			Since $E\epsilon_j = 0$, $E (\overline X - \theta) = E(\bar\epsilon) = 0$. 
			Thus $\overline X$ is unbiased estimator of $\theta$.
			
			Since $\epsilon_2 = \rho\epsilon_1 + \eta_2$, $\Var(\epsilon_2) = \rho^2\Var(\epsilon_1) + \Var(\eta_2) = \sigma^2 + \frac{\rho^2\sigma^2}{1-\rho^2} = \frac{\sigma^2}{1-\rho^2}$.
			In general, $\epsilon_i = \rho \epsilon_{i-1} + \eta_i$ and we can obtain $\Var(X_i) = \Var(\epsilon_i) = \frac{\sigma^2}{1-\rho^2}$. \\
			If $i > j$, 
			\begin{align*}
				\epsilon_i &= \rho\epsilon_{i-1} + \eta_i \\
						&= \rho^2(\epsilon_{i-2}+\eta_{i-1}) + \eta_i \\
						&= \cdots \\
						&= \rho^{i-j}\epsilon_j + \rho^{i-j+1}\eta_{j+1} + \cdots + \rho \eta_{i-1} + \eta_i.
			\end{align*}
			From this,
			$$ \Cov(X_i,X_j) = \Cov(\epsilon_i,\epsilon_j) = \rho^{|i-j|} \frac{\sigma^2}{1-\rho^2}. $$
			Then, the variance of $\overline X$ is
			\begin{align*}
				\Var(\overline X) &= \Var(\overline \epsilon) \\
							 &= \frac{1}{n^2}\Var\left(\sum_{i=1}^n \epsilon_i\right) =  \frac{1}{n^2}\Cov\left(\sum_{i=1}^n \epsilon_i, \sum_{i=1}^n \epsilon_i\right) \\
							 &= \frac{1}{n^2}\left[ \sum_{i=1}^n\Cov(\epsilon_i, \epsilon_i) + 2\sum_{i>j}\Cov(\epsilon_i,\epsilon_j) \right] \\
							 &= \frac{\sigma^2}{n^2(1-\rho^2)} \left[ n + 2\left\{ (n-1)\rho+(n-2)\rho^2 + \cdots + \rho^{n-1} \right\} \right] \\
							 &= \frac{\sigma^2}{n^2(1-\rho^2)} \left[ n + 2\sum_{j=1}^{n-1}(n-j)\rho^j \right] \\
							 &= \frac{\sigma^2}{n^2(1-\rho^2)} \left[ n + 2\left( n\frac{\rho-\rho^{n}}{1-\rho} -  \frac{\rho-\rho^{n+1}}{(1-\rho)^2} + \frac{n\rho^n}{1-\rho} \right) \right] \tag{$*$} \\
							 &=  \frac{\sigma^2}{n^2(1-\rho^2)} \left[ n + 2\frac{n\rho-n\rho^2-\rho+\rho^{n+1}}{(1-\rho)^2} \right] \\
							 &=  \frac{\sigma^2}{n^2(1-\rho^2)} \left[ \frac{n-2n\rho+n\rho^2+2n\rho-2n\rho^2-2\rho+2\rho^{n+1}}{(1-\rho)^2} \right] \\
							 &= \frac{\sigma^2}{n^2(1-\rho^2)} \left[ \frac{-n\rho^2-2\rho+n+2\rho^{n+1}}{(1-\rho)^2} \right] \\
							 &= \frac{\sigma^2}{n^2(1-\rho^2)} \left[ \frac{n(1-\rho^2)-2(\rho-\rho^{n+1})}{(1-\rho)^2} \right] \\
							 &= \frac{\sigma^2}{n(1-\rho)^2} - \frac{2\sigma^2\rho(1-\rho^n)}{n^2(1-\rho^2)(1-\rho)^2}.
			\end{align*}
			
			\underline{Note.} ($*$) is obtained as follows:
			\begin{align*}
				\sum_{j=1}^{n-1}\rho^j &= \frac{\rho(1-\rho^{n-1})}{1-\rho}, \\
				\sum_{j=1}^{n-1}j\rho^j &= \sum_j \rho \frac{d}{d\rho}\rho^j = \rho\frac{d}{d\rho}\sum_{j=1}^{n-1}\rho^j \\
								   &= \rho \frac{d}{d\rho} \left( \frac{\rho-\rho^n}{1-\rho} \right) \\
								   &= \rho \frac{(1-n\rho^{n-1})(1-\rho)+(\rho-\rho^n)}{(1-\rho)^2} \\
								   &= \rho \frac{1-\rho-n\rho^{n-1}+n\rho^n+\rho-\rho^n}{(1-\rho)^2} \\
								   &= \rho \frac{1-\rho^n - n\rho^{n-1}(1-\rho)}{(1-\rho)^2} \\
								   &= \frac{\rho-\rho^{n+1}}{(1-\rho)^2} - \frac{n\rho^n}{1-\rho}.
			\end{align*}
            	\end{proof}
\end{enumerate}

% Chapter 5. Curved Exponential Families
\chapter{Curved Exponential Families}

\section{Problem 5.1}
Suppose $X$ has a binomial distribution with $m$ trials and success probability $\theta$, $Y$ has a binomial distribution with $n$ trials and success probability $\theta^2$, and $X$ and $Y$ are independent.
\begin{enumerate}
	\item[a)] Find a minimal sufficient statistic $T$.
	       	\begin{proof}[\underline{\textbf{Solution}}] $\newline$
			Note that $f(x) = {m \choose x} \theta^x(1-\theta)^{m-x}$ and $f(y) = {n \choose y} (\theta^2)^y (1-\theta^2)^{n-y}$.
			Then the joint density is
			$$ f(x,y) = {m \choose x} {n \choose y} \exp\left[ x\log\frac{\theta}{1-\theta}+y\log\frac{\theta^2}{1-\theta^2} + m\log(1-\theta) + n\log(1-\theta^2) \right]. $$
			Since the canonical parameter $\eta = \left( \log\frac{\theta}{1-\theta}, \log\frac{\theta^2}{1-\theta^2} \right)$ does not satisfy a linear constraint, the joint density is the curved exponential family.
			Therefore, $T=(X,Y)$ is minimal sufficient.
		\end{proof}
	
	\item[b)] Show that $T$ is not complete, providing a non-trivial function $f$ with $E_\theta f(T)=0$.
            	\begin{proof}[\underline{\textbf{Solution}}] $\newline$
            		Note that $ET_1 = EX = m\theta, ~ ET_1^2 = EX^2 = m\theta(1-\theta)+m^2\theta^2$ and $ET_2 = EY = n\theta^2$.\\
			Now we let 
			$$ f(T) = T_1^2 - T_1 - \frac{m(m-1)}{n}T_2, $$
			then $E_\theta f(T) = 0$ for all $\theta \in (0,1)$.
			But $f(T)$ is not always 0. Thus $T$ is not complete.
            	\end{proof}
\end{enumerate}



\section{Problem 5.2}
Let $X$ and $Y$ be independent Bernoulli variables with $P(X=1)=p$ and $P(Y=1)=h(p)$ for some known function $h$.
\begin{enumerate}
	\item[a)] Show that the family of joint distributions is a curved exponential family unless
		$$ h(p) = \frac{1}{1+\exp\left\{a+b\log\frac{p}{1-p}\right\}} $$
		for some constants $a$ and $b$.
	       	\begin{proof}[\underline{\textbf{Solution}}] $\newline$
			The joint density is
			$$ f(x,y) = p^x(1-p)^{1-x}h(p)^y\big(1-h(p)\big)^{1-y} = \exp\left[ x\log\left(\frac{p}{1-p}\right) + y\log\left(\frac{h(p)}{1-h(p)}\right) \right] (1-p)\big(1-h(p)\big). $$
			If $h(p)$ defined as the above for some constants $a$ and $b$, then
			$$ \log\left(\frac{h(p)}{1-h(p)}\right) = -\left(a + b \log\left(\frac{p}{1-p}\right) \right) $$
			which is the linear constraint.
			Therefore, unless the above case, the family of joint densities is a curved exponential family.
		\end{proof}
	
	\item[b)] Give two functions $h$, one where $(X,Y)$ is minimal but not complete, and one where $(X,Y)$ is minimal and complete.
            	\begin{proof}[\underline{\textbf{Solution}}] $\newline$ \vspace{-0.5cm}
            		\begin{enumerate}
				\item[(i)] \underline{Minimal but not complete} \\
					From a), the joint density is a curved exponential family except on the above $h$ form. \\
					If we let a function $h(p) = \frac{p}{2}$, $E_p(X-2Y) = 0$ for all $p \in (0,1)$.
					But $X-2Y$ is not always 0. Therefore $T$ is not complete.
				\item[(ii)] \underline{Minimal and complete} \\
					For some function $g$,
					$$ E_pg(X,Y) = g(1,1)ph(p) + g(1,0)p\big(1-h(p)\big) + g(0,1)(1-p)h(p) + g(0,0)(1-p)\big(1-h(p)\big). $$
					To show $T$ is complete, we need to find a function $h$ that $ph(p), p\big(1-h(p)\big), (1-p)h(p)$ are linearly independent.\\
					If we let $h(p) = p^2$, then
					\begin{align*}
					E_pg(X,Y) &= p^3\left[ g(1,1)-g(1,0)-g(0,1)+g(0,0) \right] \\
							 &~~~+ p^2\left[ g(0,1)-g(0,0) \right] + p\left[g(1,0)-g(0,0)\right] + g(0,0).
					\end{align*}
					If this is 0 for all $p\in (0,1)$, it must be $g(0,0)=g(1,0)=g(0,1)=g(1,1) = 0$.
					Thus $g(X,Y) = 0$ and $T$ is complete.
			\end{enumerate}
            	\end{proof}
\end{enumerate}



\section{Problem 5.6}
Two teams A and B play a series of games, stopping as soon as one of the team has 4 wins.
Assume that game outcomes are independent and that on any given game team A has a fixed chance $\theta$ of winning.
Let $X$ and $Y$ denote the number of games won by the first and second team, respectively.
\begin{enumerate}
	\item[a)] Find the joint mass function for $X$ and $Y$. Show that as $\theta$ varies these mass functions form a curved exponential family.
	       	\begin{proof}[\underline{\textbf{Solution}}] $\newline$
			First, suppose team A has 4 wins which is the event $X=4$ and $Y=y$.
			Then, the first $(3+y)$ trials are random and last trial should be the winning of team A, thus the probability be
			$$ P(X=4, Y=y) = {3+y \choose 3}\theta^4(1-\theta)^y,~~~~ y=0,1,2,3. $$
			Similarly, 
			$$ P(X=x, Y=4) = {3+x \choose 3}\theta^x(1-\theta)^4,~~~~ x=0,1,2,3. $$
			Therefore, the joint mass have the form $h(x,y)\exp\big[ x\log\theta + y\log(1-\theta) \big]$.
			Since the canonical parameters $\log \theta$ and $\log(1-\theta)$ have non-linear relationship, the family of the joint mass is a curved exponential family.
		\end{proof}
	
	\item[b)] Show that $T=(X,Y)$ is complete.
            	\begin{proof}[\underline{\textbf{Solution}}] $\newline$
            		Let $g$ be an arbitrary function of $X$ and $Y$, and suppose
			$$ E_\theta g(X,Y) = \sum_{x=0}^3 g(x,4){3+x \choose 3}\theta^x(1-\theta)^4 + \sum_{y=0}^3 g(4,y) {3+y \choose 3}\theta^4(1-\theta)^y = 0, ~~~ \theta \in (0,1). $$
			Since it is the polynomial in $\theta$, if $\theta \to 0$, then the only constant term $g(0,4)$ is remain (other terms are tend to 0).
			Thus $g(0,4)$ should be 0. \\
			If $g(0,4)=0$, then the linear term $g(1,4)$ also should be 0 (dividing by $\theta$ and letting $\theta \to 0$).
			Similarly $g(2,4)=g(3,4)=0$. \\
			Then, now remain only
			$$ \frac{E_\theta g(X,Y)}{\theta^4} = \sum_{y=0}^3 g(4,y) {3+y \choose 3}(1-\theta)^y = 0, $$
			and this is the polynomial in $1-\theta$.
			Similarly, we can obtain $g(4,0)=\cdots=g(4,3) = 0$. Therefore, $g(X,Y) = 0$ almost surely. \\
			By definition of completeness, $T$ is complete.
            	\end{proof}
	
	\item[c)] Find a UMVU estimator of $\theta$.
            	\begin{proof}[\underline{\textbf{Solution}}] $\newline$
            		Let $\delta = 1_{\{\text{team A wins the first game}\}}$.
			Then $\delta$ is unbiased because $E\delta = P(\text{team A wins the first game}) = \theta$.
			Thus the UMVU estimator of $\theta$ be $E[\delta | T] = P[\delta=1|X,Y]$.
			By using the joint mass function,
			\begin{align*}
			E[\delta=1|X=4,Y=y] &= P(\delta=1| X=4,Y=y) \\
							&= \frac{P(\delta=1, X=4,Y=y)}{P(X=4,Y=y)} \\
							&= \frac{\theta^2 {2+y \choose 2}\theta^2(1-\theta)^y}{{3+y \choose 3}\theta^4(1-\theta)^y} \\
							&= \frac{3}{3+y},
			\end{align*}
			and
			$$ P(\delta=1| X=x,Y=4) = \frac{\theta(1-\theta) {2+x \choose x-1}\theta^{x-1}(1-\theta)^3}{{3+x \choose 3}\theta^x(1-\theta)^4} = \frac{x}{3+x}. $$
			By combining 2 results, the UMVU estimator of $\theta$ is 
			$$ \frac{X-1_{\{X=4\}}}{X+Y-1}. $$
            	\end{proof}
\end{enumerate}



\section{Problem 5.7}
Consider a sequential experiment in which observations are i.i.d. from a Poisson distribution with mean $\lambda$.
If the first observation $X$ is zero, the experiment stops, and if $X>0$, a second observation $Y$ is observed.
Let $T=0$ if $X=0$, and let $T=1+X+Y$ if $X>0$.
\begin{enumerate}
	\item[a)] Find the mass function for $T$.
	       	\begin{proof}[\underline{\textbf{Solution}}] $\newline$
			$P(T=0) = P(X=0) = e^{-\lambda}$. \\
			For $k=0,1,\dots,$,
			\begin{align*}
				P(T=k+1) &= P(X+Y=k, X>0) \\
					        &= P(X+Y=k) - P(X+Y=k, X=0) \\
					        &= \frac{(2\lambda)^ke^{-2\lambda}}{k!} - \frac{\lambda^k e^{-2\lambda}}{k!}.
			\end{align*}
			{\color{blue} 이게 왜 아니지??
			\begin{align*}
				P(T=k+1) &= P(X+Y=k, X>0) \\
					        &= P(X+Y=k) - P(X+Y=k, X=0) \\
					        &= P(X+Y=k) - P(Y=k) \\
					        &= \frac{(2\lambda)^ke^{-2\lambda}}{k!} - \frac{\lambda^k e^{-\lambda}}{k!}.
			\end{align*}
			}
		\end{proof}
	
	\item[b)] Show that $T$ is minimal sufficient.
            	\begin{proof}[\underline{\textbf{Solution}}] $\newline$
            	{\color{blue} Part c) first!!} \\
		From c), $(W,N)$ is minimal sufficient. Therefore $T$ that is one-to-one function of $(W,N)$ is also minimal sufficient.
            	\end{proof}
	
	\item[c)] Does this experiment give a curved two-parameter exponential family or full rank one-parameter exponential family?
            	\begin{proof}[\underline{\textbf{Solution}}] $\newline$
            		Let $N$ is the sample size(or \# of experiments), and $W$ as follows:
			$$ N=\begin{cases}
			1, ~~~\text{if } X=0, \\
			2, ~~~\text{if } X>0,
			\end{cases} ~~~~ W=\begin{cases}
			0, ~~~~~~~~~~\text{if } X=0, \\
			X+Y, ~~~\text{if } X>0.
			\end{cases} $$
			Note that $X$ and $Y$ are from i.i.d. Poisson($\lambda$) and it satisfies the condition in Theorem 5.4. \\
			By Theorem 5.4, the joint density is an exponential family with canonical parameter $\eta = (\log\lambda, -\lambda)$ and sufficient statistic $(X+Y, N) = (W,N)$. \\
			Since $\eta$ does not satisfy a linear constraint, the density of $W$ and $N$ is a curved exponential family.
			Thus, $(W,N)$ is minimal sufficient.
            	\end{proof}

	\item[d)] Is $T$ a complete sufficient statistic? \\
		\underline{Hint}: Write $e^\lambda E_\lambda g(T)$ as a power series in $\lambda$ and derive equations for $g$ setting coefficients for $\lambda^x$ to zero.
            	\begin{proof}[\underline{\textbf{Solution}}] $\newline$
            		Suppose $E_\lambda g(T) = 0$ for all $\lambda > 0$ and any function $g$. Then
			$$ E_\lambda g(T) = g(0)e^{-\lambda} + \sum_{k=0}^\infty g(k+1)\frac{(2\lambda)^k e^{-2\lambda}-\lambda^k e^{-2\lambda}}{k!} = 0. $$
			Multiplying $e^{2\lambda}$,
			$$ e^{2\lambda}E_\lambda g(T) = g(0)e^{\lambda} + \sum_{k=0}^\infty \frac{g(k+1)\left(2^k-1\right)}{k!}\lambda^k = 0. $$
			This is the power series in $\lambda$, so the coefficients of the power should be 0. ($g(0) = 0$ and $g(k+1) = 0$ for $k=1,2,\dots$.) \\
			Note that $T \ne 1$($\because$ If $X > 0$, then $Y\ge 0$ and $T > 1$).
			Therefore, $g(T) = 0$ thus $T$ is complete.
            	\end{proof}
\end{enumerate}



\section{Problem 5.13}
Consider a single two-way contingency table and define $R=N_{11}+N_{12}$ (the first row sum), $C=N_{11}+N_{21}$ (the first column sum), and $D=N_{11}+N_{22}$ (the sum of the diagonal entries).
\begin{enumerate}
	\item[a)] Show that the joint mass function can be written as a full rank three-parameter exponential family with $T=(R,C,D)$ as the canonical sufficient statistic.
	       	\begin{proof}[\underline{\textbf{Solution}}] $\newline$
			The contingency table be as follows:
			\begin{table}[h]
                            \centering
                            \begin{tabular}{|c|c|c|}
                            \hline
                            $N_{11}$ & $N_{12}$  & $R$ \\ \hline
                            $N_{21}$ & $N_{22}$ &   \\ \hline
                            $C$ &  & $n$ \\ \hline
                            \end{tabular}
			\end{table}
			
			By representing the elements with $R,C,D, $ and $n$,
			$$ N_{11} = (R+C+D-n)/2,~~~ N_{12} = (n-D-C+R)/2, $$
			$$ N_{21} = (n-D-R+C)/2,~~~ N_{22} = (n+D-R-C)/2. $$
			Thus the joint mass function is
			\begin{align*}
			{n \choose n_{11},\dots,n_{22}} p_{11}^{n_{11}} \cdots p_{22}^{n_{22}} &= {n \choose n_{11},\dots,n_{22}} \left(\sqrt{p_{11}}\right)^{r+c+d-n} \left(\sqrt{p_{12}}\right)^{n-d-c+r} \left(\sqrt{p_{21}}\right)^{n-d-r+c} \left(\sqrt{p_{22}}\right)^{n+d-r-c} \\
				&= {n \choose n_{11},\dots,n_{22}} \sqrt{\frac{p_{11}p_{12}}{p_{21}p_{22}}}^r \sqrt{\frac{p_{11}p_{21}}{p_{12}p_{22}}}^c \sqrt{\frac{p_{11}p_{22}}{p_{12}p_{21}}}^d \sqrt{\frac{p_{12}p_{21}p_{22}}{p_{11}}}^n \\
				&= {n \choose n_{11},\dots,n_{22}} \exp\left[ r\log\sqrt{\frac{p_{11}p_{12}}{p_{21}p_{22}}} + c\log\sqrt{\frac{p_{11}p_{21}}{p_{12}p_{22}}} \right. \\
				&~~~~~~~~~~~~~~~~~~~~~~~~~~~~~~~~~\left.+ d\log\sqrt{\frac{p_{11}p_{22}}{p_{12}p_{21}}} + n\log\sqrt{\frac{p_{12}p_{21}p_{22}}{p_{11}}} \right],
			\end{align*}
			where $r, c$, and $d$ are defined as before.\\
			The density is a full-rank(3-parameter) exponential family, and the sufficient statistic $(R,C,D)$ do not satisfy a linear constraint because there is a one-to-one linear association between it and $(N_{11},N_{12},N_{21})$, and the 3 canonical parameters $\eta_r, \eta_c$, and $\eta_d$ can vary freely over $\bbR^3$.
			{\color{blue} 마지막 부분 잘 이해 안됨,,,}
		\end{proof}
	
	\item[b)] Relate the canonical parameter associated with $D$ to the "cross-product ratio" $\alpha$ defined as $\alpha = p_{11}p_{22}/(p_{12}p_{21})$.	
            	\begin{proof}[\underline{\textbf{Solution}}] $\newline$
            		$\eta_d = \log\sqrt\alpha$
            	\end{proof}
	
	\item[c)] Suppose we observe $m$ independent two-by-two contingency tables.
		Let $n_i,~i=1,\dots,m$, denote the trials for table $i$.
		Assume that cell probabilities for the tables may differ, but that the cross-product ratios for all $m$ tables are all the same.
		Show that the joint mass functions form a full rank exponential family.
		Express the sufficient statistic as a function of the variables $R_1,\dots,R_m,C_1,\dots,C_m$, and $D_1,\dots,D_m$.
            	\begin{proof}[\underline{\textbf{Solution}}] $\newline$
            		The joint mass can be obtained as
			$$ \exp\left[ \sum_{i=1}^m r_i\eta_{r,i} + \sum_{i=1}^m c_i \eta_{c,i} + \eta_d\sum_{i=1}^md_i - \sum_{i=1}^m A_i(\eta_i) \right]h(x), $$
			which is a full-rank($2m+1$ parameter) exponential family.
			Therefore, the complete sufficient statistic is
			$$ T= \left( R_1,\dots,R_m, C_1,\dots,C_m, \sum_{i=1}^mD_i \right), $$
			which is a function of $R_1,\dots,R_m,C_1,\dots,C_m$.
            	\end{proof}
\end{enumerate}



\section{Problem 5.16}
For an $I \times J$ contingency table with independence, the UMVU estimator of $p_{ij}$ is $\hat p_{i+} \hat p_{+j} = N_{i+}N_{+j}/n^2$.
\begin{enumerate}
	\item[a)] Determine the variance of this estimator, $\Var(\hat p_{i+} \hat p_{+j})$.
	       	\begin{proof}[\underline{\textbf{Solution}}] $\newline$
			Since $N_{i+}$ and $N_{+j}$ are independent with
			$$ N_{i+} \sim \text{Binomial}(n, p_{i+}), ~~~~ N_{+j} \sim \text{Binomial}(n, p_{+j}). $$
			$E\hat p_{i+} = \frac{1}{n}EN_{i+} = p_{i+}$ and similarly, $E\hat p_{+j} = p_{+j}$.
			\begin{align*}
				E\hat p_{i+}^2\hat p_{+j}^2 &= \frac{1}{n^4}EN_{i+}^2N_{+j}^2 \\
									 &= \frac{1}{n^4}\left[np_{i+}(1-p_{i+})+n^2p_{i+}^2\right] \left[np_{+j}(1-p_{+j})+n^2p_{+j}^2\right] \\
									 &= \frac{1}{n^2}\left[p_{i+}(1-p_{i+})+np_{i+}^2\right] \left[p_{+j}(1-p_{+j})+np_{+j}^2\right].
			\end{align*}
			\begin{align*}
				\therefore \Var(\hat p_{i+} \hat p_{+j}) &= E\hat p_{i+}^2\hat p_{+j}^2 - \left[ E\hat p_{i+}\hat p_{+j} \right]^2 \\
									 &= \frac{1}{n^2} \left[p_{i+}(1-p_{i+})+np_{i+}^2\right] \left[p_{+j}(1-p_{+j})+np_{+j}^2\right] - p_{i+}^2p_{+j}^2 \\
									 &= \frac{1}{n^2}\left[ p_{i+}(1-p_{i+})p_{+j}(1-p_{+j}) + np_{i+}(1-p_{i+})p_{+j}^2 + np_{i+}^2p_{+j}(1-p_{+j}) \right] \\
									 &= \frac{p_{i+}(1-p_{i+})p_{+j}^2 + p_{+j}(1-p_{+j})p_{i+}^2}{n} + \frac{p_{i+}(1-p_{i+})p_{+j}(1-p_{+j})}{n^2}.
			\end{align*}
		\end{proof}
	
	\item[b)] Find the UMVU estimator of the variance in (a).
            	\begin{proof}[\underline{\textbf{Solution}}] $\newline$
			First, we should find the unbiased estimators in the numerator term.
			\begin{align*}
			EN_{i+}^2 &= np_{i+}(1-p_{i+}) + n^2p_{i+}^2 \\
					&= np_{i+}-np_{i+}^2+n^2p_{i+}^2 \\
					&= EN_{i+} + (n^2-n)p_{i+}^2
			\end{align*}
			$$ \Rightarrow E\left( \frac{N_{i+}(N_{i+}-1)}{n^2-n} \right) = p_{i+}^2, $$			
			\begin{align*}
				p_{i+}(1-p_{i+}) &= p_{i+} - p_{i+}^2 \\
							&= E\frac{N_{i+}}{n} - E\left( \frac{N_{i+}(N_{i+}-1)}{n^2-n} \right) \\
							&= E\left( \frac{N_{i+}(n-N_{i+})}{n^2-n} \right),
			\end{align*}
			From the above, we obtain the unbiased estimators as follows:
			\begin{align*}
				E\left( \frac{N_{i+}(N_{i+}-1)}{n^2-n} \right) &= p_{i+}^2, \\
				E\left( \frac{N_{i+}(n-N_{i+})}{n^2-n} \right) &= p_{i+}(1-p_{i+}), \\
				E\left( \frac{N_{+j}(N_{+j}-1)}{n^2-n} \right) &= p_{+j}^2, \\
				E\left( \frac{N_{+j}(n-N_{+j})}{n^2-n} \right) &= p_{+j}(1-p_{+j}).
			\end{align*}
			By using these, the expectation of $\Var(\hat p_{i+} \hat p_{+j})$ is
			\begin{align*}
				E\left[ \Var(\hat p_{i+} \hat p_{+j}) \right] &= E\left[ \frac{p_{i+}(1-p_{i+})p_{+j}^2 + p_{+j}(1-p_{+j})p_{i+}^2}{n} + \frac{p_{i+}(1-p_{i+})p_{+j}(1-p_{+j})}{n^2} \right] \\
					&= \frac{N_{i+}(n-N_{i+}N_{+j}(N_{+j}-1)+N_{+j}(n-N_{+j})N_{i+}(N_{i+}-1)}{n(n^2-n)^2} \\
					&~~~~~~~~~~~~ + \frac{N_{i+}(n-N_{i+})N_{+j}(n-N_{+j})}{n^2(n^2-n)^2}. \tag{$*$}
			\end{align*}			
			Therefore, ($*$) is UMVU estimator of $\Var(\hat p_{i+} \hat p_{+j})$.
            	\end{proof}
\end{enumerate}


% Chapter 6. Conditional Distributions
\chapter{Conditional Distributions}

\section{Problem 6.2}
Let $X$ and $Y$ be independent random variables with cumulative distribution functions $F_X$ and $F_Y$.
\begin{enumerate}
	\item[a)] Assuming $Y$ is continuous, use smoothing to derive a formula expressing the cumulative distribution function of $X^2Y^2$ as the expected value of a suitable function of $X$.
		Also, if $Y$ is absolutely continuous, give a formula for the density.
	       	\begin{proof}[\underline{\textbf{Solution}}] $\newline$
		
		\end{proof}
	
	\item[b)] Suppose $X$ and $Y$ are both exponential with the same failure rate $\lambda$.
		Find the density of $X-Y$.
            	\begin{proof}[\underline{\textbf{Solution}}] $\newline$
            
            	\end{proof}
\end{enumerate}



\section{Problem 6.3}
Suppose that $X$ and $Y$ are independent and positive. Use a smoothing argument to show that if $x\in(0,1)$, then
\begin{equation*}
	P\left(\frac{X}{X+Y} \le x\right) = EF_X\left(\frac{xY}{1-x}\right), \tag{6.8} 
\end{equation*}
where $F_X$ is the cumulative distribution function of $X$.
\begin{proof}[\underline{\textbf{Solution}}] $\newline$
		
\end{proof}



\section{Problem 6.4}
Differentiating (6.8), if $X$ is absolutely continuous with density $p_X$, then $V=X/(X+Y)$ is absolutely continuous with density
$$ p_V(x) = E\left[\frac{Y}{(1-x)^2}p_X\left(\frac{xY}{1-x}\right)\right], ~~~~~ x\in(0,1). $$
Use this formula to derive the beta distribution introduced in Problem 6.1, showing that if $X$ and $Y$ are independent with $X\sim \Gamma(\alpha,1)$ and $Y \sim \Gamma(\beta,1)$, then $V=X/(X+Y)$ has density
$$ p_V(x) = \frac{\Gamma(\alpha+\beta)}{\Gamma(\alpha)\Gamma(\beta)}x^{\alpha-1}(1-x)^{\beta-1} $$
for $x\in(0,1)$.
\begin{proof}[\underline{\textbf{Solution}}] $\newline$
		
\end{proof}



\section{Problem 6.5}
Let $X$ and $Y$ be absolutely continuous with joint density
$$ p(x,y) = \begin{cases}
2,~~~0<x<y<1, \\
0,~~~\text{otherwise.}
\end{cases} $$
\begin{enumerate}
	\item[a)] Find the marginal density of $X$ and the marginal density of $Y$.
	       	\begin{proof}[\underline{\textbf{Solution}}] $\newline$
		
		\end{proof}
	
	\item[b)] Find the conditional density of $Y$ given $X=x$.
            	\begin{proof}[\underline{\textbf{Solution}}] $\newline$
            
            	\end{proof}

	\item[c)] Find $E[Y|X]$.
            	\begin{proof}[\underline{\textbf{Solution}}] $\newline$
            
            	\end{proof}

	\item[d)] Find $EXY$ by integration against the joint density of $X$ and $Y$.
            	\begin{proof}[\underline{\textbf{Solution}}] $\newline$
            
            	\end{proof}

	\item[e)] Find $EXY$ by smoothing, using the conditional expectation you found in part (c).
            	\begin{proof}[\underline{\textbf{Solution}}] $\newline$
            
            	\end{proof}
\end{enumerate}



\section{Problem 6.6}
Let $\mu$ be Lebesgue measure on $\bbR$ and let $\nu$ be counting measure on $\{0,1,\dots\}^2$.
Suppose the joint density of $X$ and $Y$ with respect to $\mu \times \nu$ is given by
$$ p(x,y_1,y_2) = x^2(1-x)^{y_1+y_2} $$
for $x\in(0,1)$, $y_1 = 0,1,2,\dots,$ and $y_2 = 0,1,2,\dots$.
\begin{enumerate}
	\item[a)] Find the marginal density of $X$.
	       	\begin{proof}[\underline{\textbf{Solution}}] $\newline$
		
		\end{proof}
	
	\item[b)] Find the conditional density of $X$ given $Y=y$ (i.e., given $Y_1=y_1$ and $Y_2=y_2$).
            	\begin{proof}[\underline{\textbf{Solution}}] $\newline$
            
            	\end{proof}

	\item[c)] Find $E[X|Y]$ and $E[X^2|Y]$. \\
		\underline{Hint}: The formula
		$$ \int_0^1 x^{\alpha-1}(1-x)^{\beta-1}dx \frac{\Gamma(\alpha)\Gamma(\beta)}{\Gamma(\alpha+\beta)} $$
		may be useful.
            	\begin{proof}[\underline{\textbf{Solution}}] $\newline$
            
            	\end{proof}

	\item[d)] Find $E[1/(4+Y_1+Y_2)]$. \\
		\underline{Hint}: Find $EX$ using the density in part (a) and find an expression for $EX$ using smoothing and the conditional expectation in part (c).
            	\begin{proof}[\underline{\textbf{Solution}}] $\newline$
            
            	\end{proof}
\end{enumerate}



\section{Problem 6.11}
Suppose $X$ and $Y$ are independent, both absolutely continuous with common density $f$.
Let $M=\max\{X,Y\}$ and $Z=\min\{X,Y\}$.
Determine the conditional distribution for the pair $(X,Y)$ given $(M,Z)$.
\begin{proof}[\underline{\textbf{Solution}}] $\newline$
		
\end{proof}



\section{Problem 6.12}
Let $X$ and $Y$ be independent exponential variables with failure rate $\lambda$, so the common marginal density is $\lambda e^{-\lambda x}, ~ x>0$. Let $T=X+Y$. 
Give a formula expressing $E[f(X,Y)|T=t]$ as a one-dimensional integral. \\
\underline{Hint}: Review the initial example on sufficiency in Section 3.2.
\begin{proof}[\underline{\textbf{Solution}}] $\newline$
		
\end{proof}



\section{Problem 6.14}
Let $X$ and $Y$ be absolutely continuous with density $p(x,y) = e^{-x}$, if $0<y<x;~p(x,y) = 0$, otherwise.
\begin{enumerate}
	\item[a)] Find the marginal densities of $X$ and $Y$.
	       	\begin{proof}[\underline{\textbf{Solution}}] $\newline$
		
		\end{proof}
	
	\item[b)] Compute $EY$ and $EY^2$ integrating against the marginal density of $Y$.
            	\begin{proof}[\underline{\textbf{Solution}}] $\newline$
            
            	\end{proof}

	\item[c)] Find the conditional density of $Y$ given $X=x$, and use it to compute $E[Y|X]$ and $E[Y^2|X]$.
            	\begin{proof}[\underline{\textbf{Solution}}] $\newline$
            
            	\end{proof}

	\item[d)] Find the expectations of $E[Y|X]$ and $E[Y^2|X]$ integrating against the marginal density of $X$.
            	\begin{proof}[\underline{\textbf{Solution}}] $\newline$
            
            	\end{proof}
\end{enumerate}















\end{document}
