\documentclass[11pt]{book}

\usepackage{fancyhdr}
\usepackage{extramarks}
\usepackage{amsmath}
\usepackage{amsthm}
\usepackage{amsfonts}
\usepackage{tikz}
\usepackage[plain]{algorithm}
\usepackage{algpseudocode}

\usetikzlibrary{automata,positioning}

%%%%%%%%%%%%%%%%%%%%%%%%
%%% Useful packages and commands
%%%%%%%%%%%%%%%%%%%%%%%%
% packages
\usepackage{amsmath,amssymb,amsthm}
%\usepackage{times}
%\usepackage{setspace}
\usepackage{indentfirst}
\usepackage{epsfig}
\usepackage{graphicx}
\usepackage{latexsym}
\usepackage{amscd}
\usepackage{multirow}
\usepackage{array}
\usepackage{caption}
\usepackage{rotating}
\usepackage{subfig}
\usepackage{color}
\usepackage{natbib}
\usepackage{lscape}
\usepackage{graphics}
\usepackage{enumerate}
%\usepackage{fancyvrb}
%\usepackage{mathtools}
\usepackage{verbatim}
\usepackage{afterpage}
%\usepackage[ruled,vlined]{algorithm2e}
\usepackage{hyperref}
\usepackage[flushleft]{threeparttable}
\usepackage{rotating}
\usepackage{kotex}   % for Korean

% new commands
%\DeclarePairedDelimiter\abs{\lvert}{\rvert}
%\DeclarePairedDelimiter\norm{\lVert}{\rVert}
\long\def\comment#1{}

%\newtheorem*{thm}{Theorem}
\newtheorem{thm}{Theorem}[section]
\newtheorem{cor}[thm]{Corollary}
\newtheorem{lem}[thm]{Lemma}
\newcommand{\rb}[1]{\raisebox{-.5em}[0pt]{#1}}
% \renewcommand{\baselinestretch}{1.8}
\renewcommand{\mid}{\, | \ }
\newcommand{\eighth}{{\textstyle \frac{1}{8}}}

\def \bY { \mathbf{ Y } }
\def \bX { \mathbf{ X } }
\def \bU { \mathbf{ U } }
\def \bmu { \boldsymbol{ \mu } }
\def \bSigma { \boldsymbol{ \Sigma } }
\def \bphi { \boldsymbol{ \phi } }
\def \bepsilon { \boldsymbol{ \epsilon } }
\def \bD { \boldsymbol{\mathcal{D}} }

\newcommand{\eqdis}{\overset{\mathrm{d}}{=\joinrel=}}
\newcommand{\ba}{\mbox{\boldmath $a$}}
\newcommand{\bg}{\mbox{\boldmath $g$}}
\newcommand{\bx}{\mbox{\boldmath $x$}}
\newcommand{\by}{\mbox{\boldmath $y$}}
\newcommand{\bd}{\mbox{\boldmath $d$}}
\newcommand{\bff}{\mbox{\boldmath $f$}}
\newcommand{\bz}{\mbox{\boldmath $z$}}
\newcommand{\bu}{\mbox{\boldmath $u$}}
\newcommand{\bv}{\mbox{\boldmath $v$}}
\newcommand{\bW}{\mbox{\boldmath $W$}}
\newcommand{\bI}{\mbox{\boldmath $I$}}
\newcommand{\bJ}{\mbox{\boldmath $J$}}
\newcommand{\bL}{\mbox{\boldmath $L$}}
\newcommand{\bQ}{\mbox{\boldmath $Q$}}
\newcommand{\bZ}{\mbox{\boldmath $Z$}}
\newcommand{\bV}{\mbox{\boldmath $V$}}
\newcommand{\bG}{\mbox{\boldmath $G$}}
\newcommand{\bdm}{\begin{displaymath}}
\newcommand{\edm}{\end{displaymath}}
\newcommand{\bnu}{\mbox{\boldmath $\nu$}}
\newcommand{\btau}{\mbox{\boldmath $\tau$}}
\newcommand{\biota}{\mbox{\boldmath $\iota$}}
\newcommand{\bbeta}{\mbox{\boldmath $\beta$}}
\newcommand{\bomega}{\mbox{\boldmath $\omega$}}
\newcommand{\btheta}{\mbox{\boldmath $\theta$}}
\newcommand{\bep}{\mbox{\boldmath $\epsilon$}}
\newcommand{\bdelta}{\mbox{\boldmath $\delta$}}
\newcommand{\balpha}{\mbox{\boldmath $\alpha$}}
\newcommand{\bxi}{\mbox{\boldmath $\xi$}}
\newcommand{\bgamma}{\mbox{\boldmath $\gamma$}}
\newcommand{\bOmega}{\mbox{\boldmath $\Omega$}}
\newcommand{\bPi}{\mbox{\boldmath $\Pi$}}
\newcommand{\bzeta}{\mbox{\boldmath $\zeta$}}
\newcommand{\bpsi}{\mbox{\boldmath $\psi$}}
\newcommand{\bPsi}{\mbox{\boldmath $\Psi$}}
\newcommand{\bl}{\mbox{\boldmath $l$}}
\newcommand{\C}{{\rm Cov}}
\newcommand{\bH}{\bold H}
\newcommand{\blambda}{\mbox{\boldmath $\lambda$}}
\newcommand{\bbh}{\bld h}
\newcommand{\calA}{\mathcal{A}}
\newcommand{\calB}{\mathcal{B}}
\newcommand{\calE}{\mathcal{E}}
\newcommand{\calX}{\mathcal{X}}
\newcommand{\calY}{\mathcal{Y}}
\newcommand{\bbR}{\mathbb{R}}
\newcommand{\union}{\bigcup}
\newcommand{\intersect}{\bigcap}
\newcommand{\eqdef}{\overset{\mathrm{def}}{=}}

%
% Basic Document Settings
%

\topmargin=-0.45in
\evensidemargin=0in
\oddsidemargin=0in
\textwidth=6.5in
\textheight=9.0in
\headsep=0.25in

\linespread{1.2}

\pagestyle{fancy}
\lhead{\leftmark}
%\chead{\hmwkClass\ (\hmwkClassInstructor\ \hmwkClassTime): \hmwkTitle}
\rhead{\authorName}
%\lfoot{\lastxmark}
\cfoot{\thepage}

\renewcommand\headrulewidth{0.4pt}
\renewcommand\footrulewidth{0.4pt}

\setlength\parindent{0pt}

%%
%% Create Problem Sections
%%
%
%\newcommand{\enterProblemHeader}[1]{
%    \nobreak\extramarks{}{Problem \arabic{#1} continued on next page\ldots}\nobreak{}
%    \nobreak\extramarks{Problem \arabic{#1} (continued)}{Problem \arabic{#1} continued on next page\ldots}\nobreak{}
%}
%
%\newcommand{\exitProblemHeader}[1]{
%    \nobreak\extramarks{Problem \arabic{#1} (continued)}{Problem \arabic{#1} continued on next page\ldots}\nobreak{}
%    \stepcounter{#1}
%    \nobreak\extramarks{Problem \arabic{#1}}{}\nobreak{}
%}
%
%\setcounter{secnumdepth}{0}
%\newcounter{partCounter}
%\newcounter{homeworkProblemCounter}
%\setcounter{homeworkProblemCounter}{1}
%\nobreak\extramarks{Problem \arabic{homeworkProblemCounter}}{}\nobreak{}
%
%%
%% Homework Problem Environment
%%
%% This environment takes an optional argument. When given, it will adjust the
%% problem counter. This is useful for when the problems given for your
%% assignment aren't sequential. See the last 3 problems of this template for an
%% example.
%%
%\newenvironment{homeworkProblem}[1][-1]{
%    \ifnum#1>0
%        \setcounter{homeworkProblemCounter}{#1}
%    \fi
%    \section{Problem \arabic{homeworkProblemCounter}}
%    \setcounter{partCounter}{1}
%    \enterProblemHeader{homeworkProblemCounter}
%}{
%    \exitProblemHeader{homeworkProblemCounter}
%}

%
% Homework Details
%   - Title
%   - Due date
%   - Class
%   - Section/Time
%   - Instructor
%   - Author
%

%\newcommand{\hmwkTitle}{}
%\newcommand{\hmwkClass}{Theoretical Statistics: Topics for a Core Course\\ Solution}
%\newcommand{\hmwkClassTime}{}
%\newcommand{\hmwkClassInstructor}{Robert W. Keener}

\newcommand{\courseTitle}{Theoretical Statistics: Topics for a Core Course}
\newcommand{\bookAuthor}{Robert W. Keener}
\newcommand{\authorName}{Hyunsung Kim}



%
% Title Page
%

\title{
    \vspace{1.5in}
%    \textmd{\textbf{\hmwkClass:\ \hmwkTitle}}\\
    \textmd{\textbf{\courseTitle}}\\
%    \normalsize\vspace{0.1in}\hmwkDueDate\\
    \vspace{0.1in}\large{\textit{\bookAuthor}}
    \vspace{2in}
}
\date{Update: \today}
\author{
    {\sc \authorName} \\
    Department of Statistics\\
    Chung-Ang University
}



\renewcommand{\part}[1]{\textbf{\large Part \Alph{partCounter}}\stepcounter{partCounter}\\}

%
% Various Helper Commands
%

% Useful for algorithms
\newcommand{\alg}[1]{\textsc{\bfseries \footnotesize #1}}

% For derivatives
\newcommand{\deriv}[1]{\frac{\mathrm{d}}{\mathrm{d}x} (#1)}

% For partial derivatives
\newcommand{\pderiv}[2]{\frac{\partial}{\partial #1} (#2)}

% Integral dx
\newcommand{\dx}{\mathrm{d}x}
\newcommand{\dmu}{\mathrm{d}\mu}
\newcommand{\dP}{\mathrm{d}P}

% Alias for the Solution section header
\newcommand{\solution}{\textbf{\large Solution}}

% Probability commands: Expectation, Variance, Covariance, Bias
\newcommand{\E}{\mathrm{E}}
\newcommand{\Var}{\mathrm{Var}}
\newcommand{\Cov}{\mathrm{Cov}}
\newcommand{\Bias}{\mathrm{Bias}}

% Specify level of toc(table of content)
\setcounter{tocdepth}{0}

% Setting for blank page appeared when begin new chapter
\let\cleardoublepage=\clearpage



\newtheorem{theorem}{Theorem}[section]
\newtheorem{proposition}[theorem]{Proposition}
\newtheorem{corollary}[theorem]{Corollary}
\newtheorem{lemma}[theorem]{Lemma}
\newtheorem{definition}[theorem]{Definition}

\theoremstyle{definition}
\newtheorem{remark}[theorem]{Remark}
\newtheorem{example}[theorem]{Example}
%\newtheorem{definition}{Definition}[section]


%%% Main part
\begin{document}

% Title page
\maketitle

% Table of Contents
\tableofcontents

% Preface
\addcontentsline{toc}{chapter}{Preface}
\chapter*{Preface}
%\section{Preface}
This note contains summaries of the textbook, {\em Theoretical Statistics: Topics for a Core Course}, and it was created by Hyunsung Kim, who is a Ph.D. student.
I made it when I study a theoretical statistics based on this textbook on my own.

\section*{Textbook}
\begin{itemize}
	\item Keener, {\em Theoretical Statistics: Topics for a Core Course}.
\end{itemize}

\section*{Reference}
\begin{itemize}
	\item Durrett, {\em Probability: Theory and Examples, 5th edition}.
	\item Royden, {\em Real Analysis, 4th edition}.
\end{itemize}

% Test file
%\chapter{Math symbol test}

\section{Theorem symbol}

\begin{theorem}[Pythagorean theorem]
\label{pythagorean}
This is a theorem about right triangles and can be summarised in the next 
equation 
\[ x^2 + y^2 = z^2 \]
\end{theorem}
\begin{proof}
	dkdkdk
\end{proof}

And a consequence of theorem \ref{pythagorean} is the statement in the next 
corollary.

\begin{corollary}
There's no right rectangle whose sides measure 3cm, 4cm, and 6cm.
\end{corollary}

You can reference theorems such as \ref{pythagorean} when a label is assigned.

\begin{proposition}[Consistnecy]
ddfafa
\end{proposition}

\begin{lemma}
Given two line segments whose lengths are \(a\) and \(b\) respectively there is a 
real number \(r\) such that \(b=ra\).
\end{lemma}

\begin{remark}
This statement is true, I guess.
\end{remark}

\begin{definition}[Fibration]
A fibration is a mapping between two topological spaces that has the homotopy lifting property for every space \(X\).
\end{definition}

\begin{example}[Continuous prob]
This statement is true, I guess.
\end{example}

% Chapter 1. Probability and Measure
\chapter{Probability and Measure}

%% # 1.1
\section{Problem 1.1}
Prove (1.1). If measurable sets $B_n, ~ n \ge 1$, are increasing, with $B = \union_{n=1}^\infty B_n$, called the limit of the sequence, then
$$ \mu(B) = \lim_{n \rightarrow \infty} \mu(B_n). $$
\begin{proof}[\underline{\textbf{Solution}}] $\newline$
    First, we showed that $A_n$'s are disjoint.
    If $j < k$, then $B_j \subseteq B_{k-1} $. \\
    Since $A_j \subset B_j \subseteq B_{k-1}$ and $A_k \subset B_{k-1}^c$, $A_j$ and $A_k$ are disjoint.
    
    Also $B_n = \union_{j=1}^n A_j$ and $\union_{n=1}^\infty A_n = B$,
    $$ \mu(B) = \sum_{i=1}^\infty \mu(A_i) = \lim_{n \rightarrow \infty} \sum_{i=1}^n \mu(A_i) =  \lim_{n \rightarrow \infty} \mu\left(\union_{i=1}^n A_i\right) = \lim_{n \rightarrow \infty} \mu(B_n). $$
\end{proof}


%% # 1.8
\section{Problem 1.8}
Prove {\em Boole's inequality}: For any events $B_1, B_2, \dots $,
$$ P\left(\union_{i \ge 1} B_i\right) \le \sum_{i \ge 1}P(B_i). $$
\begin{proof}[\underline{\textbf{Solution}}] $\newline$
    Let $B = \union_{i=1}^\infty B_i$, then $1_B \le \sum 1_{B_i}$. Then by Fubini's theorem,
    $$ P(B) = \int 1_B dP \le \int \sum 1_{B_i} dP = \sum \int 1_{B_i} dP = \sum P(B_i).$$
\end{proof}


%% # 1.10
\section{Problem 1.10}
Let $\mu$ and $\nu$ be measures on $(\calE, \calB)$.
\begin{itemize}
	\item[a)] Show that the sum $\eta$ defined by $\eta(B) = \mu(B) + \nu(B)$ is also a measure.
	\begin{proof}[\underline{\textbf{Solution}}] $\newline$
		We should check following 2 conditions.
		\begin{itemize}
			\item[(i)] For arbitrary set $A \in \calB$, $\mu(A) \ge 0$ and $\nu(A) \ge 0$. $\Rightarrow \eta(A) = \mu(A) + \nu(A) \ge 0.$ \\
            		$\therefore \eta: \calB \rightarrow [0, \infty].$
			\item[(ii)] For disjoint set $B_1, B_2, \dots \subset \calB$, then
            	\begin{align*}
            		\eta \left( \union_{i=1}^\infty B_i \right) &= \mu\left(\union B_i\right) + \nu\left(\union B_i\right) \\
            									  &= \sum \mu(B_i) + \sum \nu(B_i) \\
            									  &= \sum \left\{ \mu(B_i) + \nu(B_i) \right\} \\
            									  &= \sum \eta(B_i)
            	\end{align*}
		\end{itemize}
            	$\therefore \eta$ is a measure.
	\end{proof}
	
	\item[b)] If $f$ is a non-negative measurable function, show that
	$$\int f d\eta = \int f d\mu + \int f d\nu.$$
	\begin{proof}[\underline{\textbf{Solution}}] $\newline$
		We show it by 2 stage.
		\begin{itemize}
			\item[(i)] Let $f = \sum_{i=1}^n a_i 1_{A_i}$, the non-negative simple function. Then,
                    		\begin{align*}
                    			\int f d\eta &= \int \sum_{i=1}^n a_i 1_{A_i} d\eta = \sum a_i\eta(A_i) \\
                    					 &= \sum a_i \left\{ \mu(A_i) + \nu(A_i) \right\}  = \int f d\mu + \int f d\nu.
                    		\end{align*}
			\item[(ii)] For general case, let $f_n$ is the sequence of non-negative simple functions increasing to $f$. (i.e. $f_1 \le f_2 \le \dots \le f$)
                    		\begin{align*}
                    			\int f d\eta &= \lim_{n \to \infty} \int f_n d\eta = \lim_{n \to \infty} \left( \int f_n d\mu + \int f_n d\nu \right)  \\
                    					&= \lim_{n \to \infty} \int f_n d\mu + \lim_{n \to \infty} \int f_n d\nu = \int f d\mu + \int f d\nu.
                    		\end{align*}
		\end{itemize}
	\end{proof}
\end{itemize}


%% 1.11
\section{Problem 1.11}
Suppose $f$ is the simple function $1_{(1/2, \pi]} + 21_{(1, 2]}$, and let $\mu$ be a measure on $\bbR$ with $\mu\{(0,a^2]\} = a, ~ a > 0$. Evaluate $\int f d\mu.$
\begin{proof}[\underline{\textbf{Solution}}] $\newline$
By the integral of simple function and the finite additivity, it can be simply computed as
\begin{align*}
	\int f d\mu &= \mu\{(1/2, \pi] \} + 2\mu\{(1,2]\} \\
			&= \left[ \mu\{(0, \pi]\} - \mu\{(1/2, \pi] \} \right] + 2\left[ \mu\{ (0, 2] \} -  \mu\{(1,2]\} \right]  ~~ \text{(finite additivity)}\\
			&= \left( \sqrt{\pi} - 1/\sqrt{2} \right) + 2\left( \sqrt{2} - 1 \right)
\end{align*}
\end{proof}


%% 1.12
\section{Problem 1.12}
Suppose that $\mu\{ (0, a) \} = a^2$ for $a > 0$ and that $f$ is defined by
$$ f(x) = \begin{cases}
    0, ~~~ x \le 0, \\
    1, ~~~ 0 < x < 2, \\
    \pi, ~~~ 2 \ge x < 5, \\
    0, ~~~ x \ge 5.
\end{cases}$$
Compute $\int f d\mu.$
\begin{proof}[\underline{\textbf{Solution}}] $\newline$
Since $f = 1_{(0, 2)} + 1_{[2, 5)}$ is simple, the integral can be computed as
\begin{align*}
	\int f d\mu &= \mu\{ (0, 2) \} + \pi \mu\{ [2, 5) \} \\
			&= \mu\{ (0, 2) \} + \pi \left[ \mu\{ (0, 5) \} - \mu\{ (0, 2) \} \right] \\
			&= 4 - \pi(25-4)
\end{align*}
\end{proof}


%% 1.13
\section{Problem 1.13}
Define the function $f$ by
$$ f(x) = \begin{cases}
    x, ~~~ 0 \le x \le 1, \\
    0, ~~~ \text{otherwise.}
\end{cases}$$
Find simple functions $f_1 \le f_2 \le \cdots$ increasing to $f$ (i.e. $f(x) = \limn f_n(x)$ for all $x \in \bbR$).
Let $\mu$ be Lebesgue measure on $\bbR$.
Using our formal definition of an integral and the faProblemct that $\mu\big( (a, b] \big) = b-a$ whenever $b > a$ (this might be used to formally define Lebesgue measure), show that $\int f d\mu = 1/2$.
\begin{proof}[\underline{\textbf{Solution}}] $\newline$
Let $f_n = \lfloor 2^nx \rfloor / 2^n$ for $0 < x \le 1$ and 0 otherwise. ($\lfloor y \rfloor$ is a floor function.)
Then,
\begin{align*}
    f_1(x) &= \lfloor 2x \rfloor / 2 = \begin{cases}
    0, ~~~ x < 1/2, \\
    1/2, ~~~ 1/2 \le x < 1, \\
    1, ~~~ x = 1
    \end{cases} \\
    f_2(x) &= \lfloor 2^2x \rfloor / 2^2 = \begin{cases}
    0, ~~~ x < 1/2^2, \\
    1/2^2, ~~~ 1/2^2 \le x < 2/2^2, \\
    2/2^2, ~~~ 2/2^2 \le x < 3/2^2, \\
    3/2^2, ~~~ 3/2^2 \le x < 1, \\
    1, ~~~ x = 1
    \end{cases}\\
    &\vdots \\
    f_n(x) &= \lfloor 2^nx \rfloor / 2^n = \begin{cases}
    0, ~~~ x < 1/2^n, \\
    1/2^n, ~~~ 1/2^n \le x < 2/2^n, \\
    \vdots \\
    (2^n-1)/2^n, ~~~ (2^n-1)/2^n \le x < 1, \\
    1, ~~~ x = 1
    \end{cases}\\
\end{align*}
\begin{align*}
\therefore \int f_n d\mu &= \frac{1}{2^n} \left( \frac{1}{2^n} + \frac{2}{2^n} + \cdots + \frac{2^n-1}{2^n} \right) \\
		    &= \frac{1+2+\cdots + (2^n-1)}{4^n} \\
		    &= \frac{2^n(2^n-1)}{2 \cdot 4^n} \\
		    &\longrightarrow \frac{1}{2}.
\end{align*}

{\color{blue} \underline{My solution} \\
Let the simple function $f_n = \sum_{i=1}^n \frac{i}{n}1_{(\frac{i-1}{n}, \frac{i}{n} ]}$. Then,
$$ \int f_n d\mu = \sum_{i=1}^n \frac{i}{n} \frac{1}{n} = \frac{1}{n^2} \frac{n(n+1)}{2} \longrightarrow \frac{1}{2}.$$
}
\end{proof}


%% 1.16
\section{Problem 1.16}
Define $F(a-) = \lim_{x\uparrow a}F(x)$. Then, if $F$ is non-decreasing, $F(a-) = \limn F(a-1/n)$.
Use (1.1)[Continuity of measure] to show that if a random variable $X$ has cumulative distribution function $F_X$,
$$ P(X < a) = F_X(a-). $$
Also, show that
$$ P(X=a) = F_X(a) - F_X(a-). $$
\begin{proof}[\underline{\textbf{Solution}}] $\newline$
\begin{itemize}
    \item[(i)] Let $B_n = \{ X \le a-1/n \}$ and $\union_{n=1}^\infty B_n = \{ X < a \}$.
    By continuity of measure,
    $$ P(B) = P(X < a) = \limn P(X \le a-1/n) = \limn F_X(a-1/n) = F_X(a-). $$
    \item[(ii)] Since $\{ X<a \}$ and $\{X=a\}$ are disjoint with union $\{X \le a\}$,
    $$ P(X < a) + P(X = a) = P(X \le a).$$
    $$\therefore P(X = a) = F_X(a) - F_X(a-).$$
\end{itemize}
\end{proof}


%% 1.17
\section{Problem 1.17}
Suppose $X$ is a geometric random variable with mass function
$$ p(x) = P(X=x) = \theta(1-\theta)^x, ~~~ x= 0,1,\dots, $$
where $\theta \in (0,1)$ is a constant. Find the probability that $X$ is even.
\begin{proof}[\underline{\textbf{Solution}}]
\begin{align*}
P(X\text{ is even}) &= P(X=0) + P(X=2) + P(X=4) \cdots \\
			     &= \theta + \theta(1-\theta)^2 + \theta(1-\theta)^4 + \cdots \\
			     &= \frac{\theta}{1-(1-\theta)^2} \\
			     &= \frac{1}{2-\theta}
\end{align*}
\end{proof}


%% 1.18
\section{Problem 1.18}
Let $X$ be a function mapping $\calE$ into $\bbR$.
Recall that if $B$ is a subset of $\bbR$, then $X^{-1}(B) = \{e \in \calE : X(e) \in B\}.$
Use this definition to prove that
$$ X^{-1}(A \cap B) = X^{-1}(A) \cap X^{-1}(B), $$
$$ X^{-1}(A \cup B) = X^{-1}(A) \cup X^{-1}(B), $$
and
$$ X^{-1}\left(\union_{i=0}^\infty A_i \right) = \union_{i=0}^\infty X^{-1}(A_i). $$
\begin{proof}[\underline{\textbf{Solution}}] $\newline$
\begin{itemize}
\item[(i)] \begin{align*}
    e \in X^{-1}(A \cap B) &\Leftrightarrow X(e) \in A \cap B \\
    				&\Leftrightarrow X(e) \in A \text{ and } X(e) \in B \\
    				&\Leftrightarrow e \in X^{-1}(A) \text{ and } e \in X^{-1}(B) \\
    				&\Leftrightarrow e \in X^{-1}(A) \cap X^{-1}(B).
    \end{align*}

\item[(ii)] Similarly,
    \begin{align*}
    e \in X^{-1}(A \cup B) &\Leftrightarrow X(e) \in A \cup B \\
    				&\Leftrightarrow X(e) \in A \text{ or } X(e) \in B \\
    				&\Leftrightarrow e \in X^{-1}(A) \text{ or } e \in X^{-1}(B) \\
    				&\Leftrightarrow e \in X^{-1}(A) \cup X^{-1}(B).
    \end{align*}

\item[(iii)] \begin{align*}
    e \in X^{-1}(\union_{i=0}^\infty A_i) &\Leftrightarrow X(e) \in \union_{i=0}^\infty A_i \\
    				&\Leftrightarrow X(e) \in A_i \text{ for some } i \\
    				&\Leftrightarrow e \in X^{-1}(A_i) \text{ for some } i \\
    				&\Leftrightarrow e \in  \union_{i=0}^\infty X^{-1}(A_i).
    \end{align*}

\end{itemize}
\end{proof}




\chapter{Probability and Measure}

%% 1.1
\section{Measures}

\begin{definition}[$\sigma$-algebra]
A collection of $\calA$ of subsets of a set $\calX$ is a $\sigma$-algebra (or $\sigma$-field) if
\begin{enumerate}
    \item $\calX \in \calA$ and $\emptyset \in \calA$,
    \item If $A \in \calA$, then $A^c = \calX - A \in \calA$,
    \item If $A_1, A_2, \dots \in \calA$, then $\union_{i=1}^\infty A_i \in \calA$.
\end{enumerate}
\end{definition}

\begin{definition}[Measure]
A function $\mu$ on a $\sigma$-algebra $\calA$ of $\calX$ is a measure if
\begin{enumerate}
	\item For every $A \in \calA, ~ 0 \le \mu(A) \le \infty$; that is, $\mu: \calA \rightarrow [0, \infty]$.
	\item (Countable Additivity) If $A_1, A_2, \dots$ are disjoint elements of $\calA$, then
	$$ \mu\left( \union_{i=1}^\infty A_i \right) = \sum_{i=1}^\infty \mu(A_i).$$
\end{enumerate}
\end{definition}

\begin{example}
For a measure $\mu$ on a set $\calX$ and subsets $A \in \calX$, 
\begin{enumerate}
	\item $\mu$ is {\em counting measure} on $\calX$ if $\calX$ is countable, and define
	$$ \mu(A) = \# A = \text{number of points in }A,$$
	and also its $\sigma$-algebra $\calA$ be the {\em power set} of $\calX$, denoted $\calA = 2^\calX$.

	\item $\mu$ is {\em Lebesgue measure} on $\calX$ if $\calX = \bbR^n$, and define
	$$\mu(A) = \int \underset{A}{\cdots}  \int \dx_1\cdots dx_n,$$
	and for any set $A$ in a $\sigma$-algebra $\calA$ called the {\em Borel sets} of $\calX = \bbR^n$, and formally $\calA$ is the smallest $\sigma$-algebra that contains all open set in $\bbR^n$ ("rectangles").
	
	\item $\mu$ is {\em probability measure} on $\calX$ if $\mu(\calX) = 1$.
\end{enumerate}
\end{example}




\begin{remark}
If $\calA$ is a $\sigma$-algebra of subsets of $\calX$, the pair $(\calX, \calA)$ is called a {\em measurable space}, and if $\mu$ is a measure on $\calA$, the triple $(\calX, \calA, \mu)$ is called a {\em measure space}.
\end{remark}

\begin{proposition}[Continuity property of measures]
If measurable sets $B_n, ~ n \ge 1$, are increasing ($B_1 \subset B_2 \subset \cdots$), with $B = \union_{i=1}^\infty B_n$, called the limit of the sequence, then
	$$ \mu(B) = \lim_{n\rightarrow \infty} \mu(B_n). $$
If a measure $\mu$ is a probability measure, it also holds for decreasing sets ($B_1 \supset B_2 \supset \cdots$) and its limit $B = \intersect_{n=1}^\infty B_n$.
\begin{proof}
생략
\end{proof}
\end{proposition}


\begin{definition}[$\sigma$-finite] There is some notations and definitions about the measure.
\begin{enumerate}	
	\item A measure $\mu$ is finite if $\mu(\calX) < \infty$.
	\item A measure $\mu$ is $\sigma$-finite if $\exists A_1,A_2,\dots \in \calA$ with $\mu(A_i) < \infty ~ \forall i = 1, 2, \dots$ and $\union_{i=1}^\infty A_i = \calX$.
\end{enumerate}
\end{definition}

\begin{remark}
A probability measure is $\sigma$-finite.
\end{remark}


%% 1.2
\section{Integration}

\begin{example} The integral of "nice" function $f$ against a measure $\mu$ is written as $\int f d\mu$ or $\int f(x) d\mu(x)$.
    \begin{enumerate}
        \item If $\mu$ is {\em counting measure} on $\calX$, then the integral of $f$ against $\mu$ is
        $$ \int f d\mu = \sum_{x \in \calX} f(x). $$
        \item If $\mu$ is {\em Lebesgue measure} on $\bbR^n$, then the integral of $f$ against $\mu$ is
        $$ \int f d\mu = \int \cdots \int f(x_1, \dots, x_n) dx_1 \dots dx_n. $$
    \end{enumerate}
\end{example}

\begin{definition}[Measurable]
    If $(\calX, \calA)$ is a measurable space and $f$ is a real-valued function on $\calX$, then $f$ is measureble if
    $$ f^{-1}(B) \eqdef \{ x \in \calX : f(x) \in B \} \in \calA, $$
    for every Borel set $B$.
\end{definition}

\begin{definition}[Indicator function]
    The indicator function $1_A$ of a set $A$ is defined as
    $$ 
    1_A(x) = I\{x \in A\} = \begin{cases} 
                            1, & x \in A, \\
                            0, & x \not\in A.
                            \end{cases}
    $$
\end{definition}

\begin{proposition}\label{prop:indicator}
    The integral of the indicator function $1_A$ has following properties:
    \begin{enumerate}
        \item For any set $A \in \calA$, $\int 1_A d\mu = \mu(A)$.
        \item If $f$ and $g$ are non-negative measurable functions, and if $a$ and $b$ are positive constants,
        $$ \int (af + bg) d\mu = a\int f d\mu + b\int g d\mu. $$
        \item If $f_1 \le f_2 \le \cdots$ are non-negative measurable functions, and if $f(x) = \lim_{n \rightarrow \infty} f_n(x)$, then 
        $$ \int f d\mu = \lim_{n \rightarrow \infty} \int f_n d\mu. $$
    \end{enumerate}
    \begin{proof}
    생략
    \end{proof}
\end{proposition}

\begin{definition}[Simple function]
A function $f$ is simple if
$$ f = \sum_{i=1}^m a_i 1_{A_i}, $$
where $a_a, \dots, a_m$ are positive constants, and $A_1, \dots, A_m$ are sets in $\calA$
\end{definition}

\begin{remark}
By using 1 and 2 in Proposition \ref{prop:indicator}, the integral of simple function is obtained as
$$ \int \left( \sum_{i=1}^m a_i 1_{A_i} \right) d\mu = \sum_{i=1}^m a_i \mu(A_i). $$
\end{remark}

\begin{theorem}
If $f$ is non-negative and measurable, then there exist non-negative simple functions $f_1 \le f_2 \le \cdots$ with $f = \lim_{n \rightarrow \infty} f_n$.
\begin{proof}
생략
\end{proof}
\end{theorem}

\begin{definition}[Integrable]
% For non-negative measurable function $f$, and define $f^+(x) = \max\{f(x), 0\} and f^-(x) = -\min\{f(x), 0\}$.
% \begin{enumerate}
%     \item $f = f^+ - f^-$.
%     \item $|f| = f^+ + f^-$.
%     \item A function $f$ is integrable if $\int |f| d\mu < \infty$.
% \end{enumerate}
A measurable function $f$ is integrable if $\int |f| d\mu < \infty$.
\end{definition}


%% 1.3
\section{Events, Probabilities, and Random Variables}
\begin{definition}[Probability space]
If $\calE$ is a sample space,  $\calB$ is a $\sigma$-algebra of subsets of $\calX$, and $P$ is a probability measure, then the triple $(\calX, \calA, P)$ is called a  probability space.
Also $B \in \calB$ are called events, points $e \in \calE$ are called outcomes, and $P(B)$ is called the probability of B.
\end{definition}

\begin{definition}[Random variable]
A measurable function $X: \calE \rightarrow \bbR$ is called a random variable.
\end{definition}

\begin{definition}[Distribution]
The probability measure $P_X$ is defined by
$$ P_X(A) = P\left(\{ e \in \calE : X(e) \in A \}\right) \overset{\mathrm{def}}{=} P(X \in A), $$ 
for Borel sets $A$ is called distribution of $X$, and if we let $P_X := Q$, then it is denoted as $X \sim Q$, means that a random variable $X$ has distribution $Q$.
\end{definition}

\begin{definition}[Cumulative distribution function]
The cumulative distribution function of $X$ is defined by 
$$ F_X(x) = P(X \le x) = P\left(\{ e \in \calE : X(e) \le x \}\right) = P_X\left((-\infty, x]\right),$$
for $x \in \bbR.$
\end{definition}


%% 1.4
\section{Null Sets}

\begin{definition}[Null set]
For a measure $\mu$ on the measuruable space $(\calX, \calA)$, a set $N$ is called null with respect to $\mu$ if 
$$ \mu(N) = 0. $$
\end{definition}

\begin{definition}[Almost everywhere]
If a statement holds for $x \in \calX-N$ with $N$ null, the statement is said to hold almost everywhere (a.e.) or a.e. $\mu$.
\end{definition}

\begin{example}
    $f = 0$ a.e. $\mu$ $\Longleftrightarrow \mu\left( \{ x\in \calX : f(x) \ne 0 \} \right) = 0.$ 
\end{example}

\begin{remark}[Almost sure]
For the probability space, the statement holds a.e. if and only if $\mu(B^c) = 0$ or $\mu(B) = 1$.
Especially, it can be denoted that the statement holds {\em almost surely (a.s.)} or {\em with probability one}. 
\end{remark}

\begin{proposition}
    There are a few useful facts about integration:
    \begin{enumerate}
        \item If $f = 0 ~ (a.e. ~ \mu$), then $\int f d\mu = 0$.
        \item If $f \ge 0$ and $\int f d\mu=0$, then $f = 0 ~ (a.e. ~ \mu$).
        \item If $f = g ~ (a.e. ~ \mu)$, then $\int f d\mu = \int g d\mu$ for at least of the integral exists.
        \item If $\int 1_{(c,x)}f d\mu = 0$ for all $x > c$, then $f(x) = 0$ for $a.e. ~ x > c$. The constant $c$ here can be $-\infty$.
        \item If f and g are integrable and $f > g$, then $\int f d\mu > \int g d\mu$. \rm{(From 2)}
    \end{enumerate}
    \begin{proof}
    생략
    \end{proof}
\end{proposition}


%% 1.5
\section{Densities}

\begin{definition}[Absolutely continuity]
Let P and $\mu$ be measures on a $\sigma$-field $\calA$ of $\calX$. Then P is called absolutely continuous with respect to $\mu$, written $P \ll \mu$, if $P(A) = 0$ whenever $\mu(A) = 0$.
\end{definition}

\begin{theorem}[Radon-Nikodym]\label{thm:radon}
If a finite measure P is absolutely continuous with respect to a $\sigma$-finite measure $\mu$, then there exists a non-negative measurable function f such that 
$$ P(A) = \int_A f d\mu \eqdef \int f 1_A d\mu. $$
\end{theorem}

\begin{remark}
The function $f$ in Theorem \ref{thm:radon} is called Radon-Nikodym derivitive of $P$ with respect to $\mu$, or the {\em density} of $P$ with respect to $\mu$, denoted 
$$ f = \frac{dP}{d\mu}. $$
\end{remark}

\begin{remark}
If $X \sim P_X$ and $P_X$ is absolutely continuous with respect to $\mu$ with density $p = dP_X / d\mu$, it is called that {\em X has density p with repect to $\mu$}.
\end{remark}

\begin{example}[Absolutely continuous random variables]
    If a random variable X has density p with respect to Lebesgue measure on $\bbR$, then X or its distribution $P_X$ is called absolutely continuous with density p. Then, from the Radon-Nikodym theorem,
    $$ F_X(x) = P(X \le x) = P_X\left( (-\infty, x] \right) = \int_{-\infty}^x p(u)du $$
    And also by fundamental theorem of calculus, $p(x) = F_X'(x)$.
\end{example}

\begin{example}[Discrete random variables]
    Let $\calX_0$ be a countable subset of $\bbR$. The measure $\mu$ defined by
\end{example}


\end{document}